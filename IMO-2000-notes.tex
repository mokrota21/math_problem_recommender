% © Evan Chen
% Downloaded from https://web.evanchen.cc/

\documentclass[11pt]{scrartcl}
\usepackage[sexy]{evan}
\ihead{\footnotesize\textbf{\thetitle}}
\ohead{\footnotesize\href{http://web.evanchen.cc}{\ttfamily web.evanchen.cc},
    updated \today}
\title{IMO 2000 Solution Notes}
\date{\today}

\begin{document}

\maketitle

\begin{abstract}
This is a compilation of solutions
for the 2000 IMO.
The ideas of the solution are a mix of my own work,
the solutions provided by the competition organizers,
and solutions found by the community.
However, all the writing is maintained by me.

These notes will tend to be a bit more advanced and terse than the ``official''
solutions from the organizers.
In particular, if a theorem or technique is not known to beginners
but is still considered ``standard'', then I often prefer to
use this theory anyways, rather than try to work around or conceal it.
For example, in geometry problems I typically use directed angles
without further comment, rather than awkwardly work around configuration issues.
Similarly, sentences like ``let $\mathbb{R}$ denote the set of real numbers''
are typically omitted entirely.

Corrections and comments are welcome!
\end{abstract}

\tableofcontents
\newpage

\addtocounter{section}{-1}
\section{Problems}
\begin{enumerate}[\bfseries 1.]
\item %% Problem 1
Two circles $G_1$ and $G_2$ intersect at two points $M$ and $N$.
Let $AB$ be the line tangent to these circles at $A$ and $B$,
respectively, so that $M$ lies closer to $AB$ than $N$.
Let $CD$ be the line parallel to $AB$
and passing through the point $M$,
with $C$ on $G_1$ and $D$ on $G_2$.
Lines $AC$ and $BD$ meet at $E$; lines $AN$ and $CD$ meet at $P$;
lines $BN$ and $CD$ meet at $Q$.
Show that $EP = EQ$.

\item %% Problem 2
Let $a$, $b$, $c$ be positive real numbers with $abc = 1$.
Show that
\[
  \left( a - 1 + \frac 1b \right)
  \left( b - 1 + \frac 1c \right)
  \left( c - 1 + \frac 1a \right)
  \le 1.
\]

\item %% Problem 3
Let $n \ge 2$ be a positive integer
and $\lambda$ a positive real number.
Initially there are $n$ fleas on a horizontal line,
not all at the same point.
We define a move as choosing two fleas at some points $A$ and $B$,
with $A$ to the left of $B$,
and letting the flea from $A$ jump over the flea from $B$ to the point $C$
so that $\frac{BC}{AB} = \lambda$.

Determine all values of $ \lambda$ such that,
for any point $M$ on the line
and for any initial position of the $n$ fleas,
there exists a sequence of moves that will take
them all to the position right of $M$.

\item %% Problem 4
A magician has one hundred cards numbered $1$ to $100$.
He puts them into three boxes, a red one, a white one and a blue one,
so that each box contains at least one card.
A member of the audience draws two cards from two different boxes
and announces the sum of numbers on those cards.
Given this information,
the magician locates the box from which no card has been drawn.

How many ways are there to put the cards
in the three boxes so that the trick works?

\item %% Problem 5
Does there exist a positive integer $n$
such that $n$ has exactly 2000 distinct prime divisors
and $n$ divides $2^n + 1$?

\item %% Problem 6
Let $\ol{AH_1}$, $\ol{BH_2}$, and $\ol{CH_3}$ be the
altitudes of an acute triangle $ABC$.
The incircle $\omega$ of triangle $ABC$ touches the sides
$BC$, $CA$ and $AB$ at $T_1$, $T_2$ and $T_3$, respectively.
Consider the reflections of the lines $H_1H_2$, $H_2H_3$, and
$H_3H_1$ with respect to the lines $T_1T_2$, $T_2T_3$, and $T_3T_1$.
Prove that these images form a triangle whose vertices lie on $\omega$.

\end{enumerate}
\pagebreak

\section{Solutions to Day 1}
\subsection{IMO 2000/1, proposed by Sergei Berlov (RUS)}
\textsl{Available online at \url{https://aops.com/community/p354110}.}
\begin{mdframed}[style=mdpurplebox,frametitle={Problem statement}]
Two circles $G_1$ and $G_2$ intersect at two points $M$ and $N$.
Let $AB$ be the line tangent to these circles at $A$ and $B$,
respectively, so that $M$ lies closer to $AB$ than $N$.
Let $CD$ be the line parallel to $AB$
and passing through the point $M$,
with $C$ on $G_1$ and $D$ on $G_2$.
Lines $AC$ and $BD$ meet at $E$; lines $AN$ and $CD$ meet at $P$;
lines $BN$ and $CD$ meet at $Q$.
Show that $EP = EQ$.
\end{mdframed}
First, we have $\dang EAB = \dang ACM = \dang BAM$
and similarly $\dang EBA = \dang BDM = \dang ABM$.
Consequently, $\ol{AB}$ bisects $\angle EAM$ and $\angle EBM$,
and hence $\triangle EAB \cong \triangle MAB$.

\begin{center}
\begin{asy}
pair M = dir(120);
pair U = dir(170);
pair V = dir(10);
pair N = 0.3*V+0.7*U;

pair O_1 = circumcenter(M, U, N);
pair O_2 = circumcenter(M, V, N);

real r1 = abs(U-O_1);
real r2 = abs(V-O_2);
pair Se = (r1*O_2-r2*O_1)/(r1-r2);
path w1 = CP(O_1,N);
path w2 = CP(O_2,N);
filldraw(w1, opacity(0.1)+lightcyan, blue);
filldraw(w2, opacity(0.1)+lightcyan, blue);
pair A = IP(w1, CP(midpoint(Se--O_1), Se));
pair B = IP(w2, CP(midpoint(Se--O_2), Se));

draw(A--B, red);
pair C = -M+2*foot(O_1, M, M+B-A);
pair D = -M+2*foot(O_2, M, M+B-A);
draw(C--D, red);

pair E = extension(A, C, B, D);
pair P = extension(A, N, C, D);
pair Q = extension(B, N, C, D);
draw(C--E--D, blue);
draw(A--N--B, blue);
draw(P--E--Q, heavygreen);

pair T = extension(M, N, A, B);
draw(T--N, heavycyan);
draw(A--M--B, purple);
draw(E--M, red);

dot("$M$", M, dir(350));
dot("$N$", N, dir(260));
dot("$A$", A, dir(A));
dot("$B$", B, dir(B));
dot("$C$", C, dir(C));
dot("$D$", D, dir(D));
dot("$E$", E, dir(E));
dot("$P$", P, dir(260));
dot("$Q$", Q, dir(330));
dot("$T$", T, dir(160));

/* TSQ Source:

M = dir 120 R350
U := dir 170
V := dir 10
N = 0.3*V+0.7*U R260

O_1 := circumcenter M U N R-90
O_2 := circumcenter M V N R-90

!real r1 = abs(U-O_1);
!real r2 = abs(V-O_2);
Se := (r1*O_2-r2*O_1)/(r1-r2)
!path w1 = CP(O_1,N);
!path w2 = CP(O_2,N);
w1 0.1 lightcyan / blue
w2 0.1 lightcyan / blue
A = IP w1 CP midpoint Se--O_1 Se
B = IP w2 CP midpoint Se--O_2 Se

A--B red
C = -M+2*foot O_1 M M+B-A
D = -M+2*foot O_2 M M+B-A
C--D red

E = extension A C B D
P = extension A N C D R260
Q = extension B N C D R330
C--E--D blue
A--N--B blue
P--E--Q heavygreen

T = extension M N A B R160
T--N heavycyan
A--M--B purple
E--M red

*/
\end{asy}
\end{center}

Now it is well-known that $\ol{MN}$ bisects $\ol{AB}$
and since $\ol{AB} \parallel \ol{PQ}$
we deduce that $M$ is the midpoint of $\ol{PQ}$.
As $\ol{AB}$ is the perpendicular bisector of $\ol{EM}$,
it follows that $EP = EQ$ as well.
\pagebreak

\subsection{IMO 2000/2, proposed by Titu Andreescu (USA)}
\textsl{Available online at \url{https://aops.com/community/p354109}.}
\begin{mdframed}[style=mdpurplebox,frametitle={Problem statement}]
Let $a$, $b$, $c$ be positive real numbers with $abc = 1$.
Show that
\[
  \left( a - 1 + \frac 1b \right)
  \left( b - 1 + \frac 1c \right)
  \left( c - 1 + \frac 1a \right)
  \le 1.
\]
\end{mdframed}
Let $a = x/y$, $b = y/z$, $c = z/x$ for $x,y,z > 0$.
Then the inequality rewrites as
\[ (-x+y+z)(x-y+z)(x+y-z) \le xyz \]
which when expanded is equivalent to Schur's inequality.
Alternatively, if one wants to avoid appealing to Schur,
then the following argument works:
\begin{itemize}
  \ii At most one term on the left-hand side is negative;
  if that occurs we are done from $xyz > 0 > (-x+y+z)(x-y+z)(x+y-z)$.
  \ii If all terms in the left-hand side are nonnegative,
  let us denote $m = -x+y+z \ge 0$, $n = x-y+z \ge 0$, $p = x+y-z \ge 0$.
  Then it becomes
  \[ mnp \le \frac{(m+n)(n+p)(p+m)}{8} \]
  which follows by AM-GM.
\end{itemize}
\pagebreak

\subsection{IMO 2000/3, proposed by Sergei Shikh and Igor Voronovich (BLR)}
\textsl{Available online at \url{https://aops.com/community/p354112}.}
\begin{mdframed}[style=mdpurplebox,frametitle={Problem statement}]
Let $n \ge 2$ be a positive integer
and $\lambda$ a positive real number.
Initially there are $n$ fleas on a horizontal line,
not all at the same point.
We define a move as choosing two fleas at some points $A$ and $B$,
with $A$ to the left of $B$,
and letting the flea from $A$ jump over the flea from $B$ to the point $C$
so that $\frac{BC}{AB} = \lambda$.

Determine all values of $ \lambda$ such that,
for any point $M$ on the line
and for any initial position of the $n$ fleas,
there exists a sequence of moves that will take
them all to the position right of $M$.
\end{mdframed}
The answer is $\lambda \ge \frac{1}{n-1}$.

We change the problem by replacing the fleas
with \textbf{bowling balls} $B_1$, $B_2$, \dots, $B_n$ in that order.
Bowling balls aren't exactly great at jumping,
so each move can now be described as follows:
\begin{itemize}
\ii Select two indices $i < j$.
Then ball $B_i$ moves to $B_{i+1}$'s location,
$B_{i+1}$ moves to $B_{i+2}$'s location, and so on;
until $B_{j-1}$ moves to $B_j$'s location,
\ii Finally, $B_j$ moves some distance forward;
the distance is at most $\lambda \cdot |B_j B_i|$
and $B_j$ may not pass $B_{j+1}$.
\end{itemize}

\begin{claim*}
  If $\lambda < \frac{1}{n-1}$
  the bowling balls have bounded movement.
\end{claim*}
\begin{proof}
  Let $a_i \ge 0$ denote the initial distance
  between $B_i$ and $B_{i+1}$,
  and let $\Delta_i$ denote the distance travelled by ball $i$.
  Of course we have
  $\Delta_1 \le a_1 + \Delta_2$,
  $\Delta_2 \le a_2 + \Delta_3$,
  \dots,
  $\Delta_{n-1} \le a_{n-1} + \Delta_n$
  by the relative ordering of the bowling balls.
  Finally, distance covered by $B_n$ is always
  $\lambda$ times distance travelled by other bowling balls, so
  \begin{align*}
    \Delta_n &\le \lambda \sum_{i=1}^{n-1} \Delta_i
    \le \lambda \sum_{i=1}^{n-1}
    \left( \left( a_i + a_{i+1} + \dots + a_{n-1} \right)
      + \Delta_n \right) \\
    &= (n-1)\lambda \cdot \Delta_n + \sum_{i=1}^{n-1} i a_i
  \end{align*}
  and since $(n-1)\lambda > 1$, this gives an upper bound.
\end{proof}

\begin{remark*}
  Equivalently, you can phrase the proof without
  bowling balls as follows:
  if $x_1 < \dots < x_n$ are the positions of the fleas,
  the quantity
  \[ L = x_n - \lambda(x_1 + \dots + x_{n-1}) \]
  is a monovariant which never increases;
  i.e.\ $L$ is bounded above.
  Since $L > (1-(n-1)\lambda) x_n$, it follows
  $\lambda < \frac{1}{n-1}$ is enough to stop the fleas.
\end{remark*}

\begin{claim*}
When $\lambda \ge \frac{1}{n-1}$,
it suffices to always jump the leftmost flea
over the rightmost flea.
\end{claim*}
\begin{proof}
If we let $x_i$ denote the distance travelled by $B_1$
in the $i$th step,
then $x_i = a_i$ for $1 \le i \le n-1$
and $x_i = \lambda(x_{i-1} + x_{i-2} + \dots + x_{i-(n-1)})$.

In particular, if $\lambda \ge \frac{1}{n-1}$
then each $x_i$ is at least the average of the previous $n-1$ terms.
So if the $a_i$ are not all zero,
then $\{x_{n}, \dots, x_{2n-2}\}$ are all positive
and thereafter $x_i \ge \min \left\{ x_n, \dots, x_{2n-2} \right\} > 0$
for every $i \ge 2n-1$.
So the partial sums of $x_i$ are unbounded, as desired.
\end{proof}

\begin{remark*}
  Other inductive constructions are possible.
  Here is the idea of one of them,
  although the details are more complicated.

  We claim in general that given $n-1$ fleas at $0$
  and one flea at $1$,
  we can get all the fleas arbitrarily close to
  $\frac{1}{1-(n-1)\lambda}$
  (or as far as we want if $\lambda > \frac{1}{n-1}$.).
  The proof is induction by $n \ge 2$;
  for $n=2$ we get a geometric series.
  For $n \ge 3$, we leave one flea at zero
  and move the remainder close to $\frac{1}{1-(n-2)\lambda}$,
  then jump the last flea to
  $\frac{1+\lambda}{1-(n-2)\lambda}$.

  Now we're in the same situation,
  except we shifted $\frac{1}{1-(n-2)\lambda}$ right
  and have then scaled everything by
  $r = \frac{\lambda}{1-(n-2)\lambda}$.
  If we repeat this process again and check the geometric series,
  we see the fleas converge to
  \[ \frac{1}{1-(n-2)\lambda}
    \left( 1 + r + r^2 + r^3 + \dots \right)
    = \frac{1}{1-(n-2)\lambda} \cdot \frac{1}{1-r}
    = \frac{1}{1-(n-1)\lambda}. \]
\end{remark*}
\pagebreak

\section{Solutions to Day 2}
\subsection{IMO 2000/4, proposed by Sándor Dobos (HUN)}
\textsl{Available online at \url{https://aops.com/community/p354114}.}
\begin{mdframed}[style=mdpurplebox,frametitle={Problem statement}]
A magician has one hundred cards numbered $1$ to $100$.
He puts them into three boxes, a red one, a white one and a blue one,
so that each box contains at least one card.
A member of the audience draws two cards from two different boxes
and announces the sum of numbers on those cards.
Given this information,
the magician locates the box from which no card has been drawn.

How many ways are there to put the cards
in the three boxes so that the trick works?
\end{mdframed}
There are $2 \cdot 3! = 12$ ways, which amount to:
\begin{itemize}
  \ii Partitioning the cards modulo $3$, or
  \ii Placing $1$ alone in a box,
  $100$ alone in a second box,
  and all remaining cards in the third box.
\end{itemize}
These are easily checked to work so we prove they are the only ones.

\paragraph{First solution.}
We proceed by induction on $n \ge 3$ with the base case being immediate.

For the inductive step,
consider a working partition of $\{1, 2, \dots, n\}$.
Then either $n$ is in its own box; or
deleting $n$ gives a working partition of $\{1, 2, \dots, n-1\}$.
Similarly, either $1$ is in its own box; or
deleting $1$ gives a working partition of $\{2, 3, \dots, n\}$,
and we can reduce all numbers by $1$ to get
a working partition of $\{1, 2, \dots, n-1\}$.

Therefore, we only need to consider there cases.
\begin{itemize}
  \ii If $1$ and $n$ are both in their own box,
  this yields one type of solution we already found.

  \ii If $n$ is not in a box by itself,
  then by induction hypothesis the cards $1$ through $n-1$
  are either arranged mod $3$,
  or as $\{1\} \cup \{2,3,\dots,n-2\} \cup \{n-1\}$.
  \begin{itemize}
    \ii In the former mod $3$ situation,
    since $n + (n-3) = (n-1) + (n-2)$,
    so $n$ must be in the same box as $n-3$.
    \ii In the latter case and for $n > 4$,
    since $n + 1 = 2 + (n-1)$,
    $n$ must be in the same box as $1$.
    But now $n + 2 = (n-1) + 3$ for $n > 4$, contradiction.
  \end{itemize}

  \ii The case where $1$ is in a box by itself is analogous.
\end{itemize}
This exhausts all cases, completing the proof.

\paragraph{Second solution.}
Let $A$, $B$, $C$ be the sets of cards in the three boxes.
Then $A+B$, $B+C$, $C+A$ should be disjoint,
and contained in $\{3, 4, \dots, 199\}$.
On the other hand, we have the following famous fact.
\begin{lemma*}
  Let $X$ and $Y$ be finite nonempty sets of real numbers.
  We have $|X+Y| \ge |X|+|Y|-1$,
  with equality if and only if $X$ and $Y$ are arithmetic
  progressions with the same common difference,
  or one of $X$ and $Y$ is a singleton set.
\end{lemma*}
Putting these two together gives the estimates
\[ 197 \ge |A+B| + |B+C| + |C+A|
  \ge 2\left( |A|+|B|+|C| \right)-3 = 197. \]
So all the inequalities must be sharp.
Consequently we conclude that:
\begin{claim*}
  Either the sets $A$, $B$, $C$ are disjoint arithmetic progressions
  with the same common difference
  $d = \min_{x \neq y \text{ in same set}} |x-y|$,
  or two of the sets are two singleton.
  Moreover, $\{3, 4, \dots, 199\}
    = (A+B) \sqcup (B+C) \sqcup (C+A)$.
\end{claim*}
From here it is not hard to deduce the layouts above are the only ones,
but there are some details.
First, we make the preliminary observation that
$3=1+2$, $4=1+3$, $198=98+100$, $199=99+100$
and these numbers can't be decomposed in other ways;
thus from the remark about the disjoint union:
\begin{claim*}
  [Convenient corollary]
  The pairs $(1,2)$, $(1,3)$, $(98,100)$, $(99,100)$
  are all in different sets.
\end{claim*}
We now consider the four cases.
\begin{itemize}
  \ii If two of the boxes are singletons,
  it follows from the corollary that we should have $A = \{1\}$,
  $B = \{100\}$ and $C = \{2, \dots, 99\}$, up to permutation.
  \ii Otherwise $A$, $B$, $C$ are disjoint arithmetic
  progressions with the same common difference $d$.
  As two of $\{1,2,3,4\}$ are in the same box
  (by pigeonhole), we must have $d \le 3$.
  \begin{itemize}
    \ii If $d=3$, then no two elements of different residues
    modulo $3$ can be in the same box,
    so we must be in the first construction claimed earlier.

    \ii If $d=2$, then the convenient corollary
    tells us we may assume WLOG that $1 \in A$ and $2 \in B$,
    hence $3 \in C$
    (since $3 \notin A$ by convenient corollary,
    and $3 \notin B$ because common difference $2$).
    Thus we must have $A = \{1\}$, $B = \{2, 4, \dots, 100\}$
    and $C = \{3, 5, \dots 99\}$
    which does not work since $1+4 = 2+3$.
    Therefore there are no solutions in this case.

    \ii If $d=1$, then by convenient corollary
    the numbers $1$ and $2$ are in different sets,
    as are $99$ and $100$.
    So we must have $A = \{1\}$, $B = \{2, \dots, 99\}$, $C = \{100\}$
    which we have already seen is valid.
  \end{itemize}
\end{itemize}
\pagebreak

\subsection{IMO 2000/5, proposed by Valerii Senderov (RUS)}
\textsl{Available online at \url{https://aops.com/community/p354115}.}
\begin{mdframed}[style=mdpurplebox,frametitle={Problem statement}]
Does there exist a positive integer $n$
such that $n$ has exactly 2000 distinct prime divisors
and $n$ divides $2^n + 1$?
\end{mdframed}
Answer: Yes.

We say that $n$ is \emph{Korean} if $n \mid 2^n+1$.
First, observe that $n=9$ is Korean.
Now, the problem is solved upon the following claim:
\begin{claim*}
  If $n > 3$ is Korean,
  there exists a prime $p$ not dividing $n$
  such that $np$ is Korean too.
\end{claim*}
\begin{proof}
  I claim that one can take any primitive prime divisor $p$ of $2^{2n}-1$,
  which exists by Zsigmondy theorem.
  Obviously $p \neq 2$.
  Then:
  \begin{itemize}
    \ii Since $p \nmid 2^{\varphi(n)}-1$ it follows then that $p \nmid n$.
    \ii Moreover, $p \mid 2^n+1$ since $p \nmid 2^n-1$.
  \end{itemize}
  Hence $np \mid 2^n+1 \mid 2^{np} + 1$ by Chinese Theorem,
  since $\gcd(n,p) = 1$.
  % Thanks Wanlin for catching this
\end{proof}
\pagebreak

\subsection{IMO 2000/6, proposed by Lev and Tatiana Emelianovii (RUS)}
\textsl{Available online at \url{https://aops.com/community/p351094}.}
\begin{mdframed}[style=mdpurplebox,frametitle={Problem statement}]
Let $\ol{AH_1}$, $\ol{BH_2}$, and $\ol{CH_3}$ be the
altitudes of an acute triangle $ABC$.
The incircle $\omega$ of triangle $ABC$ touches the sides
$BC$, $CA$ and $AB$ at $T_1$, $T_2$ and $T_3$, respectively.
Consider the reflections of the lines $H_1H_2$, $H_2H_3$, and
$H_3H_1$ with respect to the lines $T_1T_2$, $T_2T_3$, and $T_3T_1$.
Prove that these images form a triangle whose vertices lie on $\omega$.
\end{mdframed}
We use complex numbers with $\omega$ the unit circle.
Let $T_1 = a$, $T_2 = b$, $T_3 = c$.
The main content of the problem is to show that
the triangle in question has vertices
$ab/c$, $bc/a$, $ca/b$
(which is evident from a good diagram).

Since $A = \frac{2bc}{b+c}$, we have
\[ H_1 = \half \left( \frac{2bc}{b+c} + a + a - a^2 \cdot
  \frac{2}{b+c} \right)
  = \frac{ab+bc+ca-a^2}{b+c}. \]
The reflection of $H_1$ over $\ol{T_1 T_2}$ is
\begin{align*}
  H_1^C &= a + b - ab \ol{H_1}
    = a + b - b \cdot \frac{ac+ab+a^2-bc}{a(b+c)} \\
  &= \frac{a(a+b)(b+c) - b(a^2+ab+ac-bc)}{a(b+c)}
    = \frac{c(a^2+b^2)}{a(b+c)}.
\end{align*}
Now, we claim that $H_1^C$ lies on the chord joining
$\frac{ca}{b}$ and $\frac{cb}{a}$;
by symmetry so will $H_2^C$
and this will imply the problem
(it means that the desired triangle has vertices
$ab/c$, $bc/a$, $ca/b$).
A cartoon of this is shown below.
\begin{center}
\begin{asy}
  pair A = dir(110);
  pair B = dir(210);
  pair C = dir(330);
  dot("$\frac{bc}{a}$", A, dir(10), blue);
  dot("$\frac{ca}{b}$", B, dir(140), blue);
  dot("$\frac{ab}{c}$", C, dir(50), blue);
  dot("$H_1^C$", 2*A-B, dir(A-B));
  dot("$H_2^C$", 2*B-A, dir(B-A));
  dot("$H_2^A$", 2*B-C, dir(B-C));
  dot("$H_3^A$", 2*C-B, dir(C-B));
  dot("$H_3^B$", 2*C-A, dir(C-A));
  dot("$H_1^B$", 2*A-C, dir(A-C));
  draw( (2*A-B)--(2*B-A) );
  draw( (2*B-C)--(2*C-B) );
  draw( (2*C-A)--(2*A-C) );
\end{asy}
\end{center}
To see this, it suffices to compute
\begin{align*}
  H_1^C + \left( \frac{ca}{b} \right)\left( \frac{cb}{a} \right) \ol{H_1^C}
  &= \frac{c(a^2+b^2)}{a(b+c)}
    + c^2 \frac{\frac 1c \cdot \frac{a^2+b^2}{a^2b^2}}%
    {\frac1a\left( \frac{b+c}{bc} \right)} \\
  &= \frac{c(a^2+b^2)}{a(b+c)}
    + \frac{c(a^2+b^2)}{abc\inv(b+c)} \\
  &= \frac{c(a^2+b^2)}{a(b+c)} \left( \frac{b+c}{b} \right) \\
  &= \frac{c(a^2+b^2)}{ab}= \frac{ca}{b} + \frac{cb}{a}
\end{align*}
as desired.
\pagebreak


\end{document}
