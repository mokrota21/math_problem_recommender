\textsl{Available online at \url{https://aops.com/community/p893744}.}
\begin{mdframed}[style=mdpurplebox,frametitle={Problem statement}]
Consider five points $A$, $B$, $C$, $D$ and $E$
such that $ABCD$ is a parallelogram and $BCED$ is a cyclic quadrilateral.
Let $\ell$ be a line passing through $A$.
Suppose that $\ell$ intersects the interior of the segment $DC$ at $F$
and intersects line $BC$ at $G$.
Suppose also that $EF = EG = EC$.
Prove that $\ell$ is the bisector of angle $ DAB$.
\end{mdframed}
Let $M$, $N$, $P$ denote the midpoints of $\ol{CF}$, $\ol{CG}$, $\ol{AC}$
(noting $P$ is also the midpoint of $\ol{BD}$).

By a homothety at $C$ with ratio $\half$,
we find $\ol{MNP}$ is the image of line $\ell \equiv \ol{AGF}$.

\begin{center}
\begin{asy}
pair C = dir(200);
pair D = dir(340);
pair E = dir(240);
pair K = E*E/C;
pair B = C*K/D;
pair A = B+D-C;
pair M = foot(E, C, D);
pair N = foot(E, C, B);
pair P = foot(E, B, D);
filldraw(unitcircle, opacity(0.1)+lightcyan, blue);
draw(B--C--D--cycle, blue);
filldraw(CP(E, C), opacity(0.1)+lightred, red+dotted);
pair F = 2*M-C;
pair G = 2*N-C;
draw(E--P, orange);
draw(E--M, orange);
draw(E--N, orange);
draw(C--G, blue);

draw(N--P, deepgreen);
draw(A--G, deepgreen);
draw(B--A--D, lightblue+dashed);
draw(A--C, lightblue+dashed);

dot("$C$", C, dir(180));
dot("$D$", D, dir(D));
dot("$E$", E, dir(E));
dot("$B$", B, dir(B));
dot("$A$", A, dir(A));
dot("$M$", M, dir(135));
dot("$N$", N, dir(N));
dot("$P$", P, dir(P));
dot("$F$", F, dir(F));
dot("$G$", G, dir(G));

/* TSQ Source:

C = dir 200 R180
D = dir 340
E = dir 240
K := E*E/C
B = C*K/D
A = B+D-C
M = foot E C D R135
N = foot E C B
P = foot E B D
unitcircle 0.1 lightcyan / blue
B--C--D--cycle blue
CP E C 0.1 lightred / red dotted
F = 2*M-C
G = 2*N-C
E--P orange
E--M orange
E--N orange
C--G blue

N--P deepgreen
A--G deepgreen
B--A--D lightblue dashed
A--C lightblue dashed

*/
\end{asy}
\end{center}

However, since we also have $\ol{EM} \perp \ol{CF}$
and $\ol{EN} \perp \ol{CG}$ (from $EF=EG=EC$)
we conclude $\ol{PMN}$ is the Simson line of $E$ with respect to $\triangle BCD$,
which implies $\ol{EP} \perp \ol{BD}$.
In other words, $\ol{EP}$ is the perpendicular bisector of $\ol{BD}$,
so $E$ is the midpoint of arc $\widehat{BCD}$.

Finally,
\begin{align*}
  \dang(\ol{AB}, \ell) &= \dang(\ol{CD}, \ol{MNP}) = \dang CMN = \dang CEN \\
  &= 90\dg - \dang NCE = 90\dg + \dang ECB
\end{align*}
which means that $\ell$ is parallel to a bisector of $\angle BCD$,
and hence to one of $\angle BAD$.
(Moreover since $F$ lies on the interior of $\ol{CD}$,
it is actually the internal bisector.)
\pagebreak