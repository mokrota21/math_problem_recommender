\textsl{Available online at \url{https://aops.com/community/p6642559}.}
\begin{mdframed}[style=mdpurplebox,frametitle={Problem statement}]
A set of positive integers is called \emph{fragrant}
if it contains at least two elements and each of its elements
has a prime factor in common with at least one of the other elements.
Let $P(n)=n^2+n+1$.
What is the smallest possible positive integer value of $b$ such that
there exists a non-negative integer $a$ for which the set
\[ \{P(a+1),P(a+2),\dots,P(a+b)\} \]
is fragrant?
\end{mdframed}
The answer is $b = 6$.

First, we prove $b \ge 6$ must hold.
It is not hard to prove the following divisibilities by Euclid:
\begin{align*}
 \gcd(P(n), P(n+1)) &\mid 1 \\
 \gcd(P(n), P(n+2)) &\mid 7 \\
 \gcd(P(n), P(n+3)) &\mid 3 \\
 \gcd(P(n), P(n+4)) &\mid 19.
\end{align*}
Now assume for contradiction $b \le 5$.
Then any GCD's among $P(a+1)$, \dots, $P(a+b)$ must be among $\{3, 7, 19\}$.
Consider a multi-graph on $\{a+1, \dots, a+b\}$ where we join two elements with nontrivial GCD
and label the edge with the corresponding prime.
Then we readily see there is at most one edge each of $\{3, 7, 19\}$:
id est at most one edge of gap $2$, $3$, $4$ (and no edges of gap $1$).
(By the gap of an edge $e = \{u,v\}$ we mean $|u - v|$.)
But one can see that it's now impossible for every vertex to have nonzero degree, contradiction.

To construct $b = 6$ we use the Chinese remainder theorem: select $a$ with
\begin{align*}
 a+1 & \equiv 7 \pmod{19} \\
 a+5 & \equiv 11 \pmod{19} \\
 a+2 & \equiv 2 \pmod{7} \\
 a+4 & \equiv 4 \pmod{7} \\
 a+3 & \equiv 1 \pmod{3} \\
 a+6 & \equiv 1 \pmod{3}
\end{align*}
which does the trick.
\pagebreak