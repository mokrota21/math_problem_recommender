\textsl{Available online at \url{https://aops.com/community/p5720174}.}
\begin{mdframed}[style=mdpurplebox,frametitle={Problem statement}]
Let $ABC$ be an acute triangle with orthocenter $H$,
and let $W$ be a point on the side $\ol{BC}$, between $B$ and $C$.
The points $M$ and $N$ are the feet of the altitudes
drawn from $B$ and $C$, respectively.
Suppose $\omega_1$ is the circumcircle of triangle $BWN$
and $X$ is a point such that $\ol{WX}$ is a diameter of $\omega_1$.
Similarly, $\omega_2$ is the circumcircle of triangle $CWM$
and $Y$ is a point such that $\ol{WY}$ is a diameter of $\omega_2$.
Show that the points $X$, $Y$, and $H$ are collinear.
\end{mdframed}
We present two solutions, an elementary one
and then an advanced one by moving points.

\paragraph{First solution, classical.}
Let $P$ be the second intersection of $\omega_1$ and $\omega_2$;
this is the Miquel point, so $P$ also lies on the circumcircle of $AMN$,
which is the circle with diameter $\ol{AH}$.

\begin{center}
\begin{asy}
pair A = dir(110);
pair B = dir(210);
pair C = dir(330);
pair M = foot(B, A, C);
pair N = foot(C, A, B);
draw(B--M);
draw(C--N);
filldraw(A--B--C--cycle, opacity(0.1)+lightcyan, lightblue);
draw(B--M, lightblue);
draw(C--N, lightblue);
pair W = 0.55*C+0.45*B;
pair H = A+B+C;
pair P = foot(H, A, W);
pair X = extension(P, H, B, B+A-H);
pair Y = extension(P, H, C, C+A-H);
filldraw(circumcircle(B, N, W), opacity(0.1)+lightgreen, heavygreen);
filldraw(circumcircle(C, M, W), opacity(0.1)+lightgreen, heavygreen);
draw(X--W--Y, heavygreen);
draw(B--X, lightblue);
draw(C--Y, lightblue);

draw(X--Y, red+dotted);
draw(A--W, red+dotted);
draw(circumcircle(A, M, N), red+dashed);

dot("$A$", A, dir(A));
dot("$B$", B, dir(B));
dot("$C$", C, dir(C));
dot("$M$", M, dir(M));
dot("$N$", N, dir(N));
dot("$W$", W, dir(W));
dot("$H$", H, dir(H));
dot("$P$", P, dir(P));
dot("$X$", X, dir(X));
dot("$Y$", Y, dir(Y));

/* TSQ Source:

A = dir 110
B = dir 210
C = dir 330
M = foot B A C
N = foot C A B
B--M
C--N
A--B--C--cycle 0.1 lightcyan / lightblue
B--M lightblue
C--N lightblue
W = 0.55*C+0.45*B
H = A+B+C
P = foot H A W
X = extension P H B B+A-H
Y = extension P H C C+A-H
circumcircle B N W 0.1 lightgreen / heavygreen
circumcircle C M W 0.1 lightgreen / heavygreen
X--W--Y heavygreen
B--X lightblue
C--Y lightblue

X--Y red dotted
A--W red dotted
circumcircle A M N red dashed

*/
\end{asy}
\end{center}

We now contend:
\begin{claim*}
  Points $P$, $H$, $X$ collinear.
  (Similarly, points $P$, $H$, $Y$ are collinear.)
\end{claim*}
\begin{proof}
  [Proof using power of a point]
  By radical axis on $BNMC$, $\omega_1$, $\omega_2$,
  it follows that $A$, $P$, $W$ are collinear.
  We know that $\angle APH = 90\dg$, and also
  $\angle XPW = 90\dg$ by construction.
  Thus $P$, $H$, $X$ are collinear.
\end{proof}
\begin{proof}
  [Proof using angle chasing]
  This is essentially Reim's theorem:
  \[ \dang NPH = \dang NAH = \dang BAH = \dang ABX = \dang NBX = \dang NPX \]
  as desired.
  Alternatively, one may prove $A$, $P$, $W$ are collinear by
  $\dang NPA = \dang NMA = \dang NMC = \dang NBC = \dang NBW = \dang NPW$.
\end{proof}

\paragraph{Second solution, by moving points.}
Fix $\triangle ABC$ and vary $W$.
Let $\infty$ be the point at infinity perpendicular to $\ol{BC}$
for brevity.

By spiral similarity, the point $X$ moves linearly on $\ol{B\infty}$
as $W$ varies linearly on $\ol{BC}$.
Similarly, so does $Y$.
So in other words, the map \[ X \mapsto W \mapsto Y \]
is linear.
However, the map \[ X \mapsto Y' \coloneqq \ol{XH} \cap \ol{C\infty} \]
is linear too.

To show that these maps are the same,
it suffices to check it thus at two points.
\begin{itemize}
  \ii When $W = B$, the circle $(BNW)$
  degenerates to the circle through $B$ tangent to $\ol{BC}$,
  and $X = \ol{CN} \cap \ol{B\infty}$.
  We have $Y = Y' = C$.
  \ii When $W = C$, the result is analogous.
  \ii Although we don't need to do so,
  it's also easy to check the result if $W$
  is the foot from $A$
  since then $XHWB$ and $YHWC$ are rectangles.
\end{itemize}
\pagebreak