
\textsl{Available online at \url{https://aops.com/community/p2365045}.}
\begin{mdframed}[style=mdpurplebox,frametitle={Problem statement}]
Let $ABC$ be an acute triangle with circumcircle $\Gamma$.
Let $\ell$ be a tangent line to $\Gamma$, and let $\ell_a$, $\ell_b$, $\ell_c$ be the lines obtained
by reflecting $\ell$ in the lines $BC$, $CA$, and $AB$, respectively.
Show that the circumcircle of the triangle determined by the lines $\ell_a$, $\ell_b$, and $\ell_c$
is tangent to the circle $\Gamma$.
\end{mdframed}
This is a hard problem with many beautiful solutions.
The following solution is not very beautiful but not too hard to find during an olympiad,
as the only major insight it requires is the construction of $A_2$, $B_2$, and $C_2$.

\begin{center}
  \begin{asy}
    size(11cm);
    pair A = dir(110);
    dot("$A$", A, dir(A));
    pair B = dir(195);
    dot("$B$", B, dir(160));
    pair C = dir(325);
    dot("$C$", C, 1.4*dir(30));
    pair P = dir(270);
    dot("$P$", P, dir(P));
    draw(unitcircle);
    draw(A--B--C--cycle);

    pair U = P+(2,0);
    pair V = 2*P-U;

    pair X_1 = reflect(B,C)*P;
    pair Y_1 = reflect(C,A)*P;
    pair Z_1 = reflect(A,B)*P;
    pair X_2 = extension(B, C, U, V);
    dot(X_2);
    pair Y_2 = extension(C, A, U, V);
    dot(Y_2);
    pair Z_2 = extension(A, B, U, V);
    dot(Z_2);
    draw(B--Z_2, dotted);
    draw(C--Y_2, dotted);
    draw(C--X_2, dotted);
    draw(X_2--Z_2);

    pair A_1 = extension(Y_1, Y_2, Z_1, Z_2);
    dot("$A_1$", A_1, dir(A_1));
    pair B_1 = extension(Z_1, Z_2, X_1, X_2);
    dot("$B_1$", B_1, dir(B_1));
    pair C_1 = extension(X_1, X_2, Y_1, Y_2);
    dot("$C_1$", C_1, dir(50));

    draw(A_1--B_1--C_1--cycle, black+1);
    draw(C_1--X_2, dotted);
    pair O_1 = circumcenter(A_1, B_1, C_1);
    draw(arc(O_1, abs(O_1-A_1), -80, 140));

    pair A_2 = A*A/P;
    dot("$A_2$", A_2, dir(-20));
    pair B_2 = B*B/P;
    dot("$B_2$", B_2, dir(130));
    pair C_2 = C*C/P;
    dot("$C_2$", C_2, dir(C_2));
    draw(A_2--B_2--C_2--cycle, black+1);

    pair T = extension(A_1, A_2, B_1, B_2);
    dot("$T$", T, dir(T));
    draw(T--A_1, dashed);
    draw(T--B_1, dashed);
    draw(T--C_1, dashed);

    /*
    A = dir 110
    B = dir 195
    C = dir 325
    P = dir 270

    unitcircle
    A--B--C--cycle blue
    U := P+(2,0)
    V := 2*P-U
    X = reflect(B,C)*P
    Y = reflect(C,A)*P
    Z = reflect(A,B)*P
    Y--Z heavygreen

    X1 := extension B C U V
    Y1 := extension C A U V
    Z1 := extension A B U V
    Line Y1 Z1

    A_1 = extension Y Y1 Z Z1
    B_1 = extension Z Z1 X X1
    C_1 = extension X X1 Y Y1
    A_1--B_1--C_1--cycle red
    circumcircle A_1 B_1 C_1

    circumcircle B Z X dotted
    circumcircle C X Y dotted
    circumcircle A Y Z dotted

    M = extension A_1 A*A/P B_1 B*B/P
    */
  \end{asy}
\end{center}

We apply complex numbers with $\omega$ the unit circle and $p=1$.  Let $A_1 = \ell_B \cap \ell_C$, and let $a_2 = a^2$ (in other words, $A_2$ is the reflection of $P$ across the diameter of $\omega$ through $A$).  Define the points $B_1$, $C_1$, $B_2$, $C_2$ similarly.

We claim that $\ol{A_1A_2}$, $\ol{B_1B_2}$, $\ol{C_1C_2}$ concur at a point on $\Gamma$.

We begin by finding $A_1$. If we reflect the points $1+i$ and $1-i$ over $\ol{AB}$, then we get two points $Z_1$, $Z_2$ with
\begin{align*}
  z_1 &= a+b-ab(1-i) = a+b-ab+abi \\
  z_2 &= a+b-ab(1+i) = a+b-ab-abi. \\
  \intertext{Therefore,}
  z_1 - z_2 &= 2abi  \\
  \ol{z_1}z_2 - \ol{z_2}{z_1}
    &= -2i \left( a+b+\frac1a+\frac1b-2 \right).
\end{align*}

Now $\ell_C$ is the line $\ol{Z_1Z_2}$,
so with the analogous equation $\ell_B$ we obtain:
\begin{align*}
  a_1 &= \frac{ -2i\left( a+b+\frac1a+\frac1b-2 \right)\left( 2ac i \right) +
    2i\left( a+c+\frac1a+\frac1c-2 \right)(2abi) }
    { \left( -\frac{2}{ab}i \right)
    \left( 2ac i \right) - \left( -\frac{2}{ac}i \right) \left( 2abi \right)} \\
  &= \frac{\left[ c-b \right]a^2 + \left[ \frac cb - \frac bc - 2c + 2b \right]a + (c-b)  }{\frac cb - \frac bc} \\
  &= a + \frac{(c-b)\left[ a^2-2a+1 \right]}{(c-b)(c+b)/bc} \\
  &= a + \frac{bc}{b+c} (a-1)^2.
\end{align*}
Then the second intersection of $\ol{A_1A_2}$ with $\omega$ is given by
\begin{align*}
  \frac{a_1-a_2}{1-a_2\ol{a_1}}
  &= \frac{a+\frac{bc}{b+c}(a-1)^2-a^2}{1-a-a^2 \cdot \frac{(1-1/a)^2}{b+c}} \\
  &= \frac{a + \frac{bc}{b+c}(1-a)}{1 - \frac{1}{b+c}(1-a)} \\
  &= \frac{ab+bc+ca - abc}{a+b+c-1}.
\end{align*}
Thus, the claim is proved.

Finally, it suffices to show $\ol{A_1B_1} \parallel \ol{A_2B_2}$.
One can also do this with complex numbers;
it amounts to showing $a^2-b^2$, $a-b$, $i$
(corresponding to $\ol{A_2 B_2}$, $\ol{A_1 B_1}$, $\ol{PP}$)
have their arguments an arithmetic progression, equivalently
\[ \frac{(a-b)^2}{i(a^2-b^2)} \in \RR
  \iff
  \frac{(a-b)^2}{i(a^2-b^2)}
  = \frac{\left( \frac 1a-\frac1b \right)^2}
  {\frac1i\left(\frac{1}{a^2}-\frac{1}{b^2}\right)}
\]
which is obvious.
\begin{remark*}
One can use directed angle chasing for this last part too.
Let $\ol{BC}$ meet $\ell$ at $K$ and $\ol{B_2C_2}$ meet $\ell$ at $L$.
Evidently
\begin{align*}
  -\dang B_2LP &= \dang LPB_2 + \dang PB_2L \\
  &= 2 \dang KPB + \dang PB_2C_2 \\
  &= 2 \dang KPB + 2\dang PBC \\
  &= -2\dang PKB \\
  &= \dang PKB_1 \\
\end{align*}
as required.
\end{remark*}
\pagebreak


\end{document}
