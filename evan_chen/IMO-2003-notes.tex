% © Evan Chen
% Downloaded from https://web.evanchen.cc/

\documentclass[11pt]{scrartcl}
\usepackage[sexy]{evan}
\ihead{\footnotesize\textbf{\thetitle}}
\ohead{\footnotesize\href{http://web.evanchen.cc}{\ttfamily web.evanchen.cc},
    updated \today}
\title{IMO 2003 Solution Notes}
\date{\today}

\begin{document}

\maketitle

\begin{abstract}
This is a compilation of solutions
for the 2003 IMO.
The ideas of the solution are a mix of my own work,
the solutions provided by the competition organizers,
and solutions found by the community.
However, all the writing is maintained by me.

These notes will tend to be a bit more advanced and terse than the ``official''
solutions from the organizers.
In particular, if a theorem or technique is not known to beginners
but is still considered ``standard'', then I often prefer to
use this theory anyways, rather than try to work around or conceal it.
For example, in geometry problems I typically use directed angles
without further comment, rather than awkwardly work around configuration issues.
Similarly, sentences like ``let $\mathbb{R}$ denote the set of real numbers''
are typically omitted entirely.

Corrections and comments are welcome!
\end{abstract}

\tableofcontents
\newpage

\addtocounter{section}{-1}
\section{Problems}
\begin{enumerate}[\bfseries 1.]
\item %% Problem 1
Let $A$ be a $101$-element subset of $S=\{1,2,\dots,10^6\}$.
Prove that there exist numbers $t_1$, $t_2, \dots, t_{100}$ in $S$ such that the sets
\[ A_j=\{x+t_j\mid x\in A\},\qquad j=1,2,\dots,100  \]
are pairwise disjoint.

\item %% Problem 2
Determine all pairs of positive integers $(a,b)$ such that
\[ \frac{a^2}{2ab^2-b^3+1} \]
is a positive integer.

\item %% Problem 3
Each pair of opposite sides of convex hexagon has the property that
the distance between their midpoints is $\frac{\sqrt3}{2}$
times the sum of their lengths.
Prove that the hexagon is equiangular.

\item %% Problem 4
Let $ABCD$ be a cyclic quadrilateral.
Let $P, Q$ and $R$ be the feet of perpendiculars
from $D$ to lines $\ol{BC}$, $\ol{CA}$ and $\ol{AB}$, respectively.
Show that $PQ = QR$ if and only if the
bisectors of angles $ABC$ and $ADC$ meet on segment $\ol{AC}$.

\item %% Problem 5
Let $n$ be a positive integer and
let $x_1 \le x_2 \le \dots \le x_n$ be real numbers.
Prove that
\[ \left(\sum_{i=1}^{n}\sum_{j=1}^{n} |x_i - x_j|\right)^2
  \le \frac{2(n^2-1)}{3}\sum_{i=1}^{n}\sum_{j=1}^{n} (x_i - x_j)^2 \]
with equality if and only if $x_1$, $x_2$, \dots, $x_n$
form an arithmetic sequence.

\item %% Problem 6
Let $p$ be a prime number.
Prove that there exists a prime number $q$ such that for every integer $n$,
the number $n^p-p$ is not divisible by $q$.

\end{enumerate}
\pagebreak

\section{Solutions to Day 1}
\subsection{IMO 2003/1, proposed by Carlos Gustavo Moreira (BRA)}
\textsl{Available online at \url{https://aops.com/community/p261}.}
\begin{mdframed}[style=mdpurplebox,frametitle={Problem statement}]
Let $A$ be a $101$-element subset of $S=\{1,2,\dots,10^6\}$.
Prove that there exist numbers $t_1$, $t_2, \dots, t_{100}$ in $S$ such that the sets
\[ A_j=\{x+t_j\mid x\in A\},\qquad j=1,2,\dots,100  \]
are pairwise disjoint.
\end{mdframed}
A greedy algorithm works: suppose we have picked
\[ T = \left\{ t_1, \dots, t_n \right\} \]
as large as possible, meaning it's impossible to add any more elements to $T$.
That means, for each $t \in \left\{ 1, \dots, 10^6 \right\}$ either $t \in T$ already
or there exists two distinct elements $a, b \in A$ and $t_i \in T$ such that
\[ t = t_i + b - a \qquad (\star).  \]

There are at most
$|T| \cdot |A| \cdot \left( |A|-1 \right) = n \cdot 101 \cdot 100$
possible values for the right-hand side of $(\star)$.
So we therefore must have
\[ 101 \cdot 100 \cdot n + n \ge 10^6 \]
which implies $n > 99$, as desired.

\begin{remark*}
  It is possible to improve the bound significantly with a small optimization;
  rather than adding any $t$, we require that $t_1 < \dots < t_n$
  and that at each step we add the \emph{least} $t \in S$ which is permitted.
  In that case, one finds we only need to consider $b > a$ in $(\star)$,
  and so this will save us a factor of $2+o(1)$
  as the main term $101 \cdot 100$ becomes $\binom{101}{2}$ instead.
  This proves it's possible to choose $198$ elements.

  See, e.g., \url{https://aops.com/community/p22959828} for such a write-up.
\end{remark*}
\pagebreak

\subsection{IMO 2003/2, proposed by Aleksander Ivanov (BGR)}
\textsl{Available online at \url{https://aops.com/community/p262}.}
\begin{mdframed}[style=mdpurplebox,frametitle={Problem statement}]
Determine all pairs of positive integers $(a,b)$ such that
\[ \frac{a^2}{2ab^2-b^3+1} \]
is a positive integer.
\end{mdframed}
The answer is $(a,b) = (2\ell, 1)$, $(a,b) = (\ell, 2\ell)$
and $(a,b) = (8\ell^4-\ell, 2\ell)$, for any $\ell$.
Check these work.

In the sequel, assume $b > 1$,
and integers $a$, $b$, $k$ obey $k = \frac{a^2}{2ab^2-b^3+1}$.
Expanding, we have the polynomial
\[ X^2 - 2kb^2 \cdot X + k(b^3-1) = 0 \]
has two integer roots, one of which is $X = a$.
This means solutions to the original problem come in pairs
(even with $k$ fixed):
\[ (a,b) \longleftrightarrow
  \left( 2kb^2 - a, b\right)
  = \left( \frac{k(b^3-1)}{a}, b\right). \]
(Here, the first representation ensures
$2kb^2-a \in \ZZ$,
while the latter representation and the hypothesis $b > 1$ ensures
that $\frac{k(b^3-1)}{a} > 0$.)

On the other hand, we claim that:
\begin{claim*}
For any solution $(a,b)$,
either $2a = b$ or $a > b$.
\end{claim*}
\begin{proof}
  Since the denominator is positive, $a \ge b/2$.
  Now,
  \[ a^2 \ge 2ab^2 - b^3 + 1 \iff a^2 \ge b^2(2a-b) + 1 \]
  and so if $2a - b > 0$ then $a^2 > b^2 \implies a > b$.
\end{proof}

Now assume we have pair $(a_1, b)$ and $(a_2, b)$
of solutions with $b \neq 2a_1, 2a_2$.
Then assume $a_1 > a_2 > b$ and
\begin{align*}
  a_1 + a_2 &= 2k \cdot b^2 \\
  a_1a_2 &= k(b^3-1)
\end{align*}
That's impossible, since then $a_1 > \frac{a_1+a_2}{2} = k b^2$
and hence $a_1a_2 > kb^2 \cdot b = kb^3$.
Thus the only solutions are the ones we claimed at the beginning.

\begin{remark*}
  Important to notice that the problem is \emph{positive divides},
  not just divides.
  There is an implicit inequality built in to the problem
  statement and it is essentially impossible to solve without.
  I would be interested in a pair $(a,b)$
  for which $k < 0$, $k \in \ZZ$ yet $a, b > 0$.
\end{remark*}
\pagebreak

\subsection{IMO 2003/3, proposed by Waldemar Pompe (POL)}
\textsl{Available online at \url{https://aops.com/community/p263}.}
\begin{mdframed}[style=mdpurplebox,frametitle={Problem statement}]
Each pair of opposite sides of convex hexagon has the property that
the distance between their midpoints is $\frac{\sqrt3}{2}$
times the sum of their lengths.
Prove that the hexagon is equiangular.
\end{mdframed}
Unsurprisingly, this is a geometric inequality.
Denote the hexagon by $ABCDEF$.
Then we have that
\[
  \left\lvert
  \frac{\vec D + \vec E}{2} - \frac{\vec A + \vec B}{2}
  \right\rvert
  = \sqrt3 \cdot \frac{\left\lvert \vec B - \vec A \right\rvert
    + \left\lvert \vec E - \vec D \right\rvert}{2}
  \ge \sqrt 3 \cdot
  \left\lvert \frac{(\vec B - \vec A) - (\vec E - \vec D)}{2}  \right\rvert
\]
and cyclic variations.
Suppose we define the right-hand sides as variables
\begin{align*}
  \vec x = (\vec B - \vec A) - (\vec E - \vec D) \\
  \vec y = (\vec D - \vec C) - (\vec A - \vec F) \\
  \vec z = (\vec F - \vec E) - (\vec C - \vec B).
\end{align*}
Then we now have
\begin{align*}
  \left\lvert \vec y - \vec z \right\rvert
  &\ge \sqrt 3 \left\lvert \vec x  \right\rvert \\
  \left\lvert \vec z - \vec x \right\rvert
  &\ge \sqrt 3 \left\lvert \vec y  \right\rvert \\
  \left\lvert \vec x - \vec y \right\rvert
  &\ge \sqrt 3 \left\lvert \vec z  \right\rvert.
\end{align*}
We square all sides (using
$\left\lvert \vec v \right\rvert^2 = \vec v \cdot \vec v$)
and then sum to get
\[ \sum_{\text{cyc}} (\vec y - \vec z) \cdot (\vec y - \vec z)
  \ge 3 \sum_{\text{cyc}} \vec x \cdot \vec x \]
which rearranges to
\[- \left\lvert \vec x + \vec y + \vec z \right\rvert^2 \ge 0. \]
This can only happen if $\vec x + \vec y + \vec z =0$,
and moreover all the inequalities above were actually equalities.
That means that our triangle inequalities above were actually sharp
(and already we have $\ol{AB} \parallel \ol{DE}$ and so on).

Working with just $x$ and $y$ now we have
\begin{align*}
  3 (\vec x \cdot \vec x) &= (2 \vec y - \vec x) \cdot (2 \vec y - \vec x) \\
  &= \vec x \cdot \vec x - 4 \vec y \cdot \vec x + 4 \vec y \cdot \vec y \\
  \implies
  -\vec x \cdot \vec x + 2 (\vec y \cdot \vec y) &= 2 \vec x \cdot \vec y \\
  2 (\vec x \cdot \vec x) - \vec y \cdot \vec y &= 2 \vec x \cdot \vec y.
\end{align*}
which implies $\vec x \cdot \vec x = \vec y \cdot \vec y$,
that is, $\vec x$ and $\vec y$ have the same magnitude.
In this way we find $\vec x$, $\vec y$, $\vec z$ all
have the same magnitude,
and since $\vec x + \vec y + \vec z = 0$
they are related by $120\dg$ rotations, as desired.

\begin{remark*}
  In fact one can show further that the equiangular hexagons
  which work are exactly those formed by taking an equilateral triangle
  and cutting off equally sized corners.
  This equality case helps motivate the solution.
\end{remark*}
\begin{remark*}
  One can note this ``must'' be an inequality
  because the space of such hexagons is $2$-dimensional,
  even though \emph{a priori} the space of hexagons satisfying
  three given conditions should have dimension $9-3=6$.
\end{remark*}
\pagebreak

\section{Solutions to Day 2}
\subsection{IMO 2003/4, proposed by Matti Lehtinen (FIN)}
\textsl{Available online at \url{https://aops.com/community/p264}.}
\begin{mdframed}[style=mdpurplebox,frametitle={Problem statement}]
Let $ABCD$ be a cyclic quadrilateral.
Let $P, Q$ and $R$ be the feet of perpendiculars
from $D$ to lines $\ol{BC}$, $\ol{CA}$ and $\ol{AB}$, respectively.
Show that $PQ = QR$ if and only if the
bisectors of angles $ABC$ and $ADC$ meet on segment $\ol{AC}$.
\end{mdframed}
Let $\gamma$ denote the circumcircle of $ABCD$.
The condition on bisectors is equivalent to $(AC;BD)_\gamma = -1$.
Meanwhile if $\infty$ denotes the point at infinity along Simson line $\ol{PQR}$
then $PQ = QR$ if and only if $(PR;Q\infty) = -1$.

Let rays $BQ$ and $DQ$ meet the circumcircle again at $F$ and $E$.

\begin{center}
\begin{asy}
pair B = dir(150);
pair A = dir(200);
pair C = dir(340);
pair D = IP(2*A*C/(A+C)--B, unitcircle);

draw(A--D--C--B--A--C, lightblue);
filldraw(unitcircle, opacity(0.1)+lightcyan, blue);
pair P = foot(D, B, C);
pair Q = foot(D, A, C);
pair R = foot(D, A, B);
draw(P--D--R, blue);
draw(R--A, lightblue);

draw(P--R, red);
pair O = origin;
pair E = -D+2*foot(O, D, Q);
pair F = -B+2*foot(O, B, Q);

draw(B--E, red);
draw(B--F, lightblue);
draw(D--E, lightblue);

dot("$B$", B, dir(B));
dot("$A$", A, dir(A));
dot("$C$", C, dir(C));
dot("$D$", D, dir(D));
dot("$P$", P, dir(60));
dot("$Q$", Q, dir(315));
dot("$R$", R, dir(R));
dot("$E$", E, dir(E));
dot("$F$", F, dir(F));

/* TSQ Source:

B = dir 150
A = dir 200
C = dir 340
D = IP 2*A*C/(A+C)--B unitcircle

A--D--C--B--A--C lightblue
unitcircle 0.1 lightcyan / blue
P = foot D B C R60
Q = foot D A C R315
R = foot D A B
P--D--R blue
R--A lightblue

P--R red
O := origin
E = -D+2*foot O D Q
F = -B+2*foot O B Q

B--E red
B--F lightblue
D--E lightblue

*/
\end{asy}
\end{center}

\begin{lemma*}
  [EGMO Proposition 4.1]
  Then $\ol{BE} \parallel \ol{PQR}$.
\end{lemma*}
\begin{proof}
  Since $\dang DPR = \dang DAR = \dang DAB = \dang DEB$.
\end{proof}

Now we have
\[ (PR;Q\infty) \overset{B}{=} (CA;FE)_\gamma
  \overset{Q}{=} (AC;BD)_\gamma \]
as desired.
\pagebreak

\subsection{IMO 2003/5, proposed by Finbarr Holland (IRL)}
\textsl{Available online at \url{https://aops.com/community/p265}.}
\begin{mdframed}[style=mdpurplebox,frametitle={Problem statement}]
Let $n$ be a positive integer and
let $x_1 \le x_2 \le \dots \le x_n$ be real numbers.
Prove that
\[ \left(\sum_{i=1}^{n}\sum_{j=1}^{n} |x_i - x_j|\right)^2
  \le \frac{2(n^2-1)}{3}\sum_{i=1}^{n}\sum_{j=1}^{n} (x_i - x_j)^2 \]
with equality if and only if $x_1$, $x_2$, \dots, $x_n$
form an arithmetic sequence.
\end{mdframed}
Let $d_1 = x_2 - x_1$, \dots, $d_{n-1} = x_n - x_{n-1}$.
The inequality in question becomes:
\[
  \left( \sum_i i(n-i) d_i \right)^2
  \le
  \frac{n^2-1}{3} \cdot
  \left( \sum_i i(n-i) d_i^2 + 2\sum_{i<j} i(n-j) d_i d_j  \right).
\]
Clearing the square on the right-hand side we want to show
\[
  \sum_{i<j} \left( 3ij(n-i)(n-j) - (n^2-1)i(n-j) \right) \cdot 2d_i d_j
  \le \sum_i (n^2-1-3i(n-i)) \cdot i(n-i) d_i^2.
\]
We use AM-GM directly on $2d_i d_j \le d_i^2 + d_j^2$:
this actually solves the problem.
The annoying part is to check that the coefficients actually match:
\begin{claim*}
  [Big bash]
  For an index $1 \le k \le n-1$, we have
  \begin{align*}
    &\sum_{i<k} \left( 3ik(n-i)(n-k) - (n^2-1)i(n-k) \right) \\
    +& \sum_{j>k} \left( 3kj(n-k)(n-j) - (n^2-1)k(n-j) \right) \\
    = & (n^2-1-3k(n-k)) \cdot k(n-k).
  \end{align*}
\end{claim*}
\begin{proof}
  Rewrite as:
  \begin{align*}
    3k(n-k) \left( -k(n-k) + \sum_i i(n-i) \right)
      &= (n^2-1)\left( (n-k)\sum_{i<k} i + k\sum_{j>k} (n-j) \right) \\
      &+ (n^2-1-3k(n-k)) \cdot k(n-k) \\
    \iff 3k(n-k) \sum_i i(n-i)
      &= (n^2-1) \left( (n-k)\sum_{i<k} i + k\sum_{j>k}(n-j) \right) \\
      &+ (n^2-1)k(n-k) - 3k^2(n-k)^2 \\
    \iff 3k(n-k) \left( \sum_{i} i(n-i) \right)
      &= (n^2-1) \left( (n-k)\sum_{i \le k} i + k \sum_{i < n-k} i \right) \\
    \iff 3k(n-k) \frac{(n-1)n(n+1)}{6}
      &= (n^2-1) \left( (n-k)\frac{k(k+1)}{2} \right) \\
      & + (n^2-1) \left( k \frac{(n-k)(n-k-1)}{2} \right) \\
    \iff 3k(n-k) \frac{(n-1)n(n+1)}{6}
      &= (n^2-1)k(n-k) \cdot \frac n2
  \end{align*}
  which is visibly true.
\end{proof}
Equality occurs only if all $d_i$ are equal
because the coefficient of $d_i d_j$ is nonzero
for any $i \le n/2$ and $j \ge n/2$.
\pagebreak

\subsection{IMO 2003/6, proposed by Johan Yebbou (FRA)}
\textsl{Available online at \url{https://aops.com/community/p266}.}
\begin{mdframed}[style=mdpurplebox,frametitle={Problem statement}]
Let $p$ be a prime number.
Prove that there exists a prime number $q$ such that for every integer $n$,
the number $n^p-p$ is not divisible by $q$.
\end{mdframed}
By orders, we must have $q=pk+1$ for this to be possible
(since if $q \not \equiv 1 \pmod p$, then $n^p$ can be any residue modulo $q$).
Since $p \equiv n^p \pmod q \implies p^k \equiv 1 \pmod q$,
it suffices to prevent the latter situation from happening.

So we need a prime $q \equiv 1 \pmod p$ such that $p^k \not\equiv 1 \pmod q$.
To do this, we first recall the following lemma.
\begin{lemma*}
  Let $\Phi_p(X) = 1 + X + X^2 + \dots + X^{p-1}$.
  For any integer $a$, if $q$ is a prime divisor of $\Phi_p(a)$ other than $p$,
  then $a \pmod q$ has order $p$. (In particular, $q \equiv 1 \pmod p$.)
\end{lemma*}
\begin{proof}
  We have $a^p-1 \equiv 0 \pmod q$, so either the order is $1$ or $p$.
  If it is $1$, then $a \equiv 1 \pmod q$, so $q \mid \Phi_p(1) = p$, hence $q = p$.
\end{proof}

% Wishfully we hope the order of $p$ is $p$ and $k \nmid p$.
Now the idea is to extract a prime factor $q$
from the cyclotomic polynomial
\[ \Phi_p(p) = \frac{p^p-1}{p-1} \equiv 1+p \pmod{p^2} \]
such that $q \not\equiv 1 \pmod{p^2}$;
hence $k \not\equiv 0 \pmod p$,
and as $p \pmod q$ has order $p$ we have $p^k \not\equiv 1 \pmod q$.
\pagebreak


\end{document}
