\textsl{Available online at \url{https://aops.com/community/p1190553}.}
\begin{mdframed}[style=mdpurplebox,frametitle={Problem statement}]
Let $H$ be the orthocenter of an acute-angled triangle $ABC$.
The circle $\Gamma_{A}$ centered at the midpoint of $\ol{BC}$ and passing
through $H$ intersects the sideline $BC$ at points  $A_1$ and $A_2$.
Similarly, define the points $B_1$, $B_2$, $C_1$, and $C_2$.
Prove that six points $A_1$, $A_2$, $B_1$, $B_2$, $C_1$, $C_2$ are concyclic.
\end{mdframed}
We show two solutions.

\paragraph{First solution using power of a point.}
Let $D$, $E$, $F$ be the centers of $\Gamma_A$, $\Gamma_B$, $\Gamma_C$
(in other words, the midpoints of the sides).

We first show that $B_1$, $B_2$, $C_1$, $C_2$ are concyclic.
It suffices to prove that $A$
lies on the radical axis of the circles $\Gamma_B$ and $\Gamma_C$.

\begin{center}
\begin{asy}
pair A = dir(110);
pair B = dir(210);
pair C = dir(330);
draw(A--B--C--cycle);

pair D = midpoint(B--C);
pair E = midpoint(C--A);
pair F = midpoint(A--B);
pair H = orthocenter(A, B, C);
draw(E--F);

pair X = -H+2*foot(H, E, F);
path w1 = Drawing(CP(E,H));
path w2 = Drawing(CP(F,H));
pair B_1 = IP(w1, A--C);
pair B_2 = OP(w1, A--C);
pair C_1 = IP(w2, A--B);
pair C_2 = OP(w2, A--B);

pair T = foot(A, B, C);
draw(A--T, dotted);

dot("$A$", A, dir(A));
dot("$B$", B, dir(B));
dot("$C$", C, dir(C));
dot("$D$", D, dir(D));
dot("$E$", E, dir(E));
dot("$F$", F, dir(F));
dot("$H$", H, 1.4*dir(-10));
dot("$X$", X, 1.4*dir(10));
dot("$B_1$", B_1, 1.5*dir(90));
dot("$B_2$", B_2, 1.2*dir(-10));
dot("$C_1$", C_1, 1.2*dir(120));
dot("$C_2$", C_2, dir(C_2));

/* TSQ Source:

A = dir 110
B = dir 210
C = dir 330
A--B--C--cycle

D = midpoint B--C
E = midpoint C--A
F = midpoint A--B
H = orthocenter A B C 1.4R-10
E--F

X = -H+2*foot H E F 1.4R10
! path w1 = Drawing(CP(E,H));
! path w2 = Drawing(CP(F,H));
B_1 = IP w1 A--C 1.5R90
B_2 = OP w1 A--C 1.2R-10
C_1 = IP w2 A--B 1.2R120
C_2 = OP w2 A--B

T := foot A B C
A--T dotted

*/
\end{asy}
\end{center}

Let $X$ be the second intersection of $\Gamma_B$ and $\Gamma_C$.
Clearly $\ol{XH}$ is perpendicular to the line
joining the centers of the circles, namely $\ol{EF}$.
But $\ol{EF} \parallel \ol{BC}$, so $\ol{XH} \perp \ol{BC}$.
Since $\ol{AH} \perp \ol{BC}$ as well,
we find that $A$, $X$, $H$ are collinear, as needed.

Thus, $B_1$, $B_2$, $C_1$, $C_2$ are concyclic.
Similarly, $C_1$, $C_2$, $A_1$, $A_2$ are concyclic,
as are $A_1$, $A_2$, $B_1$, $B_2$.
Now if any two of these three circles coincide, we are done;
else the pairwise radical axii are not concurrent, contradiction.
(Alternatively, one can argue directly that $O$ is the center of all
three circles, by taking the perpendicular bisectors.)

\paragraph{Second solution using length chase (Ritwin Narra).}
We claim the circumcenter $O$ of $\triangle ABC$
is in fact the center of $(A_1A_2B_1B_2C_1C_2)$.

Define $D$, $E$, $F$ as before.
Then since $\ol{OD} \perp \ol{A_1A_2}$ and $DA_1 = DA_2$,
which means $OA_1 = OA_2$. Similarly, we have $OB_1 = OB_2$ and $OC_1 = OC_2$.

Now since $DA_1 = DA_2 = DH$, we have $OA_1^2 = OD^2 + HD^2$.
We seek to show
\[ OD^2 + HD^2 = OE^2 + HE^2 = OF^2 + HF^2. \]
This is clear by Appollonius's Theorem
since $D$, $E$, and $F$ lie on the nine-point circle,
which is centered at the midpoint of $\ol{OH}$.
\pagebreak