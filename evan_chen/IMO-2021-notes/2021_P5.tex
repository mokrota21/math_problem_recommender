\textsl{Available online at \url{https://aops.com/community/p22697921}.}
\begin{mdframed}[style=mdpurplebox,frametitle={Problem statement}]
Two squirrels, Bushy and Jumpy, have collected $2021$ walnuts for the winter.
Jumpy numbers the walnuts from $1$ through $2021$, and digs $2021$ little holes
in a circular pattern in the ground around their favourite tree.
The next morning Jumpy notices that Bushy had placed one walnut into each hole,
but had paid no attention to the numbering.
Unhappy, Jumpy decides to reorder the walnuts by performing a sequence of 2021 moves.
In the $k$th move, Jumpy swaps the positions of the two walnuts adjacent to walnut $k$.

Prove that there exists a value of $k$ such that, on the $k$th move,
Jumpy swaps some walnuts $a$ and $b$ such that $a<k<b$.
\end{mdframed}
Assume for contradiction no such $k$ exists.
We will use a so-called ``threshold trick''.

This process takes exactly $2021$ steps.
Right after the $k$th move, we consider a situation where
we color walnut $k$ red as well, so at the $k$th step there are $k$ ones.
For brevity, a non-red walnut is called black.
An example is illustrated below with $2021$ replaced by $6$.
\begin{center}
\begin{asy}
  size(14cm);
  dotfactor *= 1.5;
  picture base;
  picture pA, pB, pC, pD, pE, pF, pG;
  pen old = black;
  pen used = mediumred;
  pen active = red;

  draw(pA, unitcircle);
  label(pA, "Initial", origin);
  dot(pA, "$1$", dir(  0), dir(  0), old);
  dot(pA, "$4$", dir( 60), dir( 60), old);
  dot(pA, "$2$", dir(120), dir(120), old);
  dot(pA, "$5$", dir(180), dir(180), old);
  dot(pA, "$3$", dir(240), dir(240), old);
  dot(pA, "$6$", dir(300), dir(300), old);

  draw(pB, unitcircle);
  dot(pB, "$\boxed{1}$", dir(  0), dir(  0), active);
  dot(pB, "$6$", dir( 60), dir( 60), old);
  dot(pB, "$2$", dir(120), dir(120), old);
  dot(pB, "$5$", dir(180), dir(180), old);
  dot(pB, "$3$", dir(240), dir(240), old);
  dot(pB, "$4$", dir(300), dir(300), old);
  draw(pB, dir(60)--dir(300), blue+dotted, Arrows(TeXHead), Margins);

  draw(pC, unitcircle);
  dot(pC, "$1$", dir(  0), dir(  0), used);
  dot(pC, "$5$", dir( 60), dir( 60), old);
  dot(pC, "$\boxed{2}$", dir(120), dir(120), active);
  dot(pC, "$6$", dir(180), dir(180), old);
  dot(pC, "$3$", dir(240), dir(240), old);
  dot(pC, "$4$", dir(300), dir(300), old);
  draw(pC, dir(60)--dir(180), blue+dotted, Arrows(TeXHead), Margins);

  draw(pD, unitcircle);
  dot(pD, "$1$", dir(  0), dir(  0), used);
  dot(pD, "$5$", dir( 60), dir( 60), old);
  dot(pD, "$2$", dir(120), dir(120), used);
  dot(pD, "$4$", dir(180), dir(180), old);
  dot(pD, "$\boxed{3}$", dir(240), dir(240), active);
  dot(pD, "$6$", dir(300), dir(300), old);
  draw(pD, dir(180)--dir(300), blue+dotted, Arrows(TeXHead), Margins);

  draw(pE, unitcircle);
  dot(pE, "$1$", dir(  0), dir(  0), used);
  dot(pE, "$5$", dir( 60), dir( 60), old);
  dot(pE, "$3$", dir(120), dir(120), used);
  dot(pE, "$\boxed{4}$", dir(180), dir(180), active);
  dot(pE, "$2$", dir(240), dir(240), used);
  dot(pE, "$6$", dir(300), dir(300), old);
  draw(pE, dir(120)--dir(240), blue+dotted, Arrows(TeXHead), Margins);

  draw(pF, unitcircle);
  dot(pF, "$1$", dir(  0), dir(  0), used);
  dot(pF, "$\boxed{5}$", dir( 60), dir( 60), active);
  dot(pF, "$3$", dir(120), dir(120), used);
  dot(pF, "$4$", dir(180), dir(180), used);
  dot(pF, "$2$", dir(240), dir(240), used);
  dot(pF, "$6$", dir(300), dir(300), old);
  draw(pF, dir(120)--dir(0), blue+dotted, Arrows(TeXHead), Margins);

  draw(pG, unitcircle);
  dot(pG, "$1$", dir(  0), dir(  0), used);
  dot(pG, "$5$", dir( 60), dir( 60), used);
  dot(pG, "$3$", dir(120), dir(120), used);
  dot(pG, "$4$", dir(180), dir(180), used);
  dot(pG, "$2$", dir(240), dir(240), used);
  dot(pG, "$\boxed{6}$", dir(300), dir(300), active);
  draw(pG, dir(60)--dir(240), blue+dotted, Arrows(TeXHead), Margins);

  add(shift(-3,3)*pA);
  add(shift(0,3)*pB);
  add(shift(3,3)*pC);
  add(shift(6,3)*pD);
  add(shift(0,0)*pE);
  add(shift(3,0)*pF);
  add(shift(6,0)*pG);
\end{asy}
\end{center}

\begin{claim*}
  At each step, the walnut that becomes red is between two non-red or two red walnuts.
\end{claim*}
\begin{proof}
  By definition.
\end{proof}

On the other hand, if there are $2021$ walnuts,
one obtains a parity obstruction to this simplified process:
\begin{claim*}
  After the first step, there is always a consecutive block of black walnuts positive even length.
\end{claim*}
\begin{proof}
  After the first step, there is a block of $2020$ black walnuts.

  Thereafter, note that a length $2$ block of black walnuts can never be changed.
  Meanwhile for even lengths at least $4$, if one places a red walnut inside it,
  the even length block splits into an odd length block and an even length block.
\end{proof}

\begin{remark*}
  The statement is true with $2021$ replaced by any odd number, and false for any even number.

  The motivation comes from the following rephrasing of the problem:
  \begin{quote}
    Start with all $0$'s and at each step change a $0$ between
    two matching numbers from a $0$ to a $1$.
  \end{quote}
  Although the coloring (or $0$/$1$) argument may appear to lose information
  at first, I think it should be \emph{equivalent} to the original process;
  the ``extra'' information comes down to the choice of
  which walnut to color red at each step.
\end{remark*}
\pagebreak