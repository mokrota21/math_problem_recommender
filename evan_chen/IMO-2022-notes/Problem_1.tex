The Bank of Oslo issues two types of coin: aluminum (denoted $A$) and bronze
(denoted $B$). Marianne has $n$ aluminum coins and $n$ bronze coins arranged in a
row in some arbitrary initial order.
A chain is any subsequence of consecutive coins of the same type.
Given a fixed positive integer $k \leq 2n$,
Gilberty repeatedly performs the following operation:
he identifies the longest chain containing the $k$\ts{th} coin from the left
and moves all coins in that chain to the left end of the row.
For example, if $n=4$ and $k=4$, the process starting
from the ordering $AABBBABA$ would be
$AABBBABA \to BBBAAABA \to AAABBBBA \to BBBBAAAA \to \dotsb$.

Find all pairs $(n,k)$ with $1 \leq k \leq 2n$
such that for every initial ordering,
at some moment during the process,
the leftmost $n$ coins will all be of the same type.