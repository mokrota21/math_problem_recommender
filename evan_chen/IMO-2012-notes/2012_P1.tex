\textsl{Available online at \url{https://aops.com/community/p2736397}.}
\begin{mdframed}[style=mdpurplebox,frametitle={Problem statement}]
Let $ABC$ be a triangle and $J$ the center of the $A$-excircle.
This excircle is tangent to the side $BC$ at $M$,
and to the lines $AB$ and $AC$ at $K$ and $L$, respectively.
The lines $LM$ and $BJ$ meet at $F$, and the lines $KM$ and $CJ$ meet at $G$.
Let $S$ be the point of intersection of the lines $AF$ and $BC$,
and let $T$ be the point of intersection of the lines $AG$ and $BC$.
Prove that $M$ is the midpoint of $\ol{ST}$.
\end{mdframed}
We employ barycentric coordinates with reference $\triangle ABC$.
As usual $a = BC$, $b = CA$, $c = AB$, $s = \half(a+b+c)$.

It's obvious that $K = ( -(s-c): s : 0)$, $M = ( 0 : s-b : s-c)$.
Also, $J = (-a : b : c)$.
We then obtain
\[ G = \left( -a: b : \frac{-as + (s-c)b}{s-b} \right). \]
It follows that
\[ T = \left( 0 : b : \frac{-as + (s-c)}{s-b} \right) = ( 0 : b(s-b) : b(s-c) - as). \]
Normalizing, we see that $T = \left( 0, -\frac{b}{a}, 1 + \frac{b}{a} \right)$,
from which we quickly obtain $MT = s$.
Similarly, $MS = s$, so we're done.
\pagebreak