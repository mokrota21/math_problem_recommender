The liar's guessing game is a game played between two players $A$ and $B$.
The rules of the game depend on two fixed positive integers $k$ and $n$
which are known to both players.

At the start of the game $A$
chooses integers $x$ and $N$ with $1 \le x \le N$.
Player $A$ keeps $x$ secret, and truthfully tells $N$ to player $B$.
Player $B$ now tries to obtain information about $x$
by asking player $A$ questions as follows:
each question consists of $B$ specifying an arbitrary set $S$
of positive integers (possibly one specified in some previous question),
and asking $A$ whether $x$ belongs to $S$.
Player $B$ may ask as many questions as he wishes.
After each question, player $A$ must immediately answer
it with yes or no, but is allowed to lie as many times as she wants;
the only restriction is that, among any $k+1$ consecutive answers,
at least one answer must be truthful.

After $B$ has asked as many questions as he wants,
he must specify a set $X$ of at most $n$ positive integers.
If $x$ belongs to $X$, then $B$ wins;
otherwise, he loses.
Prove that:

\begin{enumerate}[(a)]
  \ii If $n \ge 2^k$, then $B$ can guarantee a win.
  \ii For all sufficiently large $k$,
  there exists an integer $n \ge (1.99)^k$
  such that $B$ cannot guarantee a win.
\end{enumerate}