\textsl{Available online at \url{https://aops.com/community/p31218657}.}
\begin{mdframed}[style=mdpurplebox,frametitle={Problem statement}]
Let triangle $ABC$ with incenter $I$ satisfying $AB < AC < BC$.
Let $X$ be a point on line $BC$, different from $C$,
such that the line through $X$ and parallel to $AC$ is tangent to the incircle.
Similarly, let $Y$ be a point on line $BC$, different from $B$,
such that the line through $Y$ and parallel to $AB$ is tangent to the incircle.
Line $AI$ intersects the circumcircle of triangle $ABC$ again at $P$.
Let $K$ and $L$ be the midpoints of $AC$ and $AB$, respectively.
Prove that $\angle KIL + \angle YPX = 180^{\circ}$.
\end{mdframed}
Let $T$ be the reflection of $A$ over $I$, the most important point to add
since it gets rid of $K$ and $L$ as follows.
\begin{claim*}
  We have $\angle KIL = \angle BTC$,
  and lines $TX$ and $TY$ are tangent to the incircle.
\end{claim*}
\begin{proof}
  The first part is true since $\triangle BTC$ is the image of $\triangle KIL$
  under a homothety of ratio $2$.
  The second part is true because lines $AB$, $AC$, $TX$, $TY$
  determine a rhombus with center $I$.
\end{proof}

We thus delete $K$ and $L$ from the picture altogether; they aren't needed anymore.

\begin{center}
\begin{asy}
size(11cm);
pair A = dir(105);
pair B = dir(200);
pair C = dir(340);
pair P = dir(270);
filldraw(unitcircle, opacity(0.1)+lightcyan, blue);
pair I = incenter(A, B, C);
filldraw(incircle(A, B, C), opacity(0.1)+lightcyan, blue);
draw(A--B--C--cycle, blue);
pair E = foot(I, C, A);
pair F = foot(I, A, B);
pair T = 2*I-A;
pair U = 2*I-E;
pair V = 2*I-F;
draw(U--T--V, deepgreen);
draw(B--T--C, red);
pair K = midpoint(A--B);
pair L = midpoint(A--C);
draw(K--I--L, red+dashed);
pair X = extension(U, T, B, C);
pair Y = extension(V, T, B, C);
draw(A--P, grey);
draw(circumcircle(B, X, P), grey+dashed);
draw(circumcircle(C, Y, P), grey+dashed);

dot("$A$", A, dir(A));
dot("$B$", B, dir(160));
dot("$C$", C, dir(20));
dot("$P$", P, dir(45));
dot("$I$", I, dir(250));
dot("$T$", T, dir(225));
dot("$K$", K, dir(K));
dot("$L$", L, dir(L));
dot("$X$", X, dir(X));
dot("$Y$", Y, dir(310));

/* -----------------------------------------------------------------+
|                 TSQX: by CJ Quines and Evan Chen                  |
| https://github.com/vEnhance/dotfiles/blob/main/py-scripts/tsqx.py |
+-------------------------------------------------------------------+
!size(11cm);
A = dir 105
B 160 = dir 200
C 20 = dir 340
P 45 = dir 270
unitcircle / 0.1 lightcyan / blue
I 250 = incenter A B C
incircle A B C / 0.1 lightcyan / blue
A--B--C--cycle / blue
E := foot I C A
F := foot I A B
T 225 = 2*I-A
U := 2*I-E
V := 2*I-F
U--T--V / deepgreen
B--T--C / red
K = midpoint A--B
L = midpoint A--C
K--I--L / red dashed
X = extension U T B C
Y 310 = extension V T B C
A--P / grey
circumcircle B X P / grey dashed
circumcircle C Y P / grey dashed
*/
\end{asy}
\end{center}

\begin{claim*}
  We have $BXPT$ and $CYPT$ are cyclic.
\end{claim*}
\begin{proof}
  $\dang TYC = \dang TYB = \dang ABC = \dang APC = \dang TPC$ and similarly.
  (Some people call this Reim's theorem.)
\end{proof}


To finish, observe that
\[ \dang CTB = \dang CTP + \dang PTB = \dang CYP + \dang PXB
  = \dang XYP + \dang XYP = \dang XPY \]
as desired.
(The length conditions $AC > AB > BC$ ensure that $B$, $X$, $Y$, $C$ are collinear
in that order, and that $T$ lies on the opposite side of $\ol{BC}$ as $A$.
Hence the directed equality $\dang CTB = \dang XPY$
translates to the undirected $\angle BTC + \angle XPY = 180\dg$.)
\pagebreak