\textsl{Available online at \url{https://aops.com/community/p31205921}.}
\begin{mdframed}[style=mdpurplebox,frametitle={Problem statement}]
Find all real numbers $\alpha$ so that, for every positive integer $n$, the integer
\[ \left\lfloor \alpha \right\rfloor + \left\lfloor 2 \alpha \right\rfloor
  + \left\lfloor 3 \alpha \right\rfloor + \dots + \left\lfloor n \alpha \right\rfloor \]
is divisible by $n$.
\end{mdframed}
The answer is that $\alpha$ must be an even integer.
Let $S(n, \alpha)$ denote the sum in question.

\paragraph{Analysis for $\alpha$ an integer.}
If $\alpha$ is an integer, then the sum equals
\[ S(n, \alpha) = (1+2+\dots+n) \alpha = \frac{n(n+1)}{2} \cdot \alpha \]
which is obviously a multiple of $n$ if $2 \mid \alpha$;
meanwhile, if $\alpha$ is an odd integer then $n = 2$ gives a counterexample.

\paragraph{Main case.}
Suppose $\alpha$ is not an integer; we show the desired condition can never be true.
Note that replacing $\alpha$ with $\alpha \pm 2$ changes by
\[ S(n \pm 2, \alpha) - S(n, \alpha) = 2(1+2+\dots+n) = n(n+1) \equiv 0 \pmod n \]
for every $n$.
Thus, by shifting appropriately we may assume $-1 < \alpha < 1$ and $\alpha \notin \ZZ$.

\begin{itemize}
  \ii If $0 < \alpha < 1$,
  then let $m \ge 2$ be the smallest integer such that $m \alpha \ge 1$.
  Then
  \[ S(m, \alpha) = \underbrace{0 + \dots + 0}_{m-1\text{ terms}} + 1 = 1 \]
  is not a multiple of $m$.

  \ii If $-1 < \alpha < 0$,
  then let $m \ge 2$ be the smallest integer such that $m \alpha \le -1$.
  Then
  \[ S(m, \alpha) = \underbrace{(-1) + \dots + (-1)}_{m-1\text{ terms}} + 0 = -(m-1) \]
  is not a multiple of $m$.
\end{itemize}
\pagebreak