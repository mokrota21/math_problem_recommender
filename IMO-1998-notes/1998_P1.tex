\textsl{Available online at \url{https://aops.com/community/p124387}.}
\begin{mdframed}[style=mdpurplebox,frametitle={Problem statement}]
A convex quadrilateral $ABCD$ has perpendicular diagonals.
The perpendicular bisectors of the sides $AB$ and $CD$ meet
at a unique point $P$ inside $ABCD$.
Prove that the quadrilateral $ABCD$ is cyclic
if and only if triangles $ABP$ and $CDP$ have equal areas.
\end{mdframed}
If $ABCD$ is cyclic, then $P$ is the circumcenter,
and $\angle APB + \angle PCD = 180\dg$.
The hard part is the converse.

\begin{center}
\begin{asy}
import graph; size(10cm);
pen zzttqq = rgb(0.6,0.2,0.); pen aqaqaq = rgb(0.62745,0.62745,0.62745); pen cqcqcq = rgb(0.75294,0.75294,0.75294);
draw((2.28154,-9.32038)--(8.62921,-11.79133)--(2.93422,-2.54008)--cycle, linewidth(0.6) + zzttqq);
draw((2.28154,-9.32038)--(-2.,-4.)--(-4.,-12.)--cycle, linewidth(0.6) + zzttqq);

draw((8.62921,-11.79133)--(-2.,-4.), linewidth(0.6));
draw((2.93422,-2.54008)--(-4.,-12.), linewidth(0.6));
draw((8.62921,-11.79133)--(2.93422,-2.54008), linewidth(0.6));
draw((-2.,-4.)--(-4.,-12.), linewidth(0.6));
draw((2.28154,-9.32038)--(5.78172,-7.16571), linewidth(0.6) + blue);
draw((2.28154,-9.32038)--(-3.,-8.), linewidth(0.6) + blue);
draw((2.28154,-9.32038)--(8.62921,-11.79133), linewidth(0.6) + zzttqq);
draw((8.62921,-11.79133)--(2.93422,-2.54008), linewidth(0.6) + zzttqq);
draw((2.93422,-2.54008)--(2.28154,-9.32038), linewidth(0.6) + zzttqq);
draw((2.28154,-9.32038)--(-2.,-4.), linewidth(0.6) + zzttqq);
draw((-2.,-4.)--(-4.,-12.), linewidth(0.6) + zzttqq);
draw((-4.,-12.)--(2.28154,-9.32038), linewidth(0.6) + zzttqq);
draw((2.93422,-2.54008)--(-0.31533,2.73866), linewidth(0.6));
draw((-2.,-4.)--(-0.31533,2.73866), linewidth(0.6));
draw(circle((2.25532,-9.31383), 6.80768), linewidth(0.6) + aqaqaq);
draw((0.51354,-5.84245)--(-3.,-8.), linewidth(0.6) + blue);
draw((0.51354,-5.84245)--(5.78172,-7.16571), linewidth(0.6) + blue);
dot("$D$", (-4.,-12.), dir((-76.113, -27.810)));
dot("$C$", (-2.,-4.), dir((-75.616, 41.734)));
dot("$B$", (2.93422,-2.54008), dir((1.940, 37.397)));
dot("$A$", (8.62921,-11.79133), dir((31.751, -33.422)));
dot("$P$", (2.28154,-9.32038), dir((-11.247, -66.945)));
dot("$M$", (5.78172,-7.16571), dir((39.728, 27.050)));
dot("$N$", (-3.,-8.), dir((-82.402, 23.307)));
dot("$E$", (0.51354,-5.84245), dir((-30.585, 62.531)));
dot("$X$", (-0.31533,2.73866), dir((-23.973, 36.915)));
\end{asy}
\end{center}

Let $M$ and $N$ be the midpoints of $\ol{AB}$ and $\ol{CD}$.
\begin{claim*}
  Unconditionally, we have $\dang NEM = \dang MPN$.
\end{claim*}
\begin{proof}
  Note that $\ol{EN}$ is the median of right triangle $\triangle ECD$, and similarly for $\ol{EM}$.
  Hence $\dang NED = \dang EDN = \dang BDC$, while $\dang AEM = \dang ACB$.
  Since $\dang DEA = 90\dg$, by looking at quadrilateral $XDEA$ where $X = \ol{CD} \cap \ol{AB}$,
  we derive that $\dang NED + \dang AEM + \dang DXA = 90\dg$, so
  \[ \dang NEM = \dang NED + \dang AEM + 90\dg = -\dang DXA = -\dang NXM = -\dang NPM \]
  as needed.
\end{proof}

However, the area condition in the problem tells us
\[ \frac{EN}{EM} = \frac{CN}{CM} = \frac{PM}{PN}. \]
Finally, we have $\angle MEN > 90\dg$ from the configuration.
These properties uniquely determine the point $E$:
it is the reflection of $P$ across the midpoint of $MN$.

So $EMPN$ is a parallelogram, and thus $\ol{ME} \perp \ol{CD}$.
This implies $\dang BAE = \dang CEM = \dang EDC$ giving $ABCD$ cyclic.
\pagebreak