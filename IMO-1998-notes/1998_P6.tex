
\textsl{Available online at \url{https://aops.com/community/p124426}.}
\begin{mdframed}[style=mdpurplebox,frametitle={Problem statement}]
Classify all functions $f \colon \NN \to \NN$
satisfying the identity
\[ f(n^2 f(m)) = m f(n)^2. \]
\end{mdframed}
Let $\mathcal P$ be the set of primes,
and let $g \colon \mathcal P \to \mathcal P$ be any involution on them.
Extend $g$ to a completely multiplicative function on $\NN$.
Then $f(n) = d g(n)$ is a solution for any $d \in \NN$
which is fixed by $g$.

It's straightforward to check these all work,
since $g \colon \NN \to \NN$ is an involution on them.
So we prove these are the only functions.

Let $d = f(1)$.
\begin{claim*}
  We have $df(n) = f(dn)$ and $d \cdot f(ab) = f(a) f(b)$.
\end{claim*}
\begin{proof}
  Let $P(m,n)$ denote the assertion in the problem statement.
  Off the bat,
  \begin{itemize}
    \ii $P(1,1)$ implies $f(d) = d^2$.
    \ii $P(n,1)$ implies $f(f(n)) = d^2n$.
    In particular, $f$ is injective.
    \ii $P(1,n)$ implies $f(dn^2) = f(n)^2$.
  \end{itemize}
  Then
  \begin{align*}
    f(a)^2 f(b)^2 &= f(da^2) f(b)^2 & \text{by third bullet}\\
    &= f(b^2 f(f(da^2))) & \text{by problem statement} \\
    &= f(b^2 \cdot d^2 \cdot da^2) & \text{by second bullet} \\
    &= f(dab)^2 & \text{by third bullet} \\
    \implies f(a) f(b) &= f(dab).
  \end{align*}
  This implies the first claim by taking $(a,b) = (1,n)$.
  Then $df(a) = f(da)$, and so we actually have
  $f(a) f(b) = d f(ab)$.
\end{proof}
\begin{claim*}
  All values of $f$ are divisible by $d$.
\end{claim*}
\begin{proof}
  We have
  \begin{align*}
    f(n^2) &= \frac 1d f(n)^2 \\
    f(n^3) &= \frac{f(n^2) f(n)}{d} = \frac{f(n)^3}{d^2} \\
    f(n^4) &= \frac{f(n^3) f(n)}{d} = \frac{f(n)^4}{d^3}
  \end{align*}
  and so on,
  which implies the result.
\end{proof}
Then, define $g(n) = f(n) / d$.
We conclude that $g$ is completely multiplicative, with $g(1) = 1$.
However, $f(f(n)) = d^2n$ also implies $g(g(n)) = n$,
i.e.\ $g$ is an involution.
Moreover, since $f(d) = d^2$, $g(d) = d$.

All that remains is to check that $g$ must map primes to primes
to finish the description in the problem.
This is immediate; since $g$ is multiplicative and $g(1) = 1$,
if $g(g(p)) = p$ then $g(p)$ can have at most one prime factor,
hence $g(p)$ is itself prime.

\begin{remark*}
  The IMO problem actually asked for the least value of $f(1998)$.
  But for instruction purposes,
  it is probably better to just find all $f$.
  Since $1998 = 2 \cdot 3^3 \cdot 37$,
  this answer is $2^3 \cdot 3 \cdot 5 = 120$, anyways.
\end{remark*}
\pagebreak


\end{document}
