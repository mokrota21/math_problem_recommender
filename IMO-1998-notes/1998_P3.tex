\textsl{Available online at \url{https://aops.com/community/p124439}.}
\begin{mdframed}[style=mdpurplebox,frametitle={Problem statement}]
For any positive integer $n$,
let $\tau(n)$ denote the number of its positive divisors (including $1$ and itself).
Determine all positive integers $m$ for which
there exists a positive integer $n$ such that
\[ \frac{\tau(n^{2})}{\tau(n)}=m. \]
\end{mdframed}
The answer is odd integers $m$ only.
If we write $n = p_1^{e_1} \dots p_k^{e_k}$ we get
\[ \prod \frac{2e_i+1}{e_i+1} = m. \]
It's clear now that $m$ must be odd,
since every fraction has odd numerator.

We now endeavor to construct odd numbers.
The proof is by induction, in which we are curating sets of
fractions of the form $\frac{2e+1}{e+1}$ that multiply
to a given target.

The base cases are easy to verify by hand.
Generally, assume $p = 2^t k - 1$ is odd, where $k$ is odd.
Then we can write
\[
  \frac{2^{2t}k-2^t(k+1)+1}{2^{2t-1}k-2^{t-1}(k+1)+1}
  \cdot
  \frac{2^{2t-1}k-2^{t-1}(k+1)+1}{2^{2t-2}k-2^{t-2}(k+1)+1}
  \cdot \dots \cdot
  \frac{2^{t+1}k-2(k+1)+1}{2^tk-2^0(k+1)+1}.
\]
Note that $2^{2t}k-2^t(k+1)+1 = (2^t k - 1)(2^t - 1)$,
and $2^t k - k = k(2^t-1)$, so the above fraction simplifies to
\[ \frac{2^t k - 1}{k} \]
meaning we just need to multiply by $k$,
which we can do using induction hypothesis.
\pagebreak

\section{Solutions to Day 2}