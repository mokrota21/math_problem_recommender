\textsl{Available online at \url{https://aops.com/community/p121417}.}
\begin{mdframed}[style=mdpurplebox,frametitle={Problem statement}]
Let $I$ be the incenter of triangle $ABC$.
Let the incircle of $ABC$ touch
the sides $BC$, $CA$, and $AB$ at $K$, $L$, and $M$, respectively.
The line through $B$ parallel to $MK$ meets the lines
$LM$ and $LK$ at $R$ and $S$, respectively.
Prove that angle $RIS$ is acute.
\end{mdframed}
Observe that $\triangle MKL$ is acute with circumcenter $I$.
We now present two proofs.

\paragraph{First simple proof (grobber).}
The problem is equivalent to showing $BI^2 > BR \cdot BS$.
But from
\[ \triangle BRK \sim \triangle MKL \sim \triangle BLS \]
we conclude
\[ BR = t \cdot \frac{MK}{ML},
  \qquad BS = t \cdot \frac{ML}{MK} \]
where $t = BK  = BL$ is the length
of the tangent from $B$.
Hence $BR \cdot BS = t^2$.
Since $BI > t$ is clear, we are done.

\paragraph{Second projective proof.}
Let $N$ be the midpoint of $\ol{KL}$,
and let ray $MN$ meet the incircle again at $P$.

Note that line $\ol{RBS}$ is the polar of $N$.
By Brokard's theorem, lines $MK$ and $PL$ should thus
meet the polar of $N$, so we conclude $R = \ol{MK} \cap \ol{PL}$.
Analogously, $S = \ol{ML} \cap \ol{PK}$.

Again by Brokard's theorem, $\triangle NRS$ is self-polar,
so $N$ is the orthocenter of $\triangle RIS$.
Since $N$ lies between $I$ and $B$ we are done.
\pagebreak