% © Evan Chen
% Downloaded from https://web.evanchen.cc/

\documentclass[11pt]{scrartcl}
\usepackage[sexy]{evan}
\ihead{\footnotesize\textbf{\thetitle}}
\ohead{\footnotesize\href{http://web.evanchen.cc}{\ttfamily web.evanchen.cc},
    updated \today}
\title{IMO 1998 Solution Notes}
\date{\today}

\begin{document}

\maketitle

\begin{abstract}
This is a compilation of solutions
for the 1998 IMO.
The ideas of the solution are a mix of my own work,
the solutions provided by the competition organizers,
and solutions found by the community.
However, all the writing is maintained by me.

These notes will tend to be a bit more advanced and terse than the ``official''
solutions from the organizers.
In particular, if a theorem or technique is not known to beginners
but is still considered ``standard'', then I often prefer to
use this theory anyways, rather than try to work around or conceal it.
For example, in geometry problems I typically use directed angles
without further comment, rather than awkwardly work around configuration issues.
Similarly, sentences like ``let $\mathbb{R}$ denote the set of real numbers''
are typically omitted entirely.

Corrections and comments are welcome!
\end{abstract}

\tableofcontents
\newpage

\addtocounter{section}{-1}
\section{Problems}
\begin{enumerate}[\bfseries 1.]
\item %% Problem 1
A convex quadrilateral $ABCD$ has perpendicular diagonals.
The perpendicular bisectors of the sides $AB$ and $CD$ meet
at a unique point $P$ inside $ABCD$.
Prove that the quadrilateral $ABCD$ is cyclic
if and only if triangles $ABP$ and $CDP$ have equal areas.

\item %% Problem 2
In a competition, there are $a$ contestants
and $b$ judges, where $b \ge 3$ is an odd integer.
Each judge rates each contestant as either ``pass'' or ``fail''.
Suppose $k$ is a number such that for any two judges,
their ratings coincide for at most $k$ contestants.
Prove that
\[ \frac ka \ge \frac{b-1}{2b}. \]

\item %% Problem 3
For any positive integer $n$,
let $\tau(n)$ denote the number of its positive divisors (including $1$ and itself).
Determine all positive integers $m$ for which
there exists a positive integer $n$ such that
\[ \frac{\tau(n^{2})}{\tau(n)}=m. \]

\item %% Problem 4
Determine all pairs $(x,y)$ of positive integers
such that $x^{2}y+x+y$ is divisible by $xy^{2}+y+7$.

\item %% Problem 5
Let $I$ be the incenter of triangle $ABC$.
Let the incircle of $ABC$ touch
the sides $BC$, $CA$, and $AB$ at $K$, $L$, and $M$, respectively.
The line through $B$ parallel to $MK$ meets the lines
$LM$ and $LK$ at $R$ and $S$, respectively.
Prove that angle $RIS$ is acute.

\item %% Problem 6
Classify all functions $f \colon \NN \to \NN$
satisfying the identity
\[ f(n^2 f(m)) = m f(n)^2. \]

\end{enumerate}
\pagebreak

\section{Solutions to Day 1}
\subsection{IMO 1998/1}
\textsl{Available online at \url{https://aops.com/community/p124387}.}
\begin{mdframed}[style=mdpurplebox,frametitle={Problem statement}]
A convex quadrilateral $ABCD$ has perpendicular diagonals.
The perpendicular bisectors of the sides $AB$ and $CD$ meet
at a unique point $P$ inside $ABCD$.
Prove that the quadrilateral $ABCD$ is cyclic
if and only if triangles $ABP$ and $CDP$ have equal areas.
\end{mdframed}
If $ABCD$ is cyclic, then $P$ is the circumcenter,
and $\angle APB + \angle PCD = 180\dg$.
The hard part is the converse.

\begin{center}
\begin{asy}
import graph; size(10cm);
pen zzttqq = rgb(0.6,0.2,0.); pen aqaqaq = rgb(0.62745,0.62745,0.62745); pen cqcqcq = rgb(0.75294,0.75294,0.75294);
draw((2.28154,-9.32038)--(8.62921,-11.79133)--(2.93422,-2.54008)--cycle, linewidth(0.6) + zzttqq);
draw((2.28154,-9.32038)--(-2.,-4.)--(-4.,-12.)--cycle, linewidth(0.6) + zzttqq);

draw((8.62921,-11.79133)--(-2.,-4.), linewidth(0.6));
draw((2.93422,-2.54008)--(-4.,-12.), linewidth(0.6));
draw((8.62921,-11.79133)--(2.93422,-2.54008), linewidth(0.6));
draw((-2.,-4.)--(-4.,-12.), linewidth(0.6));
draw((2.28154,-9.32038)--(5.78172,-7.16571), linewidth(0.6) + blue);
draw((2.28154,-9.32038)--(-3.,-8.), linewidth(0.6) + blue);
draw((2.28154,-9.32038)--(8.62921,-11.79133), linewidth(0.6) + zzttqq);
draw((8.62921,-11.79133)--(2.93422,-2.54008), linewidth(0.6) + zzttqq);
draw((2.93422,-2.54008)--(2.28154,-9.32038), linewidth(0.6) + zzttqq);
draw((2.28154,-9.32038)--(-2.,-4.), linewidth(0.6) + zzttqq);
draw((-2.,-4.)--(-4.,-12.), linewidth(0.6) + zzttqq);
draw((-4.,-12.)--(2.28154,-9.32038), linewidth(0.6) + zzttqq);
draw((2.93422,-2.54008)--(-0.31533,2.73866), linewidth(0.6));
draw((-2.,-4.)--(-0.31533,2.73866), linewidth(0.6));
draw(circle((2.25532,-9.31383), 6.80768), linewidth(0.6) + aqaqaq);
draw((0.51354,-5.84245)--(-3.,-8.), linewidth(0.6) + blue);
draw((0.51354,-5.84245)--(5.78172,-7.16571), linewidth(0.6) + blue);
dot("$D$", (-4.,-12.), dir((-76.113, -27.810)));
dot("$C$", (-2.,-4.), dir((-75.616, 41.734)));
dot("$B$", (2.93422,-2.54008), dir((1.940, 37.397)));
dot("$A$", (8.62921,-11.79133), dir((31.751, -33.422)));
dot("$P$", (2.28154,-9.32038), dir((-11.247, -66.945)));
dot("$M$", (5.78172,-7.16571), dir((39.728, 27.050)));
dot("$N$", (-3.,-8.), dir((-82.402, 23.307)));
dot("$E$", (0.51354,-5.84245), dir((-30.585, 62.531)));
dot("$X$", (-0.31533,2.73866), dir((-23.973, 36.915)));
\end{asy}
\end{center}

Let $M$ and $N$ be the midpoints of $\ol{AB}$ and $\ol{CD}$.
\begin{claim*}
  Unconditionally, we have $\dang NEM = \dang MPN$.
\end{claim*}
\begin{proof}
  Note that $\ol{EN}$ is the median of right triangle $\triangle ECD$, and similarly for $\ol{EM}$.
  Hence $\dang NED = \dang EDN = \dang BDC$, while $\dang AEM = \dang ACB$.
  Since $\dang DEA = 90\dg$, by looking at quadrilateral $XDEA$ where $X = \ol{CD} \cap \ol{AB}$,
  we derive that $\dang NED + \dang AEM + \dang DXA = 90\dg$, so
  \[ \dang NEM = \dang NED + \dang AEM + 90\dg = -\dang DXA = -\dang NXM = -\dang NPM \]
  as needed.
\end{proof}

However, the area condition in the problem tells us
\[ \frac{EN}{EM} = \frac{CN}{CM} = \frac{PM}{PN}. \]
Finally, we have $\angle MEN > 90\dg$ from the configuration.
These properties uniquely determine the point $E$:
it is the reflection of $P$ across the midpoint of $MN$.

So $EMPN$ is a parallelogram, and thus $\ol{ME} \perp \ol{CD}$.
This implies $\dang BAE = \dang CEM = \dang EDC$ giving $ABCD$ cyclic.
\pagebreak

\subsection{IMO 1998/2}
\textsl{Available online at \url{https://aops.com/community/p124458}.}
\begin{mdframed}[style=mdpurplebox,frametitle={Problem statement}]
In a competition, there are $a$ contestants
and $b$ judges, where $b \ge 3$ is an odd integer.
Each judge rates each contestant as either ``pass'' or ``fail''.
Suppose $k$ is a number such that for any two judges,
their ratings coincide for at most $k$ contestants.
Prove that
\[ \frac ka \ge \frac{b-1}{2b}. \]
\end{mdframed}
This is a ``routine'' problem with global ideas.
We count pairs of coinciding ratings,
i.e.\ the number $N$ of tuples
\[(\{J_1, J_2\}, C) \]
of two distinct judges and a contestant
for which the judges gave the same rating.

On the one hand, if we count by the judges,
we have \[ N \le \binom b2 k \]
by he problem condition.

On the other hand, if $b=2m+1$, then
each contestant $C$
contributes at least $\binom{m}{2} + \binom{m+1}{2} = m^2$ to $N$,
and so
\[ N \ge a \cdot \left( \frac{b-1}{2} \right)^2 \]
Putting together the two estimates for $N$ yields the conclusion.
\pagebreak

\subsection{IMO 1998/3}
\textsl{Available online at \url{https://aops.com/community/p124439}.}
\begin{mdframed}[style=mdpurplebox,frametitle={Problem statement}]
For any positive integer $n$,
let $\tau(n)$ denote the number of its positive divisors (including $1$ and itself).
Determine all positive integers $m$ for which
there exists a positive integer $n$ such that
\[ \frac{\tau(n^{2})}{\tau(n)}=m. \]
\end{mdframed}
The answer is odd integers $m$ only.
If we write $n = p_1^{e_1} \dots p_k^{e_k}$ we get
\[ \prod \frac{2e_i+1}{e_i+1} = m. \]
It's clear now that $m$ must be odd,
since every fraction has odd numerator.

We now endeavor to construct odd numbers.
The proof is by induction, in which we are curating sets of
fractions of the form $\frac{2e+1}{e+1}$ that multiply
to a given target.

The base cases are easy to verify by hand.
Generally, assume $p = 2^t k - 1$ is odd, where $k$ is odd.
Then we can write
\[
  \frac{2^{2t}k-2^t(k+1)+1}{2^{2t-1}k-2^{t-1}(k+1)+1}
  \cdot
  \frac{2^{2t-1}k-2^{t-1}(k+1)+1}{2^{2t-2}k-2^{t-2}(k+1)+1}
  \cdot \dots \cdot
  \frac{2^{t+1}k-2(k+1)+1}{2^tk-2^0(k+1)+1}.
\]
Note that $2^{2t}k-2^t(k+1)+1 = (2^t k - 1)(2^t - 1)$,
and $2^t k - k = k(2^t-1)$, so the above fraction simplifies to
\[ \frac{2^t k - 1}{k} \]
meaning we just need to multiply by $k$,
which we can do using induction hypothesis.
\pagebreak

\section{Solutions to Day 2}
\subsection{IMO 1998/4}
\textsl{Available online at \url{https://aops.com/community/p124428}.}
\begin{mdframed}[style=mdpurplebox,frametitle={Problem statement}]
Determine all pairs $(x,y)$ of positive integers
such that $x^{2}y+x+y$ is divisible by $xy^{2}+y+7$.
\end{mdframed}
The answer is $(7k^2,7k)$ for all $k \ge 1$,
as well as $(11,1)$ and $(49,1)$.

We are given $xy^2+y+7 \mid x^2y+x+y$.
Multiplying the right-hand side by $y$ gives
\[ xy^2+y+7 \mid x^2y^2+xy+y^2 \]
Then subtracting $x$ times the left-hand side gives
\[ xy^2+y+7 \mid y^2-7x. \]
We consider cases based on the sign of $y^2=7x$.
\begin{itemize}
  \ii If $y^2 > 7x$, then $0 < y^2-7x < xy^2+y+7$,
  contradiction.
  \ii If $y^2=7x$, let $y = 7k$, so $x = 7k^2$.
  Plugging this back in to the original equation reads
  \[ 343k^4 + 7k + 7 \mid 343k^5 + 7k^2 + 7k \]
  which is always valid, hence these are all solutions.
  \ii  If $y^2 < 7x$, then $|y^2-7x| \le 7x$,
  so $y \in \{1,2\}$.

  When $y=1$ we get
  \[ x+8 \mid x^2+x+1 \iff x+8 \mid 64-8+1=57.\]
  This has solutions $x=11$ and $x=49$.

  When $y=2$
  \begin{align*}
    4x+9 \mid 2x^2+x+2 \\
    \implies 4x+9 &\mid 16x^2+8x+16 \\
    \implies 4x+9 &\mid 81-18+16 = 79
  \end{align*}
  which never occurs.
\end{itemize}
\pagebreak

\subsection{IMO 1998/5}
\textsl{Available online at \url{https://aops.com/community/p121417}.}
\begin{mdframed}[style=mdpurplebox,frametitle={Problem statement}]
Let $I$ be the incenter of triangle $ABC$.
Let the incircle of $ABC$ touch
the sides $BC$, $CA$, and $AB$ at $K$, $L$, and $M$, respectively.
The line through $B$ parallel to $MK$ meets the lines
$LM$ and $LK$ at $R$ and $S$, respectively.
Prove that angle $RIS$ is acute.
\end{mdframed}
Observe that $\triangle MKL$ is acute with circumcenter $I$.
We now present two proofs.

\paragraph{First simple proof (grobber).}
The problem is equivalent to showing $BI^2 > BR \cdot BS$.
But from
\[ \triangle BRK \sim \triangle MKL \sim \triangle BLS \]
we conclude
\[ BR = t \cdot \frac{MK}{ML},
  \qquad BS = t \cdot \frac{ML}{MK} \]
where $t = BK  = BL$ is the length
of the tangent from $B$.
Hence $BR \cdot BS = t^2$.
Since $BI > t$ is clear, we are done.

\paragraph{Second projective proof.}
Let $N$ be the midpoint of $\ol{KL}$,
and let ray $MN$ meet the incircle again at $P$.

Note that line $\ol{RBS}$ is the polar of $N$.
By Brokard's theorem, lines $MK$ and $PL$ should thus
meet the polar of $N$, so we conclude $R = \ol{MK} \cap \ol{PL}$.
Analogously, $S = \ol{ML} \cap \ol{PK}$.

Again by Brokard's theorem, $\triangle NRS$ is self-polar,
so $N$ is the orthocenter of $\triangle RIS$.
Since $N$ lies between $I$ and $B$ we are done.
\pagebreak

\subsection{IMO 1998/6}
\textsl{Available online at \url{https://aops.com/community/p124426}.}
\begin{mdframed}[style=mdpurplebox,frametitle={Problem statement}]
Classify all functions $f \colon \NN \to \NN$
satisfying the identity
\[ f(n^2 f(m)) = m f(n)^2. \]
\end{mdframed}
Let $\mathcal P$ be the set of primes,
and let $g \colon \mathcal P \to \mathcal P$ be any involution on them.
Extend $g$ to a completely multiplicative function on $\NN$.
Then $f(n) = d g(n)$ is a solution for any $d \in \NN$
which is fixed by $g$.

It's straightforward to check these all work,
since $g \colon \NN \to \NN$ is an involution on them.
So we prove these are the only functions.

Let $d = f(1)$.
\begin{claim*}
  We have $df(n) = f(dn)$ and $d \cdot f(ab) = f(a) f(b)$.
\end{claim*}
\begin{proof}
  Let $P(m,n)$ denote the assertion in the problem statement.
  Off the bat,
  \begin{itemize}
    \ii $P(1,1)$ implies $f(d) = d^2$.
    \ii $P(n,1)$ implies $f(f(n)) = d^2n$.
    In particular, $f$ is injective.
    \ii $P(1,n)$ implies $f(dn^2) = f(n)^2$.
  \end{itemize}
  Then
  \begin{align*}
    f(a)^2 f(b)^2 &= f(da^2) f(b)^2 & \text{by third bullet}\\
    &= f(b^2 f(f(da^2))) & \text{by problem statement} \\
    &= f(b^2 \cdot d^2 \cdot da^2) & \text{by second bullet} \\
    &= f(dab)^2 & \text{by third bullet} \\
    \implies f(a) f(b) &= f(dab).
  \end{align*}
  This implies the first claim by taking $(a,b) = (1,n)$.
  Then $df(a) = f(da)$, and so we actually have
  $f(a) f(b) = d f(ab)$.
\end{proof}
\begin{claim*}
  All values of $f$ are divisible by $d$.
\end{claim*}
\begin{proof}
  We have
  \begin{align*}
    f(n^2) &= \frac 1d f(n)^2 \\
    f(n^3) &= \frac{f(n^2) f(n)}{d} = \frac{f(n)^3}{d^2} \\
    f(n^4) &= \frac{f(n^3) f(n)}{d} = \frac{f(n)^4}{d^3}
  \end{align*}
  and so on,
  which implies the result.
\end{proof}
Then, define $g(n) = f(n) / d$.
We conclude that $g$ is completely multiplicative, with $g(1) = 1$.
However, $f(f(n)) = d^2n$ also implies $g(g(n)) = n$,
i.e.\ $g$ is an involution.
Moreover, since $f(d) = d^2$, $g(d) = d$.

All that remains is to check that $g$ must map primes to primes
to finish the description in the problem.
This is immediate; since $g$ is multiplicative and $g(1) = 1$,
if $g(g(p)) = p$ then $g(p)$ can have at most one prime factor,
hence $g(p)$ is itself prime.

\begin{remark*}
  The IMO problem actually asked for the least value of $f(1998)$.
  But for instruction purposes,
  it is probably better to just find all $f$.
  Since $1998 = 2 \cdot 3^3 \cdot 37$,
  this answer is $2^3 \cdot 3 \cdot 5 = 120$, anyways.
\end{remark*}
\pagebreak


\end{document}
