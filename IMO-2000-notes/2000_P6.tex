
\textsl{Available online at \url{https://aops.com/community/p351094}.}
\begin{mdframed}[style=mdpurplebox,frametitle={Problem statement}]
Let $\ol{AH_1}$, $\ol{BH_2}$, and $\ol{CH_3}$ be the
altitudes of an acute triangle $ABC$.
The incircle $\omega$ of triangle $ABC$ touches the sides
$BC$, $CA$ and $AB$ at $T_1$, $T_2$ and $T_3$, respectively.
Consider the reflections of the lines $H_1H_2$, $H_2H_3$, and
$H_3H_1$ with respect to the lines $T_1T_2$, $T_2T_3$, and $T_3T_1$.
Prove that these images form a triangle whose vertices lie on $\omega$.
\end{mdframed}
We use complex numbers with $\omega$ the unit circle.
Let $T_1 = a$, $T_2 = b$, $T_3 = c$.
The main content of the problem is to show that
the triangle in question has vertices
$ab/c$, $bc/a$, $ca/b$
(which is evident from a good diagram).

Since $A = \frac{2bc}{b+c}$, we have
\[ H_1 = \half \left( \frac{2bc}{b+c} + a + a - a^2 \cdot
  \frac{2}{b+c} \right)
  = \frac{ab+bc+ca-a^2}{b+c}. \]
The reflection of $H_1$ over $\ol{T_1 T_2}$ is
\begin{align*}
  H_1^C &= a + b - ab \ol{H_1}
    = a + b - b \cdot \frac{ac+ab+a^2-bc}{a(b+c)} \\
  &= \frac{a(a+b)(b+c) - b(a^2+ab+ac-bc)}{a(b+c)}
    = \frac{c(a^2+b^2)}{a(b+c)}.
\end{align*}
Now, we claim that $H_1^C$ lies on the chord joining
$\frac{ca}{b}$ and $\frac{cb}{a}$;
by symmetry so will $H_2^C$
and this will imply the problem
(it means that the desired triangle has vertices
$ab/c$, $bc/a$, $ca/b$).
A cartoon of this is shown below.
\begin{center}
\begin{asy}
  pair A = dir(110);
  pair B = dir(210);
  pair C = dir(330);
  dot("$\frac{bc}{a}$", A, dir(10), blue);
  dot("$\frac{ca}{b}$", B, dir(140), blue);
  dot("$\frac{ab}{c}$", C, dir(50), blue);
  dot("$H_1^C$", 2*A-B, dir(A-B));
  dot("$H_2^C$", 2*B-A, dir(B-A));
  dot("$H_2^A$", 2*B-C, dir(B-C));
  dot("$H_3^A$", 2*C-B, dir(C-B));
  dot("$H_3^B$", 2*C-A, dir(C-A));
  dot("$H_1^B$", 2*A-C, dir(A-C));
  draw( (2*A-B)--(2*B-A) );
  draw( (2*B-C)--(2*C-B) );
  draw( (2*C-A)--(2*A-C) );
\end{asy}
\end{center}
To see this, it suffices to compute
\begin{align*}
  H_1^C + \left( \frac{ca}{b} \right)\left( \frac{cb}{a} \right) \ol{H_1^C}
  &= \frac{c(a^2+b^2)}{a(b+c)}
    + c^2 \frac{\frac 1c \cdot \frac{a^2+b^2}{a^2b^2}}%
    {\frac1a\left( \frac{b+c}{bc} \right)} \\
  &= \frac{c(a^2+b^2)}{a(b+c)}
    + \frac{c(a^2+b^2)}{abc\inv(b+c)} \\
  &= \frac{c(a^2+b^2)}{a(b+c)} \left( \frac{b+c}{b} \right) \\
  &= \frac{c(a^2+b^2)}{ab}= \frac{ca}{b} + \frac{cb}{a}
\end{align*}
as desired.
\pagebreak


\end{document}
