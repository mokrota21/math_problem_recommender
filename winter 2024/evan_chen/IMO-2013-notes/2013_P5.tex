\textsl{Available online at \url{https://aops.com/community/p5720286}.}
\begin{mdframed}[style=mdpurplebox,frametitle={Problem statement}]
Suppose a function $f \colon \QQ_{>0} \to \RR$ satisfies:
\begin{enumerate}
  \ii [(i)] If $x,y \in \QQ_{>0}$, then $f(x)f(y) \ge f(xy)$.
  \ii [(ii)] If $x,y \in \QQ_{>0}$, then $f(x+y) \ge f(x) + f(y)$.
  \ii [(iii)] There exists a rational number $a > 1$ with $f(a) = a$.
\end{enumerate}
Prove that $f(x) = x$ for all positive rational numbers $x$.
\end{mdframed}
First, we dispense of negative situations by proving:
\begin{claim*}
  For any integer $n > 0$, we have $f(n) \ge n$.
\end{claim*}
\begin{proof}
  Note by induction on (ii) we have $f(nx) \ge n f(x)$.
  Taking $(x,y) = (a,1)$ in (i) gives $f(1) \ge 1$,
  and hence $f(n) \ge n$.
\end{proof}

\begin{claim*}
  The $f$ takes only positive values,
  and hence by (ii) is strictly increasing.
\end{claim*}
\begin{proof}[Proof, suggested by Gopal Goel]
  Let $p,q > 0$ be integers.
  Then $f(q) f(p/q) \ge f(p)$,
  and since both $\min(f(p), f(q)) > 0$
  it follows $f(p/q) > 0$.
\end{proof}

\begin{claim*}
  For any $x > 1$ we have $f(x) \ge x$.
\end{claim*}
\begin{proof}
  Note that
  \[ f(x)^N \ge f(x^N) \ge f\left( \left\lfloor x^N \right\rfloor \right)
    \ge \left\lfloor x^N \right\rfloor > x^N-1 \]
  for any integer $N$.
  Since $N$ can be arbitrarily large,
  we conclude $f(x) \ge x$ for $x > 1$.
\end{proof}

On the other hand, $f$ has arbitrarily large fixed points
(namely powers of $a$) so from (ii) we're essentially done.
First, for $x > 1$ pick a large $m$ and note
\[ a^m = f(a^m) \ge f(a^m-x) + f(x) \ge (a^m-x)+x = a^m. \]
Finally, for $x \le 1$ use
\[ nf(x) = f(n)f(x) \ge f(nx) \ge nf(x) \]
for large $n$.

\begin{remark*}
Note that $a > 1$ is essential;
if $b \ge 1$ then $f(x) = bx^2$ works with unique fixed point $1/b \le 1$.
\end{remark*}
\pagebreak