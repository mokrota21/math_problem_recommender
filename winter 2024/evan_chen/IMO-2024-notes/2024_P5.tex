\textsl{Available online at \url{https://aops.com/community/p31218774}.}
\begin{mdframed}[style=mdpurplebox,frametitle={Problem statement}]
Turbo the snail is in the top row of a grid with $2024$ rows and $2023$ columns
and wants to get to the bottom row.
However, there are $2022$ hidden monsters, one in every row except the first and last,
with no two monsters in the same column.

Turbo makes a series of attempts to go from the first row to the last row.
On each attempt, he chooses to start on any cell in the first row,
then repeatedly moves to an orthogonal neighbor.
(He is allowed to return to a previously visited cell.)
If Turbo reaches a cell with a monster,
his attempt ends and he is transported back to the first row to start a new attempt.
The monsters do not move between attempts, and Turbo remembers whether or not each cell
he has visited contains a monster.
If he reaches any cell in the last row, his attempt ends and Turbo wins.

Find the smallest integer $n$ such that Turbo has a strategy which guarantees
being able to reach the bottom row in at most $n$ attempts,
regardless of how the monsters are placed.
\end{mdframed}
Surprisingly the answer is $n = 3$ for \emph{any} grid size $s \times (s-1)$ when $s \ge 4$.
We prove this in that generality.

\paragraph{Proof that at least three attempts are needed.}
When Turbo first moves into the second row, Turbo could encounter a monster $M_1$ right away.
Then on the next attempt, Turbo must enter the third row in different column as $M_1$,
and again could encounter a monster $M_2$ right after doing so.
This means no strategy can guarantee fewer than three attempts.

\paragraph{Strategy with three attempts.}
On the first attempt, we have Turbo walk through the entire second row
until he finds the monster $M_1$ in it.
Then we get two possible cases.

\subparagraph{Case where $M_1$ is not on the edge.}
In the first case, if that monster $M_1$ is not on the edge of the row,
then Turbo can trace two paths below it as shown below.
At least one of these paths works, hence three attempts is sufficient.

\begin{center}
\begin{asy}
usepackage("amssymb");
unitsize(0.7cm);
pen gr = grey+linetype("4 2");
int n = 6;
for (int i=0; i<=n-1; ++i) {
  draw((0,i)--(n,i), gr);
}
for (int i=0; i<=n; ++i) {
  draw((i,0)--(i,n-1), gr);
}
draw(box((0,-1), (n,0)), black);
draw(box((0,n-1), (n,n)), black);
draw(box((0,-1), (n,n)), black);
label((n/2,n-0.5), "Starting row");
label((n/2,-0.5), "Goal row");

label((2.5,4.5), "$M_1$", red);
dotfactor *= 2;
dot((1.5,4.5), deepgreen);
dot((3.5,4.5), deepgreen);
draw((1.5,4.5)--(1.5,3.5)--(2.3,3.5)--(2.3,0.2), deepgreen+1.2, EndArrow(TeXHead));
draw((3.5,4.5)--(3.5,3.5)--(2.7,3.5)--(2.7,0.2), deepgreen+1.2, EndArrow(TeXHead));
\end{asy}
\end{center}

\subparagraph{Case where $M_1$ is on the edge.}
WLOG, $M_1$ is in the leftmost cell.
Then Turbo follows the green staircase pattern shown in the left figure below.
If the staircase is free of monsters, then Turbo wins on the second attempt.
Otherwise, if a monster $M_2$ is encountered on the staircase,
Turbo has found a safe path to the left of $M_2$;
then Turbo can use this to reach the column $M_1$ is in, and escape from there.
This is shown in purple in the center and right figure
(there are two slightly different cases depending on whether $M_2$
was encountered going east or south).
\begin{center}
\begin{asy}
usepackage("amssymb");
unitsize(0.65cm);
pen gr = grey+linetype("4 2");
int n = 6;
dotfactor *= 2;

picture pic1, pic2, pic3;
picture[] pics = {pic1, pic2, pic3};

for (int j=0; j<3; ++j) {
  for (int i=0; i<=n-1; ++i) {
    draw(pics[j], (0,i)--(n,i), gr);
  }
  for (int i=0; i<=n; ++i) {
    draw(pics[j], (i,0)--(i,n-1), gr);
  }
  draw(pics[j], box((0,-1), (n,0)), black);
  draw(pics[j], box((0,n-1), (n,n)), black);
  draw(pics[j], box((0,-1), (n,n)), black);
  label(pics[j], "Starting row", (n/2,n-0.5));
  label(pics[j], "Goal row", (n/2,-0.5));
}

label(pic1, "$M_1$", (0.5,4.5), red);
dot(pic1, (1.5,4.5), deepgreen);
draw(pic1, (1.5,4.5)--(2.5,4.5)--(2.5,3.5)--(3.5,3.5)--(3.5,2.5)
  --(4.5,2.5)--(4.5,1.5)--(5.5,1.5)--(5.5,0.2), deepgreen+1.2, EndArrow(TeXHead));

label(pic2, "$M_1$", (0.5,4.5), red);
label(pic2, "$M_2$", (4.5,2.5), red);
dot(pic2, (1.5,4.5), deepgreen);
draw(pic2, (3.5,2.5)--(4.5,2.5)--(4.5,1.5)--(5.5,1.5)--(5.5,0.2), deepgreen+dashed);
draw(pic2, (1.5,4.5)--(2.5,4.5)--(2.5,3.5)--(3.5,3.5)--(3.5,2.5)--(0.5,2.5)--(0.5,0.2),
  purple+1.5, EndArrow(TeXHead));

label(pic3, "$M_1$", (0.5,4.5), red);
label(pic3, "$M_2$", (4.5,1.5), red);
dot(pic3, (1.5,4.5), deepgreen);
draw(pic3, (3.5,2.5)--(4.5,2.5)--(4.5,1.5)--(5.5,1.5)--(5.5,0.2), deepgreen+dashed);
draw(pic3, (1.5,4.5)--(2.5,4.5)--(2.5,3.5)--(3.5,3.5)--(3.5,1.5)--(0.5,1.5)--(0.5,0.2),
  purple+1.5, EndArrow(TeXHead));

add(pic1);
add(shift(7,0)*pic2);
add(shift(14,0)*pic3);
\end{asy}
\end{center}
Thus the problem is solved in three attempts, as promised.

\paragraph{Extended remark: all working strategies look similar to this.}
As far as we know, all working strategies are variations of the above.
In fact, we will try to give a description of the space of possible strategies,
although this needs a bit of notation.

\begin{definition*}
For simplicity, we use $s$ even only in the figures below.
We define the \emph{happy triangle} as the following cells:
\begin{itemize}
  \item All $s-1$ cells in the first row (which has no monsters).
  \item The center $s-3$ cells in the second row.
  \item The center $s-5$ cells in the third row.
  \item \dots
  \item The center cell in the $\frac s2$\textsuperscript{th} row.
\end{itemize}
\end{definition*}
For $s=12$, the happy triangle is the region shaded in the thick border below.
\begin{center}
\begin{asy}
usepackage("amssymb");
unitsize(0.7cm);
pen gr = grey+linetype("4 2");
void setup(int n) {
  for (int i=0; i<=n-1; ++i) {
    draw((0,i)--(n,i), gr);
  }
  for (int i=0; i<=n; ++i) {
    draw((i,0)--(i,n-1), gr);
  }
  draw(box((0,-1), (n,0)), black);
  draw(box((0,n-1), (n,n)), black);
  draw(box((0,-1), (n,n)), black);
  label((n/2,n-0.5), "Starting row");
  label((n/2,-0.5), "Goal row");

  path p = (0,n);
  for (int i=0; i<n/2; ++i) {
    p = p--(i,n-1-i)--(i+1,n-1-i);
  }
  for (int i=(n+1)#2; i<n; ++i) {
    p = p--(i,i)--(i+1,i);
  }
  p = p--(n,n)--cycle;
  filldraw(p, opacity(0.15)+yellow, blue+1.8);
}
setup(11);
\end{asy}
\end{center}
\begin{definition*}
  Given a cell, define a \emph{shoulder} to be the cell directly northwest or northeast of it.
  Hence there are two shoulders of cells outside the first and last column,
  and one shoulder otherwise.
\end{definition*}

Then solutions roughly must distinguish between these two cases:
\begin{itemize}
  \item \textbf{Inside happy triangle:}
    If the first monster $\color{red}M_1$ is found in the \emph{happy triangle},
    and there is a safe path found by Turbo to the two shoulders
    (marked $\color{green!60!black}\bigstar$ in the figure),
    then one can finish in two more moves by considering the two paths from $\color{green!60!black}\bigstar$
    that cut under the monster $\color{red}M_1$; one of them must work.
    This slightly generalizes the easier case in the solution above
    (which focuses only on the case where $\color{red}M_1$ is in the first row).
    \begin{center}
      \begin{asy}
        usepackage("amssymb");
        unitsize(0.7cm);
        pen gr = grey+linetype("4 2");
        void setup(int n) {
          for (int i=0; i<=n-1; ++i) {
            draw((0,i)--(n,i), gr);
          }
          for (int i=0; i<=n; ++i) {
            draw((i,0)--(i,n-1), gr);
          }
          draw(box((0,-1), (n,0)), black);
          draw(box((0,n-1), (n,n)), black);
          draw(box((0,-1), (n,n)), black);
          label((n/2,n-0.5), "Starting row");
          label((n/2,-0.5), "Goal row");

          path p = (0,n);
          for (int i=0; i<n/2; ++i) {
            p = p--(i,n-1-i)--(i+1,n-1-i);
          }
          for (int i=(n+1)#2; i<n; ++i) {
            p = p--(i,i)--(i+1,i);
          }
          p = p--(n,n)--cycle;
          filldraw(p, opacity(0.15)+yellow, blue+1.8);
        }
        setup(7);
        label((2.5,4.5), "$M_1$", red);
        label((1.5,5.5), "$\bigstar$", deepgreen);
        label((3.5,5.5), "$\bigstar$", deepgreen);
        draw((1.5,5.5)--(1.5,3.5)--(2.3,3.5)--(2.3,0.2), deepgreen+1.2, EndArrow(TeXHead));
        draw((3.5,5.5)--(3.5,3.5)--(2.7,3.5)--(2.7,0.2), deepgreen+1.2, EndArrow(TeXHead));
      \end{asy}
    \end{center}


  \item \textbf{Outside happy triangle:}
    Now suppose the first monster $\color{red}M_1$ is outside the \emph{happy triangle}.
    Of the two shoulders, take the one closer to the center
    (if in the center column, either one works; if only one shoulder, use it).
    If there is a safe path to that shoulder,
    then one can take a staircase pattern towards the center, as shown in the figure.
    In that case, the choice of shoulder and position guarantees the staircase
    reaches the bottom row, so that if no monster is along this path, the algorithm ends.
    Otherwise, if one encounters a second monster along the staircase,
    then one can use the third trial to cut under the monster $\color{red}M_1$.
    \begin{center}
      \begin{asy}
        usepackage("amssymb");
        unitsize(0.7cm);
        pen gr = grey+linetype("4 2");
        void setup(int n) {
          for (int i=0; i<=n-1; ++i) {
            draw((0,i)--(n,i), gr);
          }
          for (int i=0; i<=n; ++i) {
            draw((i,0)--(i,n-1), gr);
          }
          draw(box((0,-1), (n,0)), black);
          draw(box((0,n-1), (n,n)), black);
          draw(box((0,-1), (n,n)), black);
          label((n/2,n-0.5), "Starting row");
          label((n/2,-0.5), "Goal row");

          path p = (0,n);
          for (int i=0; i<n/2; ++i) {
            p = p--(i,n-1-i)--(i+1,n-1-i);
          }
          for (int i=(n+1)#2; i<n; ++i) {
            p = p--(i,i)--(i+1,i);
          }
          p = p--(n,n)--cycle;
          filldraw(p, opacity(0.15)+yellow, blue+1.8);
        }

        setup(9);
        label((1.5,3.5), "$M_1$", red);
        label((2.5,4.5), "$\bigstar$", deepgreen);
        draw((2.5,4.5)--(2.5,3.5)--(3.5,3.5)--(3.5,2.5)--(4.5,2.5)--(4.5,1.5)--(5.5,1.5)--(5.5,0.2),
          deepgreen+1.2, EndArrow(TeXHead));
      \end{asy}
      \qquad
      \begin{asy}
        usepackage("amssymb");
        unitsize(0.7cm);
        pen gr = grey+linetype("4 2");
        void setup(int n) {
          for (int i=0; i<=n-1; ++i) {
            draw((0,i)--(n,i), gr);
          }
          for (int i=0; i<=n; ++i) {
            draw((i,0)--(i,n-1), gr);
          }
          draw(box((0,-1), (n,0)), black);
          draw(box((0,n-1), (n,n)), black);
          draw(box((0,-1), (n,n)), black);
          label((n/2,n-0.5), "Starting row");
          label((n/2,-0.5), "Goal row");

          path p = (0,n);
          for (int i=0; i<n/2; ++i) {
            p = p--(i,n-1-i)--(i+1,n-1-i);
          }
          for (int i=(n+1)#2; i<n; ++i) {
            p = p--(i,i)--(i+1,i);
          }
          p = p--(n,n)--cycle;
          filldraw(p, opacity(0.15)+yellow, blue+1.8);
        }

        setup(9);
        label((1.5,3.5), "$M_1$", red);
        label((2.5,4.5), "$\bigstar$", deepgreen);
        draw((2.5,4.5)--(2.5,3.5)--(3.5,3.5)--(3.5,2.5)--(4.5,2.5)--(4.5,1.5)--(5.5,1.5)--(5.5,0.2),
          deepgreen+dashed, EndArrow(TeXHead));
        label((5.5,1.5), "$M_2$", red);
        draw((2.5,4.5)--(2.5,3.5)--(3.5,3.5)--(3.5,2.5)--(4.5,2.5)--(4.5,1.5)--(1.5,1.5)--(1.5,0.2),
          purple+1.5, EndArrow(TeXHead));
      \end{asy}
    \end{center}
\end{itemize}

We now prove the following proposition:
in any valid strategy for Turbo,
in the case where Turbo first encounters a monster upon leaving the happy triangle,
the second path \emph{must} outline the same staircase shape.

The monsters pre-commit to choosing their pattern to be
\emph{either} a NW-SE diagonal or NE-SW diagonal, with a single one-column gap;
see figure below for an example.
Note that this forces any valid path for Turbo to pass through the particular gap.

\begin{center}
  \begin{asy}
    usepackage("amssymb");
    unitsize(0.7cm);
    pen gr = grey+linetype("4 2");
    void setup(int n) {
      for (int i=0; i<=n-1; ++i) {
        draw((0,i)--(n,i), gr);
      }
      for (int i=0; i<=n; ++i) {
        draw((i,0)--(i,n-1), gr);
      }
      draw(box((0,-1), (n,0)), black);
      draw(box((0,n-1), (n,n)), black);
      draw(box((0,-1), (n,n)), black);
      label((n/2,n-0.5), "Starting row");
      label((n/2,-0.5), "Goal row");

      path p = (0,n);
      for (int i=0; i<n/2; ++i) {
        p = p--(i,n-1-i)--(i+1,n-1-i);
      }
      for (int i=(n+1)#2; i<n; ++i) {
        p = p--(i,i)--(i+1,i);
      }
      p = p--(n,n)--cycle;
      filldraw(p, opacity(0.15)+yellow, blue+1.8);
    }
    setup(11);

    for (int i=0; i<7; ++i) {
      label((i+0.5,9.5-i), "$M$", red);
    }
    for (int i=7; i<10; ++i) {
      label((i+1.5,9.5-i), "$M$", red);
    }
  \end{asy}
\end{center}

We may assume without loss of generality that Turbo first encounters a monster $M_1$
when Turbo first leaves the happy triangle, and that this forces an NW-SE configuration.

\begin{center}
  \begin{asy}
    usepackage("amssymb");
    unitsize(0.7cm);
    pen gr = grey+linetype("4 2");
    void setup(int n) {
      for (int i=0; i<=n-1; ++i) {
        draw((0,i)--(n,i), gr);
      }
      for (int i=0; i<=n; ++i) {
        draw((i,0)--(i,n-1), gr);
      }
      draw(box((0,-1), (n,0)), black);
      draw(box((0,n-1), (n,n)), black);
      draw(box((0,-1), (n,n)), black);
      label((n/2,n-0.5), "Starting row");
      label((n/2,-0.5), "Goal row");

      path p = (0,n);
      for (int i=0; i<n/2; ++i) {
        p = p--(i,n-1-i)--(i+1,n-1-i);
      }
      for (int i=(n+1)#2; i<n; ++i) {
        p = p--(i,i)--(i+1,i);
      }
      p = p--(n,n)--cycle;
      filldraw(p, opacity(0.15)+yellow, blue+1.8);
    }
    setup(13);
    label((0.5,11.5), "($M$)", red+fontsize(9pt));
    label((1.5,10.5), "($M$)", red+fontsize(9pt));
    label((2.5,9.5), "($M$)", red+fontsize(9pt));
    label((3.5,8.5), "$M_1$", red);
    label((4.5,7.5), "X", brown);
    label((11.5,1.5), "$M_2$", red);
    for (int i=1; i<=6; ++i) {
      label((11.5-i,1.5+i), "$\clubsuit$", deepgreen);
    }
  \end{asy}
\end{center}

Then the following is true:
\begin{proposition*}
  The strategy of Turbo on the second path \emph{must}
  visit every cell in ``slightly raised diagonal'' marked with
  $\color{green!60!black}\clubsuit$ in the figure above
  in order from top to bottom, until it encounters a second Monster $M_2$
  (or reaches the bottom row and wins anyway).
  It's both okay and irrelevant if Turbo visits other cells above this diagonal,
  but the marked cells must be visited from top to bottom in that order.
\end{proposition*}
\begin{proof}
  If Turbo tries to sidestep by visiting the cell southeast of $M_1$
  (marked {\color{brown}X} in the Figure),
  then Turbo clearly cannot finish after this (for $s$ large enough).
  Meanwhile, suppose Turbo tries to ``skip'' one of the $\color{green!60!black}\clubsuit$,
  say in column $C$, then the gap could equally well be in the column to the left of $C$.
  This proves the proposition.
\end{proof}

\begin{remark*}
  [Memories of safe cells are important, not just monster cells]
  Here is one additional observation that one can deduce from this.
  We say a set $\mathcal S$ of revealed monsters is called \emph{obviously winnable} if,
  based on only the positions of the monsters
  (and not the moves or algorithm that were used to obtain them),
  one can identify a guaranteed winning path for Turbo using only $\mathcal S$.
  For example, two monsters in adjacent columns which are not diagonally
  adjacent is obviously winnable.

  Then no strategy can guarantee obtaining an obviously winnable set in $2$ moves
  (or even $k$ moves for any constant $k$, if $s$ is large enough in terms of $k$).
  So any valid strategy must \emph{also} use the \emph{memory} of identified safe cells
  that do not follow just from the revealed monster positions.
\end{remark*}
\pagebreak