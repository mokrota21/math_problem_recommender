\textsl{Available online at \url{https://aops.com/community/p31205957}.}
\begin{mdframed}[style=mdpurplebox,frametitle={Problem statement}]
For which pairs of positive integers $(a,b)$ is the sequence
\[ \gcd(a^n+b, b^n+a) \qquad n = 1, 2, \dotsc \]
eventually constant?
\end{mdframed}
The answer is $(a,b)=(1,1)$ only, which obviously works since the sequence is always $2$.

Conversely, assume the sequence
\[ x_n \coloneqq \gcd(a^n+b, b^n+a) \] is eventually constant.
The main crux of the other direction is to consider
\[ M \coloneqq ab+1. \]
\begin{remark*}
  [Motivation]
  The reason to consider the number is the same technique used in IMO 2005/4,
  namely the idea to consider ``$n = -1$''.
  The point is that the two rational numbers
  \[ \frac 1a + b = \frac{ab+1}{b}, \qquad \frac 1b + a = \frac{ab+1}{a} \]
  have a large common factor: we could write ``$x_{-1} = ab + 1$'', loosely speaking.

  Now, the sequence is really only defined for $n \ge 1$,
  so one should instead take $n \equiv -1 \pmod{\varphi(M)}$
  --- and this is exactly what we do.
\end{remark*}
Obviously $\gcd(a,M) = \gcd(b,M) = 1$.
Let $n$ be a sufficiently large multiple of $\varphi(M)$
so that \[ x_{n-1} = x_n = x_{n+1} = \dotsb. \]
We consider the first three terms;
the first one is the ``key'' one that gets the bulk of the work,
and the rest is bookkeeping and extraction.
\begin{itemize}
  \ii Consider $x_{n-1}$.
  Note that
  \[ a (a^{n-1} + b) = a^n + ab \equiv 1 + (-1) \equiv 0 \pmod M \]
  and similarly $b (b^n + a) \equiv 0 \pmod M$.
  Hence $M \mid x_{n-1}$.

  \ii Consider $x_n$, which is now known to be divisible by $M$. Note that
  \begin{align*}
    0 &\equiv a^n + b \equiv 1 + b \pmod M \\
    0 &\equiv b^n + a \equiv 1 + a \pmod M.
  \end{align*}
  So $a \equiv b \equiv -1 \pmod M$.

  \ii Consider $x_{n+1}$, which is now known to be divisible by $M$. Note that
  \[ 0 \equiv a^{n+1} + b \equiv b^{n+1} + a \equiv a + b \pmod M. \]
  We knew $a \equiv b \equiv -1 \pmod M$,
  hence this means $0 \equiv 2 \pmod M$, so $M = 2$.
\end{itemize}
From $M = 2$ we then conclude $a = b = 1$, as desired.

\begin{remark*}
  [No alternate solutions known]
  At the time nobody seems to know any solution not depending critically
  on $M = ab+1$ (or prime numbers dividing $M$, etc.).
  They vary in execution once some term of the form $x_{k\varphi(n)-1}$ is taken,
  but avoiding the key idea altogether does not currently seem possible.

  A good example to consider for ruling out candidate ideas is $(a,b) = (18,9)$.
\end{remark*}
\pagebreak