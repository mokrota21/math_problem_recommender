\textsl{Available online at \url{https://aops.com/community/p893741}.}
\begin{mdframed}[style=mdpurplebox,frametitle={Problem statement}]
Real numbers $a_1$, $a_2$, \dots, $a_n$ are fixed.
For each $1 \le i \le n$ we let
$d_i = \max\{a_j : 1 \le j \le i\} - \min\{a_j : i \le j \le n\}$
and let $d = \max \{d_i : 1 \le i \le n\}$.

\begin{enumerate}[(a)]
  \ii Prove that for any real numbers $x_1 \le \dots \le x_n$ we have
  \[
    \max \left\{ \left\lvert x_i - a_i \right\rvert :
      1 \le i \le n \right\}
      \ge \half d.
  \]
  \ii Moreover, show that there exists some
  choice of $x_1 \le \dots \le x_n$ which achieves equality.
\end{enumerate}
\end{mdframed}
Note that we can dispense of $d_i$ immediately
by realizing that the definition of $d$ just says
\[ d = \max_{1 \le i \le j \le n} \left( a_i - a_j \right). \]

If $a_1 \le \dots \le a_n$ are already nondecreasing
then $d = 0$ and there is nothing to prove
(for the equality case, just let $x_i = a_i$),
so we will no longer consider this case.

Otherwise, consider any indices $i < j$ with $a_i > a_j$.
We first prove (a) by applying the following claim
with $p = a_i$ and $q = a_j$:
\begin{claim*}
  For any $p \le q$, we have
  either $|p - a_i| \ge \half(a_i-a_j)$
  or $|q - a_j| \ge \half(a_i-a_j)$.
\end{claim*}
\begin{proof}
  Assume for contradiction both are false.
  Then $p > a_i - \half(a_i-a_j)
  = a_j + \half(a_i-a_j) > q$, contradiction.
\end{proof}

As for (b), we let $i < j$ be any indices for which
$a_i - a_j = d > 0$ achieves the maximal difference.
We then define $x_\bullet$ in three steps:
\begin{itemize}
  \ii We set $x_k = \frac{a_i + a_j}{2}$ for $k = i, \dots, j$.
  \ii We recursively set $x_{k} = \max(x_{k-1}, a_k)$
  for $k = j+1, j+2, \dots$.
  \ii We recursively set $x_{k} = \min(x_{k+1}, a_k)$
  for $k = i-1, i-2, \dots$.
\end{itemize}
By definition, these $x_\bullet$ are weakly increasing.
To prove this satisfies (b) we only need to check that
\[ \left\lvert x_k - a_k \right\rvert \le \frac{a_i-a_j}{2} \qquad
  (\star) \]
for any index $k$ (as equality holds for $k = i$ or $k = j$).

We note $(\star)$ holds for $i < k < j$ by construction.
For $k > j$, note that $x_k \in \{a_j, a_{j+1}, \dots, a_k\}$
by construction, so $(\star)$ follows from our choice of $i$ and $j$
giving the largest possible difference; the case $k < i$ is similar.
\pagebreak