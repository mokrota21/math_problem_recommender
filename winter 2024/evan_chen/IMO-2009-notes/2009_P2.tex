\textsl{Available online at \url{https://aops.com/community/p1561572}.}
\begin{mdframed}[style=mdpurplebox,frametitle={Problem statement}]
Let $ABC$ be a triangle with circumcenter $O$.
The points $P$ and $Q$ are interior points of the sides $CA$ and $AB$ respectively.
Let $K$, $L$, $M$ be the midpoints of $\ol{BP}$, $\ol{CQ}$, $\ol{PQ}$,
respectively, and let $\Gamma$ be the circumcircle of $\triangle KLM$.
Suppose that $\ol{PQ}$ is tangent to $\Gamma$. Prove that $OP = OQ$.
\end{mdframed}
By power of a point, we have $-AQ \cdot QB = OQ^2 - R^2$
and $-AP \cdot PC = OP^2 - R^2$.
Therefore, it suffices to show $AQ \cdot QB = AP \cdot PC$.

\begin{center}
\begin{asy}
pair A = dir(70);
pair B = dir(210);
pair C = dir(330);

pair P = 0.4*A+0.6*C;
pair Q = 0.25*B+0.75*A;

filldraw(A--B--C--cycle, opacity(0.2)+lightred, red);
pair M = midpoint(P--Q);
pair K = midpoint(P--B);
pair L = midpoint(Q--C);

filldraw(circumcircle(M, K, L), opacity(0.3)+lightcyan, blue);
draw(P--Q, heavygreen+1);
draw(B--P, heavygreen);
draw(C--Q, heavygreen);

filldraw(K--M--L--cycle, opacity(0.4)+cyan, heavycyan);

dot("$A$", A, dir(A));
dot("$B$", B, dir(B));
dot("$C$", C, dir(C));
dot("$P$", P, dir(P));
dot("$Q$", Q, dir(120));
dot("$M$", M, dir(M));
dot("$K$", K, dir(K));
dot("$L$", L, dir(20));

/* TSQ Source:

A = dir 70
B = dir 210
C = dir 330

P = 0.4*A+0.6*C
Q = 0.25*B+0.75*A R120

A--B--C--cycle 0.2 lightred / red
M = midpoint P--Q
K = midpoint P--B
L = midpoint Q--C R20

circumcircle M K L 0.3 lightcyan / blue
P--Q heavygreen+1
B--P heavygreen
C--Q heavygreen

K--M--L--cycle 0.4 cyan / heavycyan

*/
\end{asy}
\end{center}

As $\ol{ML} \parallel \ol{AC}$ and $\ol{MK} \parallel \ol{AB}$ we have that
\begin{align*}
  \dang APQ &= \dang LMP = \dang LKM \\
  \dang PQA &= \dang KMQ = \dang MLK
\end{align*}
and consequently we have the (opposite orientation) similarity
\[ \triangle APQ \overset{-}{\sim} \triangle MKL. \]
Therefore
\[ \frac{AQ}{AP} = \frac{ML}{MK} = \frac{2ML}{2MK} = \frac{PC}{QB} \]
id est $AQ \cdot QB = AP \cdot PC$, which is what we wanted to prove.
\pagebreak