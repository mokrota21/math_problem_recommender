\textsl{Available online at \url{https://aops.com/community/p5083464}.}
\begin{mdframed}[style=mdpurplebox,frametitle={Problem statement}]
Triangle $ABC$ has circumcircle $\Omega$ and circumcenter $O$.
A circle $\Gamma$ with center $A$
intersects the segment $BC$ at points $D$ and $E$,
such that $B$, $D$, $E$, and $C$ are all different
and lie on line $BC$ in this order.
Let $F$ and $G$ be the points of intersection of $\Gamma$ and $\Omega$,
such that $A$, $F$, $B$, $C$, and $G$ lie on $\Omega$ in this order.
Let $K = (BDF) \cap \ol{AB} \neq B$
and $L = (CGE) \cap \ol{AC} \neq C$
and assume these points do not lie on line $FG$.
Define $X = \ol{FK} \cap \ol{GL}$.
Prove that $X$ lies on the line $AO$.
\end{mdframed}
Since $\ol{AO} \perp \ol{FG}$ for obvious reasons,
we will only need to show that $XF = XG$,
or that $\dang KFG = \dang LGF$.

Let line $FG$ meet $(BDF)$ and $(CGE)$
again at $F_2$ and $G_2$.
\begin{center}
\begin{asy}
size(10cm);
pair A = dir(105);
pair B = dir(200);
pair C = dir(340);
real r = 1.337;

pair F = IP(CR(A, r), unitcircle);
pair G = OP(CR(A, r), unitcircle);
pair D = IP(CR(A, r), B--C);
pair E = OP(CR(A, r), B--C);

pair K = IP(circumcircle(B, D, F), A--B);
pair L = IP(circumcircle(C, E, G), A--C);
draw(unitcircle, orange);
draw(arc(A,r,160,380), orange);
filldraw(A--B--C--cycle, opacity(0.1)+lightred, red);
pair X = extension(F, K, G, L);
draw(F--G, red);
draw(F--X--G, lightblue);

pair F_2 = IP(circumcircle(B, F, D), F--G);
pair G_2 = IP(F--G, circumcircle(C, E, G));
filldraw(circumcircle(B, F, D), opacity(0.1)+yellow, lightblue);
filldraw(circumcircle(C, E, G), opacity(0.1)+yellow, lightblue);

draw(B--F_2--D--F--cycle, lightcyan);
draw(C--G_2--E--G--cycle, lightcyan);

dot("$A$", A, dir(A));
dot("$B$", B, dir(B));
dot("$C$", C, dir(C));
dot("$F$", F, dir(F));
dot("$G$", G, dir(G));
dot("$D$", D, dir(D));
dot("$E$", E, dir(E));
dot("$K$", K, dir(100));
dot("$L$", L, dir(L));
dot("$X$", X, dir(X));
dot("$F_2$", F_2, dir(70));
dot("$G_2$", G_2, dir(G_2));

/* TSQ Source:

A = dir 105
B = dir 200
C = dir 340
! real r = 1.337;

F = IP CR A r unitcircle
G = OP CR A r unitcircle
D = IP CR A r B--C
E = OP CR A r B--C

K = IP circumcircle B D F A--B R100
L = IP circumcircle C E G A--C
unitcircle orange
!draw(arc(A,r,160,380), orange);
A--B--C--cycle 0.1 lightred / red
X = extension F K G L
F--G red
F--X--G lightblue

F_2 = IP circumcircle B F D F--G R70
G_2 = IP F--G circumcircle C E G
circumcircle B F D 0.1 yellow / lightblue
circumcircle C E G 0.1 yellow / lightblue

B--F_2--D--F--cycle lightcyan
C--G_2--E--G--cycle lightcyan

*/
\end{asy}
\end{center}

\begin{claim*}
  Quadrilaterals $FBDF_2$ and $G_2ECG$ are similar,
  actually homothetic through $\ol{FG} \cap \ol{BC}$.
\end{claim*}
\begin{proof}
  This is essentially a repeated application
  of being ``anti-parallel'' through $\angle(FG, BC)$.
  Note the four angle relations
  \begin{align*}
    \dang(FD, FG) = \dang(BC,GE) = \dang(G_2C,FG)
      &\implies \ol{FD} \parallel \ol{G_2C} \\
    \dang(F_2B, FG) = \dang(BC,FD) = \dang(GE,FG)
      &\implies \ol{F_2B} \parallel \ol{GE} \\
    \dang(FB, FG)  = \dang(BC,GC) = \dang(G_2E,FG)
      &\implies \ol{FB} \parallel \ol{G_2E} \\
    \dang(F_2D, FG)  = \dang(BC,FB) = \dang(GC,FG)
      &\implies \ol{F_2D} \parallel \ol{GC}.
  \end{align*}
  This gives the desired homotheties.
\end{proof}
To finish the angle chase,
\begin{align*}
  \dang GFK = \dang F_2BK &= \dang F_2BF - \dang ABF
  = \dang F_2DF - \dang ABF \\
  &= \dang F_2DF - \dang GCA
  = \dang GCG_2 - \dang GCA \\
  &=  \dang LCG_2 = \dang LGF
\end{align*}
as needed.
(Here $\dang ABF = \dang GCA$ since $AF = AG$.)
\pagebreak