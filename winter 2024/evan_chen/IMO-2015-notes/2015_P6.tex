
\textsl{Available online at \url{https://aops.com/community/p5083494}.}
\begin{mdframed}[style=mdpurplebox,frametitle={Problem statement}]
The sequence $a_1,a_2,\dots$ of integers satisfies the conditions:
\begin{enumerate}[(i)]
  \ii $1\le a_j\le2015$ for all $j\ge1$,
  \ii $k+a_k\neq \ell+a_\ell$ for all $1\le k<\ell$.
\end{enumerate}
Prove that there exist two positive integers $b$ and $N$ for which
\[ \left\lvert\sum_{j=m+1}^n (a_j-b) \right\rvert \le 1007^2 \]
for all integers $m$ and $n$ such that $n > m\ge N$.
\end{mdframed}
We give two equivalent solutions with different presentations,
one with ``arrows'' and the other by ``juggling''.

\paragraph{First solution (arrows).}
Consider the map
\[ f \colon k \mapsto k + a_k. \]
This map is injective, so if we draw all arrows of the form $k \mapsto f(k)$
we get a partition of $\NN$ into one or more ascending chains
(which skip by at most $2015$).

There are at most $2015$ such chains,
since among any $2015$ consecutive points in $\NN$
every chain must have an element.

We claim we may take $b$ to be the number of such chains,
and $N$ to be the largest of the start-points of all the chains.

% TODO picture here would be nice
Consider an interval $I = [m+1, n]$.
We have that
\[ \sum_{m<j\le n} a_j = \sum_{\text{chain } c} \left[ \min \left\{ x > n, x \in c \right\}
  - \min \left\{ x > m, x \in c \right\} \right]. \]
Thus the upper bound is proved by the calculation
\begin{align*}
  \sum_{m<j\le n} (a_j-b)
  &= \sum_{\text{chain } c} \left[ (\min \left\{ x > n, x \in c \right\} - n)
    - (\min \left\{ x > m, x \in c \right\} - m) \right] \\
  &= \sum_{\text{chain } c} \left[ (\min \left\{ x > n, x \in c \right\} - n) \right]
    - \sum_{\text{chain } c} \left[
    \min \left\{ x > m, x \in c \right\} - m \right] \\
  &\le (1+2015+2014+\dots+(2015-(b-2)))-(1+2+\dots+b) = (b-1)(2015-b)
\end{align*}
from above (noting that $n+1$ has to belong to some chain).
The lower bound is similar.

\paragraph{Second solution (juggling).}
This solution is essentially the same, but phrased as a juggling problem.
Here is a solution in this interpretation:
we will consider several balls thrown in the air,
which may be at heights $0$, $1$, $2$, \dots, $2014$.
The process is as follows:
\begin{itemize}
\ii Initially, at time $t = 0$, there are no balls in the air.
\ii Then at each integer time $t$ thereafter,
if there is a ball at height $0$, it is caught;
otherwise a ball is added to the juggler's hand.
This ball (either caught or added) is then thrown to a height of $a_t$.
\ii Immediately afterwards, all balls have their height decreased by one.
\end{itemize}
The condition $a_k + k \neq \ell + a_\ell$ thus ensures that
no two balls are ever at the same height.
In particular, there will never be more than $2016$ balls,
since there are only $2015$ possible heights.

We claim we may set.
\begin{align*}
  b &= \text{number of balls in entire process} \\
  N &= \text{last moment in time at which a ball is added}.
\end{align*}
Indeed, the key fact is that if we let $S_t$ denote
the sum of the height of all the balls just after time $t+\half$, then
\[ S_{t+1} - S_t = a_{t+1} - b \]
After all, at each time step $t$, the caught ball is thrown to height $a_t$,
and then all balls have their height decreased by $1$,
from which the conclusion follows.
Hence the quantity in the problem is exactly equal to
\[ \left\lvert\sum_{j=m+1}^n (a_j-b) \right\rvert
  = \left\lvert S_m - S_n \right\rvert. \]
For a fixed $b$, we easily have the inequalities
$0 + 1 + \dots + (b-1) \le S_t \le 2014 + 2013 + \dots + (2015-b)$.
Hence $|S_m - S_n| \le (b-1)(2015-b) \le 1007^2$ as desired.
\pagebreak


\end{document}
