\textsl{Available online at \url{https://aops.com/community/p281571}.}
\begin{mdframed}[style=mdpurplebox,frametitle={Problem statement}]
Six points are chosen on the sides of an equilateral triangle $ABC$:
$A_1$, $A_2$ on $BC$, $B_1$, $B_2$ on $CA$ and $C_1$, $C_2$ on $AB$,
such that they are the vertices of a
convex hexagon $A_1A_2B_1B_2C_1C_2$ with equal side lengths.
Prove that the lines $A_1B_2$, $B_1C_2$ and $C_1A_2$ are concurrent.
\end{mdframed}
The six sides of the hexagon, when oriented, comprise
six vectors with vanishing sum.
However note that \[ \overrightarrow{A_1A_2}
+ \overrightarrow{B_1B_2}
+ \overrightarrow{C_1C_2} = 0. \]
Thus
\[ \overrightarrow{A_2B_1} + \overrightarrow{B_2C_1} +
\overrightarrow{C_2A_1} = 0 \]
and since three unit vectors with vanishing sum
must be rotations of each other by $120\dg$,
it follows they must also form an equilateral triangle.

\begin{center}
\begin{asy}
pair A_1 = origin;
pair A_2 = A_1+dir(0);
pair B_1 = A_2+dir(37);
pair B_2 = B_1+dir(120);
pair C_1 = B_2+dir(157);
pair C_2 = C_1+dir(240);

pair A = extension(B_1, B_2, C_1, C_2);
pair B = extension(C_1, C_2, A_1, A_2);
pair C = extension(A_1, A_2, B_1, B_2);

filldraw(A--B--C--cycle, opacity(0.1)+lightcyan, lightblue);

draw(A_1--A_2, red+1, EndArrow(TeXHead), Margins);
draw(B_1--B_2, red+1, EndArrow(TeXHead), Margins);
draw(C_1--C_2, red+1, EndArrow(TeXHead), Margins);
draw(A_2--B_1, blue+1, EndArrow(TeXHead), Margins);
draw(B_2--C_1, blue+1, EndArrow(TeXHead), Margins);
draw(C_2--A_1, blue+1, EndArrow(TeXHead), Margins);

filldraw(A_1--B_1--C_1--cycle, opacity(0.1)+lightgreen, heavygreen);
draw(A_1--B_2, dotted+heavycyan);
draw(B_1--C_2, dotted+heavycyan);
draw(C_1--A_2, dotted+heavycyan);

dot("$A_1$", A_1, dir(270));
dot("$A_2$", A_2, dir(270));
dot("$B_1$", B_1, dir(30));
dot("$B_2$", B_2, dir(30));
dot("$C_1$", C_1, dir(150));
dot("$C_2$", C_2, dir(150));
dot("$A$", A, dir(90));
dot("$B$", B, dir(210));
dot("$C$", C, dir(330));

/* TSQ Source:

A_1 = origin R270
A_2 = A_1+dir(0) R270
B_1 = A_2+dir(37) R30
B_2 = B_1+dir(120) R30
C_1 = B_2+dir(157) R150
C_2 = C_1+dir(240) R150

A = extension B_1 B_2 C_1 C_2 R90
B = extension C_1 C_2 A_1 A_2 R210
C = extension A_1 A_2 B_1 B_2 R330

A--B--C--cycle 0.1 lightcyan / lightblue

!draw(A_1--A_2, red+1, EndArrow(TeXHead), Margins);
!draw(B_1--B_2, red+1, EndArrow(TeXHead), Margins);
!draw(C_1--C_2, red+1, EndArrow(TeXHead), Margins);
!draw(A_2--B_1, blue+1, EndArrow(TeXHead), Margins);
!draw(B_2--C_1, blue+1, EndArrow(TeXHead), Margins);
!draw(C_2--A_1, blue+1, EndArrow(TeXHead), Margins);

A_1--B_1--C_1--cycle 0.1 lightgreen / heavygreen
A_1--B_2 dotted heavycyan
B_1--C_2 dotted heavycyan
C_1--A_2 dotted heavycyan

*/
\end{asy}
\end{center}

Consequently, triangles $A_1A_2B_1$, $B_1B_2C_1$, $C_1C_2A_1$
are congruent, as $\angle A_2 = \angle B_2 = \angle C_2$.
So triangle $A_1 B_1 C_1$ is equilateral and the
diagonals are concurrent at the center.
\pagebreak