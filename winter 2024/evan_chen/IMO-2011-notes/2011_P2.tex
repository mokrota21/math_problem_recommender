\textsl{Available online at \url{https://aops.com/community/p2363537}.}
\begin{mdframed}[style=mdpurplebox,frametitle={Problem statement}]
Let $\mathcal{S}$ be a finite set of at least two points in the plane.
Assume that no three points of $\mathcal S$ are collinear.
A \emph{windmill} is a process that starts with a
line $\ell$ going through a single point $P \in \mathcal S$.
The line rotates clockwise about the \emph{pivot} $P$ until the first time
that the line meets some other point belonging to $\mathcal S$.
This point, $Q$, takes over as the new pivot,
and the line now rotates clockwise about $Q$,
until it next meets a point of $\mathcal S$.
This process continues indefinitely.

Show that we can choose a point $P$ in $\mathcal S$ and
a line $\ell$ going through $P$ such that the resulting windmill
uses each point of $\mathcal S$ as a pivot infinitely many times.
\end{mdframed}
Orient $\ell$ in some direction,
and color the plane such that its left half is red
and right half is blue.
The critical observation is that:
\begin{claim*}
  The number of points on the red side of $\ell$ does not change,
  nor does the number of points on the blue side
  (except at a moment when $\ell$ contains two points).
\end{claim*}

Thus, if $|\mathcal S| = n+1$,
it suffices to pick the initial configuration
so that there are $\left\lfloor n/2 \right\rfloor$
red and $\left\lceil n/2 \right\rceil$ blue points.
Then when the line $\ell$ does a full $180\dg$ rotation,
the red and blue sides ``switch'',
so the windmill has passed through every point.

(See official shortlist for verbose write-up;
this is deliberately short to make a point.)
\pagebreak