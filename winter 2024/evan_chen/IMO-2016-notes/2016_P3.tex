\textsl{Available online at \url{https://aops.com/community/p6637660}.}
\begin{mdframed}[style=mdpurplebox,frametitle={Problem statement}]
Let $P=A_1A_2\dots A_k$ be a convex polygon in the plane.
The vertices $A_1$, $A_2$, \dots, $A_k$ have integral coordinates
and lie on a circle. Let $S$ be the area of $P$.
An odd positive integer $n$ is given such that
the squares of the side lengths of $P$ are integers divisible by $n$.
Prove that $2S$ is an integer divisible by $n$.
\end{mdframed}
Solution by Jeck Lim:
We will prove the result just for $n = p^e$
where $p$ is an odd prime and $e \ge 1$.
The case $k=3$ is resolved by Heron's formula directly:
we have $S = \frac14\sqrt{2(a^2b^2 + b^2c^2 + c^2a^2) - a^4-b^4-c^4}$,
so if $p^e \mid \gcd(a^2,b^2,c^2)$ then $p^{2e} \mid S^2$.

Now we show we can pick a diagonal and induct down on $k$ by using inversion.

Let the polygon be $A_1 A_2 \dots A_{k+1}$
and suppose for contradiction that all sides are divisible by $p^e$
but no diagonals are.
Let $O = A_{k+1}$ for notational convenience.
By applying inversion around $O$ with radius $1$,
we get the ``generalized Ptolemy theorem''
\[
  \frac{A_1A_2}{OA_1 \cdot OA_2}
  + \frac{A_2A_3}{OA_2 \cdot OA_3}
  + \dots
  + \frac{A_{k-1} A_k}{OA_{k-1} \cdot OA_k}
  = \frac{A_1 A_k}{OA_1 \cdot OA_k}
\]
or, making use of square roots,
\[
  \sqrt{\frac{A_1A_2^2}{OA_1^2 \cdot OA_2^2}}
  + \sqrt{\frac{A_2A_3^2}{OA_2^2 \cdot OA_3^2}}
  + \dots
  + \sqrt{\frac{A_{k-1} A_k^2}{OA_{k-1}^2 \cdot OA_k^2}}
  = \sqrt{\frac{A_1 A_k^2}{OA_1^2 \cdot OA_k^2}}
\]
Suppose $\nu_p$ of all diagonals is strictly less than $e$.
Then the relation becomes
\[ \sqrt{q_1} + \dots + \sqrt{q_{k-1}} = \sqrt q \]
where $q_i$ are positive rational numbers.
Since there are no nontrivial relations between square roots
(see \href{https://qchu.wordpress.com/2009/07/02/square-roots-have-no-unexpected-linear-relationships/}{this link})
there is a positive rational number $b$
such that $r_i = \sqrt{q_i/b}$ and $r = \sqrt{q/b}$
are all rational numbers.
Then
\[ \sum r_i = r. \]
However, the condition implies that $\nu_p(q_i^2) > \nu_p(q^2)$ for all $i$
(check this for $i=1$, $i=k-1$ and $2 \le i \le k-2$),
and hence $\nu_p(r_i) > \nu_p(r)$.
This is absurd.

\begin{remark*}
  I think you basically have to use some Ptolemy-like geometric property,
  and also all correct solutions I know of for $n = p^e$
  depend on finding a diagonal and inducting down.
  (Actually, the case $k=4$ is pretty motivating;
  Ptolemy implies one can cut in two.)
\end{remark*}
\pagebreak

\section{Solutions to Day 2}