\textsl{Available online at \url{https://aops.com/community/p118710}.}
\begin{mdframed}[style=mdpurplebox,frametitle={Problem statement}]
Let $n$ be a positive integer.
Let $T$ be the set of points $(x,y)$ in the plane
where $x$ and $y$ are non-negative integers with $x+y<n$.
Each point of $T$ is coloured red or blue,
subject to the following condition:
if a point $(x,y)$ is red,
then so are all points $(x',y')$ of $T$
with $x'\leq x$ and $y'\leq y$.
Let $A$ be the number of ways to choose $n$ blue points
with distinct $x$-coordinates,
and let $B$ be the number of ways to choose $n$ blue
points with distinct $y$-coordinates.
Prove that $A=B$.
\end{mdframed}
Let $a_x$ denote the number of blue points
with a given $x$-coordinate.
Define $b_y$ to be the number of blue points
with a given $y$-coordinate.

We actually claim that
\begin{claim*}
The multisets $\mathcal A \coloneqq \{ a_x \mid x \}$
and $\mathcal B \coloneqq \{ b_y \mid y \}$ are equal.
\end{claim*}
\begin{proof}
By induction on the number of red points.
If there are no red points at all,
then $\mathcal A = \mathcal B = \{1, \dots, n\}$.
\begin{center}
\begin{asy}
size(5cm);
draw( (1,5)--(1,2)--(4,2), lightgreen );
for (int i=0; i<=6; ++i) {
for (int j=0; j<=6-i; ++j) {
dot( (i,j), blue+1.5 );
}
}
dot( (0,0), red+3 );
dot( (1,0), red+3 );
dot( (2,0), red+3 );
dot( (3,0), red+3 );
dot( (0,1), red+3 );
dot( (1,1), red+3 );
dot( (0,2), red+3 );
dot( (1,2), red+4 );
dot( (0,3), red+3 );
dot( (0,4), red+3 );
draw(circle( (1,2), 0.2 ), red );
\end{asy}
\end{center}
The proof consists of two main steps.
First, suppose we color a single point $P = (x,y)$
from blue to red (while preserving the condition).
Before the coloring, we have $a_x = b_y = n-(x+y)$;
afterwards $a_x = b_y = n-(x+y)-1$
and no other numbers change, as desired.

We also must show that this operation
(repeatedly adding a single point $P$) reaches all
possible shapes of red points.
This is well-known as the red points form a Young tableaux;
for example, one way is to add all the points with $x=0$
first one by one, then all the points with $x=1$, and so on.
So the induction implies the result.
\end{proof}
Finally, \[ A = \prod_{x=0}^{n-1} a_x = \prod_{y=0}^{n-1} b_y = B. \]
\pagebreak