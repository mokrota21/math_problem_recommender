\textsl{Available online at \url{https://aops.com/community/p99450}.}
\begin{mdframed}[style=mdpurplebox,frametitle={Problem statement}]
Define a ``hook'' to be a figure made up of six unit squares
as shown below in the picture,
or any of the figures obtained by applying rotations
and reflections to this figure.
\begin{center}
\begin{asy}
unitsize(0.5 cm);
draw(unitsquare);
draw(shift(0,1)*unitsquare);
draw(shift(0,2)*unitsquare);
draw(shift(1,2)*unitsquare);
draw(shift(2,1)*unitsquare);
draw(shift(2,2)*unitsquare);
\end{asy}
\end{center}
Which $m \times n$ rectangles can be tiled by hooks?
\end{mdframed}
The answer is that one requires:
\begin{itemize}
  \ii $\{1,2,5\} \notin \{m,n\}$,
  \ii $3 \mid m$ or $3 \mid n$,
  \ii $4 \mid m$ or $4 \mid n$.
\end{itemize}

First, we check all of these work, in fact we claim:
\begin{claim*}
  Any rectangle satisfying these conditions
  can be tiled by $3 \times 4$ rectangles (and hence by hooks).
\end{claim*}
\begin{proof}
  If $3 \mid m$ and $4 \mid n$, this is clear.
  Else suppose $12 \mid m$ but $3 \nmid n$, $4 \nmid n$.
  Then $n \ge 7$, so it can be written in the form
  $3a+4b$ for nonnegative integers $a$ and $b$, which permits a tiling.
\end{proof}

We now prove these conditions are necessary.
It is not hard to see that $m,n \notin \{1,2,5\}$ is necessary.

We thus turn our attention to divisibility conditions.
Each hook has a \emph{hole}, and if we associate each hook with
the one that fills its hole, we get a bijective pairing of hooks.
Thus the number of cells is divisible by $12$,
and the cells come grouped into two possible shapes,
which we will call \textbf{tiles} shown below,
(rotations and reflections permitted).
\begin{center}
\begin{asy}
unitsize(0.5 cm);
filldraw(unitsquare, white, black);
filldraw(shift(0,1)*unitsquare, white, black);
filldraw(shift(0,2)*unitsquare, white, black);
filldraw(shift(1,2)*unitsquare, white, black);
filldraw(shift(2,1)*unitsquare, white, black);
filldraw(shift(2,2)*unitsquare, white, black);

filldraw(shift(1,0)*unitsquare, grey, black);
filldraw(shift(1,1)*unitsquare, grey, black);
filldraw(shift(2,0)*unitsquare, grey, black);
filldraw(shift(3,0)*unitsquare, grey, black);
filldraw(shift(3,1)*unitsquare, grey, black);
filldraw(shift(3,2)*unitsquare, grey, black);
\end{asy}
\qquad
\begin{asy}
unitsize(0.5 cm);
filldraw(unitsquare, white, black);
filldraw(shift(0,1)*unitsquare, white, black);
filldraw(shift(0,2)*unitsquare, white, black);
filldraw(shift(1,2)*unitsquare, white, black);
filldraw(shift(2,1)*unitsquare, white, black);
filldraw(shift(2,2)*unitsquare, white, black);

filldraw(shift(1,0)*unitsquare, grey, black);
filldraw(shift(1,1)*unitsquare, grey, black);
filldraw(shift(1,-1)*unitsquare, grey, black);
filldraw(shift(0,-1)*unitsquare, grey, black);
filldraw(shift(-1,-1)*unitsquare, grey, black);
filldraw(shift(-1,0)*unitsquare, grey, black);
\end{asy}
\end{center}

In particular, the total number of cells is divisible by $12$.
Thus the problem is reduced to proving that:
\begin{claim*}
  if a $6a \times 2b$ rectangle is tiled by tiles,
  then at least one of $a$ and $b$ is even.
\end{claim*}
\begin{proof}
  Note that the tiles can be further classified into two types:
  \begin{itemize}
    \ii \textbf{First type}: These tiles have exactly four columns,
    each with exactly three cells (an odd number).
    Moreover, all rows have an even number of cells (either $2$ or $4$).
    \ii \textbf{Second type}: vice-versa.
    These tiles have exactly four rows,
    each with exactly three cells (an odd number).
    Moreover, all rows have an odd number of cells.
  \end{itemize}
  We claim that any tiling uses an even number of each type, which is enough.

  By symmetry we prove an even number of first-type tiles.
  Color red every fourth column of the rectangle.
  The number of cells colored red is even.
  The tiles of the second type cover an even number of red cells,
  and the tiles of the first type cover an odd number of red cells.
  Hence the number of tiles of the first type must be even.
\end{proof}

\begin{remark*}
  This shows that a rectangle can be tiled by hooks
  if and only if it can be tiled by $3 \times 4$ rectangles.
  But there exist tilings which do not decompose into $3 \times 4$;
  see e.g.\ \url{https://aops.com/community/c6h14023p99881}.
\end{remark*}
\pagebreak

\section{Solutions to Day 2}