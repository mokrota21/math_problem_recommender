\textsl{Available online at \url{https://aops.com/community/p1936917}.}
\begin{mdframed}[style=mdpurplebox,frametitle={Problem statement}]
Each of the six boxes $B_1$, $B_2$, $B_3$, $B_4$, $B_5$, $B_6$
initially contains one coin.
The following two types of operations are allowed:
\begin{enumerate}
  \ii Choose a non-empty box $B_j$, $1\leq j \leq 5$,
  remove one coin from $B_j$ and add two coins to $B_{j+1}$;
  \ii Choose a non-empty box $B_k$, $1\leq k \leq 4$,
  remove one coin from $B_k$ and swap the contents
  (possibly empty) of the boxes $B_{k+1}$ and $B_{k+2}$.
\end{enumerate}
Determine if there exists a finite sequence of operations of the allowed types,
such that the five boxes $B_1$, $B_2$, $B_3$, $B_4$, $B_5$ become empty,
while box $B_6$ contains exactly $2010^{2010^{2010}}$ coins.
\end{mdframed}
First,
\begin{align*}
  (1,1,1,1,1,1) &\to (0,3,1,0,3,1) \to (0,0,7,0,0,7) \\
  &\to (0,0,6,2,0,7) \to (0,0,6,1,2,7) \to (0,0,6,1,0,11) \\
  &\to (0,0,6,0,11,0) \to (0,0,5,11,0,0).
\end{align*}
and henceforth we ignore boxes $B_1$ and $B_2$,
looking at just the last four boxes;
so we write the current position as $(5,11,0,0)$.

We prove a lemma:
\begin{claim*}
  Let $k \ge 0$ and $n > 0$.
  From $(k,n,0,0)$ we may reach $(k-1,2^n,0,0)$.
\end{claim*}
\begin{proof}
  Working with only the last three boxes for now,
  \begin{align*}
    (n,0,0) &\to (n-1, 2, 0) \to (n-1, 0, 4) \\
    &\to (n-2, 4, 0) \to (n-2, 0, 8) \\
    &\to (n-3, 8, 0) \to (n-3, 0, 16) \\
    &\to \dots \to (1, 2^{n-1}, 0) \to (1, 0, 2^n) \to (0, 2^n, 0).
  \end{align*}
  Finally we have $(k,n,0,0) \to (k,0,2^n,0) \to (k-1,2^n, 0,0)$.
\end{proof}

Now from $(5,11,0,0)$ we go as follows:
\begin{align*} (5,11,0,0) &\to (4, 2^{11}, 0, 0)
  \to \left(3, 2^{2^{11}}, 0, 0\right)
  \to \left(2, 2^{2^{2^{11}}}, 0, 0\right) \\
  &\to \left( 1, 2^{2^{2^{2^{11}}}}, 0, 0\right)
  \to \left(0, 2^{2^{2^{2^{2^{11}}}}}, 0, 0\right).
\end{align*}
Let $A = 2^{2^{2^{2^{2^{11}}}}} > 2010^{2010^{2010}} = B$.
Then by using move 2 repeatedly on the fourth box
(i.e., throwing away several coins by swapping the empty $B_5$ and $B_6$),
we go from $(0,A,0,0)$ to $(0,B/4,0,0)$.
From there we reach $(0,0,0,B)$.
\pagebreak