\textsl{Available online at \url{https://aops.com/community/p8639240}.}
\begin{mdframed}[style=mdpurplebox,frametitle={Problem statement}]
Fix $N \ge 1$. A collection of $N(N+1)$ soccer players of distinct
heights stand in a row.
Sir Alex Song wishes to remove $N(N-1)$ players from this row
to obtain a new row of $2N$ players in which the following $N$
conditions hold: no one stands between the two tallest players,
no one stands between the third and fourth tallest players, \dots,
no one stands between the two shortest players.
Prove that this is possible.
\end{mdframed}
Some opening remarks:
\textbf{location and height are symmetric to each other},
if one thinks about this problem as permutation pattern avoidance.
So while officially there are multiple solutions,
they are basically isomorphic to one another,
and I am not aware of any solution otherwise.

\begin{center}
\begin{asy}
size(10cm);
int[] ys = {7,11,2,5,0,10,9,0,12,1,6,8,4,3};
real r = 0.2;

fill(box((-1, 8.5),(4,4.5)), opacity(0.3)+lightred);
fill(box((-1,12.5),(7,8.5)), opacity(0.3)+lightcyan);
fill(box((-1, 4.5),(14,0.5)), opacity(0.3)+lightgreen);
fill(box((4, 8.5),(14,4.5)), opacity(0.3)+grey);
fill(box((7,12.5),(14,8.5)), opacity(0.3)+grey);

draw( (-1,4.5)--(14,4.5) );
draw( (-1,8.5)--(14,8.5) );

pair P(int x) {
  return (x,ys[x]);
}

pair O;
int y;
pen c;
for (int x=0; x<=13; ++x) {
  y = ys[x];
  if (y==0) continue;
  O = (x, y);
  if ((y==7) || (y==5)) c = red;
  else if ((y==10) || (y==9)) c = blue;
  else if ((y==1) || (y==4)) c = heavygreen;
  else c = grey;
  filldraw(circle(O, r), c, black+1);
  label("$"+(string)y+"$", O+(r,0), 0.5*dir(0), c);
}

pen border = black+2;
pen dash = dotted+1;
draw( (4,12.5)--(4,0.5), dash );
draw( (7,12.5)--(7,0.5), dash );

path bracket(real x0, real x1) {
  return (x0,0.7)--(x0,0.5)--(x1,0.5)--(x1,0.7);
}
draw(bracket(-0.8,3.8), border);
draw(bracket(4.2,6.8), border);
draw(bracket(7.2,13.8), border);
\end{asy}
\end{center}

Take a partition of $N$ groups in order by height:
$G_1 = \{1,\dots,N+1\}$, $G_2 = \{N+2, \dots, 2N+2\}$, and so on.
We will pick two people from each group $G_k$.

Scan from the left until we find two people in the same group $G_k$.
Delete all people scanned and also everyone in $G_k$.
All the groups still have at least $N$ people left,
so we can induct down with the non-deleted people;
the chosen pair is to the far left anyways.

\begin{remark*}
  The important bit is to \emph{scan by position}
  but \emph{group by height},
  and moreover not change the groups as we scan.
  Dually, one can have a solution which scans by height
  but groups by position.
\end{remark*}
\pagebreak