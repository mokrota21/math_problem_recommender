\textsl{Available online at \url{https://aops.com/community/p8639236}.}
\begin{mdframed}[style=mdpurplebox,frametitle={Problem statement}]
Let $R$ and $S$ be different points on a circle $\Omega$
such that $\ol{RS}$ is not a diameter.
Let $\ell$ be the tangent line to $\Omega$ at $R$.
Point $T$ is such that $S$ is the midpoint of $\ol{RT}$.
Point $J$ is chosen on minor arc $RS$ of $\Omega$ so that
the circumcircle $\Gamma$ of triangle $JST$ intersects $\ell$
at two distinct points.
Let $A$ be the common point of $\Gamma$ and $\ell$ closer to $R$.
Line $AJ$ meets $\Omega$ again at $K$.
Prove that line $KT$ is tangent to $\Gamma$.
\end{mdframed}
\paragraph{First solution (elementary).}
First, note
\[ \dang RKA = \dang RKJ = \dang RSJ = \dang TSJ = \dang TAJ = \dang TAK \]
so $\ol{RK} \parallel \ol{AT}$.
Now,
\begin{itemize}
  \ii $\ol{RA}$ is tangent at $R$ iff $\triangle KRS \sim \triangle RTA$ (oppositely),
  because both equate to $-\dang RKS = \dang SKR = \dang SRA = \dang TRA$.
  \ii Similarly, $\ol{TK}$ is tangent at $T$
  iff $\triangle KTS \sim \triangle ART$.
  \ii The two similarities are equivalent because $RS = ST$
  the SAS gives $KR \cdot TA = RS \cdot RT = TS \cdot TR$.
\end{itemize}

\begin{center}
\begin{asy}
size(10cm);
pair T = dir(100);
pair S = dir(165);
pair R = 2*S-T;
pair E = dir(0);
pair A = IP(unitcircle, R--(R+100*E));
pair B = OP(unitcircle, R--(R+100*E));
pair K = extension(T, T+dir(90)*T, R, R+T-A);
draw(R--B, blue);

filldraw(unitcircle, opacity(0.05)+lightcyan, blue);
filldraw(circumcircle(R, S, K), opacity(0.05)+paleblue, heavycyan);
pair J = -A+2*foot(origin, A, K);


filldraw(K--R--A--T--cycle, opacity(0.1)+lightred, red);
draw(A--K, orange);
draw(R--T, orange);

// invert

pair Js = extension(T, T+A-B, R, J);
pair Ks = extension(T, T+A-B, R, K);
draw(K--Ks, heavygreen);
draw(Ks--Js, heavygreen);
draw(Js--B, dotted+blue);
draw(R--Js, dotted+blue);

dot("$T$", T, dir(T));
dot("$S$", S, dir(355));
dot("$R$", R, dir(270));
dot("$A$", A, dir(225));
dot("$B$", B, dir(315));
dot("$K$", K, dir(200));
dot("$J$", J, dir(180));
dot("$J^\ast$", Js, dir(Js));
dot("$K^\ast$", Ks, dir(Ks));

/* TSQ Source:

!size(10cm);
T = dir 100
S = dir 165 R355
R = 2*S-T R270
E := dir 0
A = IP unitcircle R--(R+100*E) R225
B = OP unitcircle R--(R+100*E) R315
K = extension T T+dir(90)*T R R+T-A R200
T--K blue
R--B blue

unitcircle 0.05 lightcyan / blue
circumcircle R S K 0.05 paleblue / heavycyan
J = -A+2*foot origin A K R180


K--R--A--T--cycle 0.1 lightred / red
A--K orange
R--T orange

// invert

J* = extension T T+A-B R J
K* = extension T T+A-B R K
K--Ks heavygreen
Ks--Js heavygreen
Js--B dotted blue
R--Js dotted blue

*/
\end{asy}
\end{center}

\begin{remark*}
  The problem is actually symmetric with respect to two circles;
  $\ol{RA}$ is tangent at $R$ if and only if $\ol{TK}$ at $T$.
\end{remark*}

\paragraph{Second solution (inversion).}
Consider an inversion at $R$ fixing the circumcircle $\Gamma$ of $TSJA$.
Then:
\begin{itemize}
  \ii $T$ and $S$ swap,
  \ii $A$ and $B$ swap, where $B$ is the second intersection
  of $\ell$ with $\Gamma$.
  \ii Circle $\Omega$ inverts to the line through $T$
  parallel to $\ol{RAB}$, call it $\ell$.
  \ii $J^\ast$ is the second intersection of $\ell$ with $\Gamma$.
  \ii $K^\ast$ is the intersection of $\ell$ with the circumcircle
  of $RBJ^\ast$; this implies $RK^\ast J^\ast B$ is an isosceles trapezoid.
  In particular, one reads $\ol{RK^\ast} \parallel \ol{AT}$ from this,
  hence $RK^\ast TA$ is a parallelogram.
\end{itemize}
Thus we wish to show the circumcircle of $RSK^\ast$ is tangent to $\Gamma$.
But that follows from the final parallelogram observed:
$S$ is the center of the parallelogram since it is the midpoint of the diagonal.

\begin{remark*}
  This also implies $RKTB$ is cyclic, from $\ol{K^\ast SA}$ collinear.
  Moreover, quadrilateral $KK^\ast TS$ is cyclic (by power of a point);
  this leads to the second official solution to the problem.
\end{remark*}
\pagebreak