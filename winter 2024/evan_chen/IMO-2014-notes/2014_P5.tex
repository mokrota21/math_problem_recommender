\textsl{Available online at \url{https://aops.com/community/p3543144}.}
\begin{mdframed}[style=mdpurplebox,frametitle={Problem statement}]
For every positive integer $n$,
the Bank of Cape Town issues coins of denomination $\frac 1n$.
Given a finite collection of such coins (of not necessarily different denominations)
with total value at most $99 + \frac12$, prove that it is possible to split
this collection into $100$ or fewer groups, such that each group has total value at most $1$.
\end{mdframed}
We'll prove the result
for at most $k - \frac{k}{2k+1}$ with $k$ groups.
First, perform the following optimizations.
\begin{itemize}
  \ii If any coin of size $\frac{1}{2m}$ appears twice,
  then replace it with a single coin of size $\frac{1}{m}$.
  \ii If any coin of size $\frac{1}{2m+1}$ appears $2m+1$ times,
  group it into a single group and induct downwards.
\end{itemize}
Apply this operation repeatedly until it cannot be done anymore.

Now construct boxes $B_0$, $B_1$, \dots, $B_{k-1}$.
In box $B_0$ put any coins of size $\tfrac 12$ (clearly there is at most one).
More generally for $m \ge 0$, $B_m$, put coins of size
$\frac{1}{2m+1}$ and $\frac{1}{2m+2}$
(there at most $2m$ of the former and at most one of the latter).
Note that
\[ \text{total weight in $B_m$} \le 2m \cdot \frac{1}{2m+1} + \frac{1}{2m+2} < 1. \]
Finally, place the remaining ``light'' coins of size at most $\frac{1}{2k+1}$ in a pile.
\begin{remark*}
  One way to explain where this boxing comes from is to imagine
  what happens after applying the operation repeatedly until it can't be done any more.
  We expect the most troublesome situation would be if the leftover coins are
  as large as possible, which looks like
  \[ \half, \qquad \frac13 \times 2, \qquad \frac14, \qquad \frac15 \times 4,
    \qquad \frac16, \qquad \frac17 \times 6, \qquad \frac 18, \qquad \dots \]
  Seeing this the boxing $B_i$ is quite reasonable because it groups these
  into groups of almost $1$ without too much effort:
  \begin{align*}
    B_0 &\longrightarrow \half \\
    B_1 &\longrightarrow \frac13 \cdot 2 + \frac 14 = \frac{11}{12} \\
    B_2 &\longrightarrow \frac15 \cdot 4 + \frac 16 = \frac{29}{30} \\
    B_3 &\longrightarrow \frac17 \cdot 6 + \frac 18 = \frac{55}{56} \\
    &\vdotswithin\longrightarrow
  \end{align*}
\end{remark*}

Then just toss coins from the pile into the boxes arbitrarily,
other than the proviso that no box should have its weight exceed $1$.
We claim this uses up all coins in the pile.
Assume not, and that some coin remains in the pile
when all the boxes are saturated.
Then all the boxes must have at least $1 -\frac{1}{2k+1}$,
meaning the total amount in the boxes is strictly greater than
\[ k \left( 1 - \frac{1}{2k+1} \right) > k - \tfrac 12 \]
which is a contradiction.
\begin{remark*}
  This gets a stronger bound $k - \frac{k}{2k+1}$ than the requested $k-\tfrac 12$.
\end{remark*}
\pagebreak