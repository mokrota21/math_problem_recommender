\textsl{Available online at \url{https://aops.com/community/p25635154}.}
\begin{mdframed}[style=mdpurplebox,frametitle={Problem statement}]
Let $ABCDE$ be a convex pentagon such that $BC=DE$.
Assume that there is a point $T$ inside $ABCDE$
with $TB=TD$, $TC=TE$ and $\angle ABT = \angle TEA$.
Let line $AB$ intersect lines $CD$ and $CT$ at points $P$ and $Q$, respectively.
Assume that the points $P$, $B$, $A$, $Q$ occur on their line in that order.
Let line $AE$ intersect $CD$ and $DT$ at points $R$ and $S$, respectively.
Assume that the points $R$, $E$, $A$, $S$ occur on their line in that order.
Prove that the points $P$, $S$, $Q$, $R$ lie on a circle.
\end{mdframed}
The conditions imply
\[ \triangle BTC \cong \triangle DTE,
  \qquad\text{and}\qquad
  \triangle BTY \overset{-}{\sim} \triangle ETX. \]
Define $K = \ol{CT} \cap \ol{AE}$, $L = \ol{DT} \cap \ol{AB}$,
$X = \ol{BT} \cap \ol{AE}$, $Y = \ol{ET} \cap \ol{BY}$.

\begin{center}
\begin{asy}
size(12cm);
pair Y = dir(116.9642725);
pair X = dir(76.9642725);
pair B = dir(175.9642725);
pair E = dir(24.9642725);
pair A = extension(X, E, B, Y);
pair T = extension(X, B, E, Y);
pair D = T+(B-T)*dir(83);
pair C = T+(E-T)*dir(-83);
pair Q = extension(C, T, A, B);
pair S = extension(D, T, A, E);
pair P = extension(C, D, A, B);
pair R = extension(C, D, A, E);
pair L = extension(D, T, A, B);
pair K = extension(C, T, A, E);

filldraw(B--T--C--cycle, opacity(0.1)+yellow, 0.1+yellow);
filldraw(D--T--E--cycle, opacity(0.1)+yellow, 0.1+yellow);

filldraw(B--T--Q--cycle, opacity(0.1)+lightred, red);
filldraw(E--T--S--cycle, opacity(0.1)+lightcyan, blue);
draw(T--X, blue);
draw(T--Y, red);
draw(circumcircle(L, K, Q), grey);
draw(circumcircle(P, Q, R), dotted);
draw(T--D, red+dashed);
draw(T--C, blue+dashed);
draw(K--L, grey);
draw(B--P--R--E, grey);
draw(B--C, dotted);
draw(D--E, dotted);
write(angle(D-C));

dot("$Y$", Y, dir(Y));
dot("$X$", X, dir(60));
dot("$B$", B, dir(B));
dot("$E$", E, dir(E));
dot("$A$", A, dir(A));
dot("$T$", T, 1.8*dir(280));
dot("$D$", D, dir(270));
dot("$C$", C, dir(270));
dot("$Q$", Q, dir(Q));
dot("$S$", S, dir(S));
dot("$P$", P, dir(P));
dot("$R$", R, dir(R));
dot("$L$", L, 1.4*dir(200));
dot("$K$", K, 1.4*dir(355));

/* TSQ Source:

!size(12cm);
Y = dir 116.9642725
X = dir 76.9642725 R60
B = dir 175.9642725
E = dir 24.9642725
A = extension X E B Y
T = extension X B E Y 1.8R280
D = T+(B-T)*dir(83) R270
C = T+(E-T)*dir(-83) R270
Q = extension C T A B
S = extension D T A E
P = extension C D A B
R = extension C D A E
L = extension D T A B 1.4R200
K = extension C T A E 1.4R355

B--T--C--cycle 0.1 yellow / 0.1 yellow
D--T--E--cycle 0.1 yellow / 0.1 yellow

B--T--Q--cycle 0.1 lightred / red
E--T--S--cycle 0.1 lightcyan / blue
T--X blue
T--Y red
circumcircle L K Q grey
circumcircle P Q R dotted
T--D red dashed
T--C blue dashed
K--L grey
B--P--R--E grey
B--C dotted
D--E dotted
!write(angle(D-C));

*/
\end{asy}
\end{center}

\begin{claim*}
  [Main claim]
  We have
  \[ \triangle BTQ \overset{-}{\sim} \triangle ETS,
    \qquad\text{and}\qquad
    BY:YL:LQ = EX:XK:KS. \]
  In other words, $TBYLQ \overset{-}{\sim} TEXKS$.
\end{claim*}
\begin{proof}
  We know $\triangle BTY \overset{-}{\sim} \triangle ETX$.
  Also, $\dang BTL = \dang BTD = \dang CTE = \dang KTE$
  and $\dang BTQ = \dang BTC = \dang DTE = \dang STE$.
\end{proof}

It follows from the claim that:
\begin{itemize}
  \ii $TL/TQ = TK/TS$, ergo $TL \cdot TS = TK \cdot TQ$,
  so $KLSQ$ is cyclic; and
  \ii $TC/TK = TE/TK = TB/TL = TD/TL$, so $\ol{KL} \parallel \ol{PCDR}$.
\end{itemize}
With these two bullets, we're done by Reim theorem.
\pagebreak