\textsl{Available online at \url{https://aops.com/community/p131833}.}
\begin{mdframed}[style=mdpurplebox,frametitle={Problem statement}]
A set $S$ of points from the space will be called
completely symmetric if it has at least three elements
and fulfills the condition that for every two distinct points
$A$ and $B$ from $S$,
the perpendicular bisector plane of the segment $AB$
is a plane of symmetry for $S$.
Prove that if a completely symmetric set is finite,
then it consists of the vertices of either a regular polygon,
or a regular tetrahedron or a regular octahedron.
\end{mdframed}
Let $G$ be the centroid of $S$.

\begin{claim*}
  All points of $S$ lie on a sphere $\Gamma$ centered at $G$.
\end{claim*}
\begin{proof}
  Each perpendicular bisector plane passes through $G$.
  So if $A,B \in S$ it follows $GA = GB$.
\end{proof}

\begin{claim*}
  Consider any plane passing through three or more points of $S$.
  The points of $S$ in the plane form a regular polygon.
\end{claim*}
\begin{proof}
  The cross section is a circle because we are intersecting
  a plane with sphere $\Gamma$.
  Now if $A$, $B$, $C$ are three adjacent points on this circle,
  by taking the perpendicular bisector we have $AB=BC$.
\end{proof}

If the points of $S$ all lie in a plane, we are done.
Otherwise, the points of $S$ determine a polyhedron
$\Pi$ inscribed in $\Gamma$.
All of the faces of $\Pi$ are evidently regular polygons,
of the same side length $s$.

\begin{claim*}
  Every face of $\Pi$ is an equilateral triangle.
\end{claim*}
\begin{proof}
  Suppose on the contrary some face $A_1 A_2 \dots A_n$
  has $n > 3$.
  Let $B$ be any vertex adjacent to $A_1$ in $\Pi$
  other than $A_2$ or $A_n$.
  Consider the plane determined by $\triangle A_1 A_3 B$.
  This is supposed to be a regular polygon,
  but arc $A_1 A_3$ is longer than arc $A_1 B$,
  and by construction there are no points inside these arcs.
  This is a contradiction.
\end{proof}

Hence, $\Pi$ has faces all congruent equilateral triangles.
This implies it is a regular polyhedron --- either
a regular tetrahedron, regular octahedron,
or regular icosahedron.
We can check the regular icosahedron fails by
taking two antipodal points as our counterexample.
This finishes the problem.
\pagebreak