\textsl{Available online at \url{https://aops.com/community/p611}.}
\begin{mdframed}[style=mdpurplebox,frametitle={Problem statement}]
An $n \times n$ matrix whose entries come
from the set $S = \{1, 2, \dots , 2n - 1\}$
is called a \emph{silver} matrix if,
for each $i = 1, 2, \dots , n$,
the $i$-th row and the $i$-th column together
contain all elements of $S$. Show that:
\begin{enumerate}[(a)]
\ii there is no silver matrix for $n = 1997$;
\ii silver matrices exist for infinitely many values of $n$.
\end{enumerate}
\end{mdframed}
For (a), define a \emph{cross} to be the union
of the $i$th row and $i$th column.
Every cell of the matrix not on the diagonal is
contained in exactly two crosses,
while each cell on the diagonal is contained in one cross.

On the other hand, if a silver matrix existed for $n=1997$,
then each element of $S$ is in all $1997$ crosses,
so it must appear at least once on the diagonal since $1997$ is odd.
However, $|S| = 3993$ while there are only $1997$ diagonal cells.
This is a contradiction.

For (b), we construct a silver matrix $M_e$ for $n = 2^e$ for each $e \ge 1$.
We write the first three explicitly for concreteness:
\begin{align*}
  M_1 &= \begin{bmatrix}
    1 & 2 \\ 3 & 1
  \end{bmatrix} \\
  M_2 &= \begin{bmatrix}
    {\color{red}1} & {\color{red}2} & 4 & 5 \\
    {\color{red}3} & {\color{red}1} & 6 & 7 \\
    7 & 5 & {\color{red}1} & {\color{red}2} \\
    6 & 4 & {\color{red}3} & {\color{red}1}
  \end{bmatrix} \\
  M_3 &= \begin{bmatrix}
    {\color{red}1} & {\color{red}2} & {\color{red}4} & {\color{red}5} & 8 & 9 & 11 & 12\\
    {\color{red}3} & {\color{red}1} & {\color{red}6} & {\color{red}7} & 10 & 15 & 13 & 14 \\
    {\color{red}7} & {\color{red}5} & {\color{red}1} & {\color{red}2} & 14 & 12 & 8 & 9 \\
    {\color{red}6} & {\color{red}4} & {\color{red}3} & {\color{red}1} & 13 & 11 & 10 & 15 \\
    15 & 9 & 11 & 12 & {\color{red}1} & {\color{red}2} & {\color{red}4}
    & {\color{red}5} \\
    10 & 8 & 13 & 14 & {\color{red}3} & {\color{red}1} & {\color{red}6}
    & {\color{red}7} \\
    14 & 12 & 15 & 9 & {\color{red}7} & {\color{red}5} & {\color{red}1}
    & {\color{red}2} \\
    13 & 11 & 10 & 8 & {\color{red}6} & {\color{red}4} & {\color{red}3}
    & {\color{red}1} \\
  \end{bmatrix}
\end{align*}
The construction is described recursively as follows.
Let
\[
  M_e' = \left[
  \begin{array}{c|c}
    {\color{red}M_{e-1}} & M_{e-1} + (2^e-1) \\ \hline
    M_{e-1} + (2^e-1) & {\color{red}M_{e-1}} \\
  \end{array}
  \right].
\]
Then to get from $M_e'$ to $M_e$,
replace half of the $2^e$'s with $2^{e+1}-1$:
in the northeast quadrant, the even-indexed ones,
and in the southwest quadrant, the odd-indexed ones.

\begin{remark*}
  In fact, it turns out silver matrices exist for all even dimensions.
  A claimed proof is outlined at \url{https://aops.com/community/p7375020}.
\end{remark*}
\pagebreak