\textsl{Available online at \url{https://aops.com/community/p131838}.}
\begin{mdframed}[style=mdpurplebox,frametitle={Problem statement}]
Two circles $\Omega_{1}$ and $\Omega_{2}$ touch internally the circle
$\Omega$ in $M$ and $N$ and the center of $\Omega_{2}$ is on $\Omega_{1}$.
The common chord of the circles $\Omega_{1}$ and $\Omega_{2}$
intersects $\Omega$ in $A$ and $B$
Lines $MA$ and $MB$ intersects $\Omega_{1}$ in $C$ and $D$.
Prove that $\Omega_{2}$ is tangent to $CD$.
\end{mdframed}
Let $P$ and $Q$ be the centers of $\Omega_1$ and $\Omega_2$.

Let line $MQ$ meet $\Omega_1$ again at $W$,
the homothetic image of $Q$ under $\Omega_1 \to \Omega$.

Meanwhile, let $T$ be the intersection of segment $PQ$
with $\Omega_2$, and let $L$ be its homothetic image on $\Omega$.
Since $\ol{PTQ} \perp \ol{AB}$, it follows $\ol{LW}$
is a diameter of $\Omega$.
Let $O$ be its center.

\begin{center}
\begin{asy}
size(10cm);
pen xfqqff = rgb(0.49803,0.,1.); pen qqffff = rgb(0.,1.,1.); pen cqcqcq = rgb(0.75294,0.75294,0.75294);
pair O = (0.,0.), M = (-0.54799,0.83648), P = (-0.16957,0.25884), Q = (-0.16861,-0.43170), A = (-0.97449,-0.22439), B = (0.97512,-0.22166), C = (-0.84251,0.10388), D = (0.50379,0.10577), T = (-0.16936,0.10483), L = (-0.00139,0.99999);
draw(circle(O, 1.), linewidth(0.6));
draw(circle(P, 0.69055), linewidth(0.6) + xfqqff);
draw(circle(Q, 0.53653), linewidth(0.6) + xfqqff);
draw(O--M, linewidth(0.6) + red);
draw(M--A, linewidth(0.6) + qqffff);
draw(M--B, linewidth(0.6) + qqffff);
draw(O--(-0.36380,-0.93147), linewidth(0.6) + red);
draw(P--Q, linewidth(0.6));
draw(M--(0.00139,-0.99999), linewidth(0.6) + green);
draw((0.00139,-0.99999)--L, linewidth(0.6));
draw(L--(-0.36380,-0.93147), linewidth(0.6) + green);
draw(M--(-0.36380,-0.93147), linewidth(0.6));
draw(A--B, linewidth(0.6));
dot("$O$", O, dir((1.868, -2.897)));
dot("$N$", (-0.36380,-0.93147), dir((-5.700, -12.058)));
dot("$M$", M, dir((-8.656, 10.478)));
dot("$P$", P, dir((-2.789, 6.521)));
dot("$Q$", Q, dir((3.600, -6.558)));
dot("$A$", A, dir((-15.478, -5.915)));
dot("$B$", B, dir((4.982, -4.507)));
dot("$C$", C, dir((-8.023, -3.199)));
dot("$D$", D, dir((1.442, 0.454)));
dot("$T$", T, dir((-7.133, 1.989)));
dot("$W$", (0.00139,-0.99999), dir((-2.113, -15.052)));
dot("$L$", L, dir((1.046, 1.812)));
dot("$E$", midpoint(A--B), dir(-45));
\end{asy}
\end{center}

\begin{claim*}
  $MNTQ$ is cyclic.
\end{claim*}
\begin{proof}
  By Reim: $\dang TQM = \dang LWM = \dang LNM = \dang TNM$.
\end{proof}

Let $E$ be the midpoint of $\ol{AB}$.
\begin{claim*}
  $OEMN$ is cyclic.
\end{claim*}
\begin{proof}
  By radical axis, the lines $MM$, $NN$, $AEB$ meet at a point $R$.
  Then $OEMN$ is on the circle with diameter $\ol{OR}$.
\end{proof}

\begin{claim*}
  $MTE$ are collinear.
\end{claim*}
\begin{proof}
  $\dang NMT = \dang TQN = \dang LON = \dang NOE = \dang NME$.
\end{proof}

Now consider the homothety mapping $\triangle WAB$
to $\triangle QCD$.
It should map $E$ to a point on line $ME$
which is also on the line through $Q$ perpendicular to $\ol{AB}$;
that is, to point $T$.
Hence $TCD$ are collinear,
and it's immediate that $T$ is the desired tangency point.
\pagebreak