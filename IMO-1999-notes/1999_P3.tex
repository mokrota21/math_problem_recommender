\textsl{Available online at \url{https://aops.com/community/p131873}.}
\begin{mdframed}[style=mdpurplebox,frametitle={Problem statement}]
Let $n$ be an even positive integer.
Find the minimal number of cells on the $n \times n$ board
that must be marked so that any cell
(marked or not marked) has a marked neighboring cell.
\end{mdframed}
For every marked cell, consider the marked cell adjacent
to it; in this way we have a \emph{domino} of two cells.
For each domino, its \emph{aura} consists of all the cells
which are adjacent to a cell of the domino.
There are up to eight squares in each aura,
but some auras could be cut off by the boundary of the board,
which means that there could be as few as five squares.

We will prove that $\half n (n+2)$
is the minimum number of auras
needed to cover the board
(the auras need not be disjoint).
\begin{itemize}
  \ii A construction is shown on the left below,
  showing that $\half n (n+2)$.
  \ii Color the board as shown to the right into ``rings''.
  Every aura takes covers exactly (!) four blue cells.
  Since there are $2n(n+2)$ blue cells, this implies the lower bound.
\end{itemize}

\begin{center}
\begin{asy}
size(11cm);

path aura = (1,0)--(2,0)--(2,1)--(3,1)--(3,3)--(2,3)--(2,4)--(1,4)--(1,3)--(0,3)--(0,1)--(1,1)--cycle;
picture aura_up(pen fill) {
  picture pic = new picture;
  fill(pic, aura, fill);
  draw(pic, (1,1)--(2,1), lightgrey );
  draw(pic, (1,3)--(2,3), lightgrey );
  draw(pic, (0,2)--(3,2), lightgrey );
  draw(pic, (1,1)--(1,3), lightgrey );
  draw(pic, (2,1)--(2,3), lightgrey );
  draw(pic, aura, blue+1.25);
  return pic;
}

picture aura_right(pen fill) {
  return shift(0,3) * rotate(-90) * aura_up(fill);
}

add(shift(-1,-1)*aura_up(palegreen));
add(shift(-1,3)*aura_up(palegreen));
add(shift(-1,7)*aura_up(palegreen));
add(shift(1,1)*aura_up(palegreen));
add(shift(1,5)*aura_up(palegreen));
add(shift(3,3)*aura_up(palegreen));
add(shift(6,3)*aura_up(palegreen));
add(shift(8,1)*aura_up(palegreen));
add(shift(8,5)*aura_up(palegreen));
add(shift(10,-1)*aura_up(palegreen));
add(shift(10,3)*aura_up(palegreen));
add(shift(10,7)*aura_up(palegreen));

add(shift(4,6)*aura_right(palered));
add(shift(2,8)*aura_right(palered));
add(shift(6,8)*aura_right(palered));
/*
add(shift(0,10)*aura_right(palered));
add(shift(4,10)*aura_right(palered));
add(shift(8,10)*aura_right(palered));
*/
add(shift(4,1)*aura_right(palered));
add(shift(2,-1)*aura_right(palered));
add(shift(6,-1)*aura_right(palered));
add(shift(0,-3)*aura_right(palered));
add(shift(4,-3)*aura_right(palered));
add(shift(8,-3)*aura_right(palered));

draw(shift(0,-2)*scale(12)*unitsquare, black+2);

path cell = shift(16,-2)*unitsquare;
for (int i = 0; i <= 11; ++i) {
for (int j = 0; j <= 11; ++j) {
  if (max(abs(i-5.5), abs(j-5.5)) == 5.5)
    filldraw(shift(i,j)*cell, paleblue, grey);
  else if (max(abs(i-5.5), abs(j-5.5)) == 3.5)
    filldraw(shift(i,j)*cell, paleblue, grey);
  else if (max(abs(i-5.5), abs(j-5.5)) == 1.5)
    filldraw(shift(i,j)*cell, paleblue, grey);
  else draw(shift(i,j)*cell, grey);
}
}
\end{asy}
\end{center}

Note that this proves that a partition into disjoint auras
actually always has exactly $\half n (n+2)$ auras,
thus also implying EGMO 2019/2.
\pagebreak

\section{Solutions to Day 2}